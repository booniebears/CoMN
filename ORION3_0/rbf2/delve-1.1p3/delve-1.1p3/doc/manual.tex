%
% $Id: manual.tex,v 1.52.2.3 1996/12/04 19:04:45 revow Exp $
%

\documentclass[12pt]{article}
\usepackage{alltt}
\usepackage{float}
\usepackage{fancyheadings}
\usepackage{mystyle}
% \usepackage{html}
% \usepackage{heqn}
\usepackage{epsf}

% \includeonly{appC}

%
% Resize the area for text on a page.
%

\setlength{\topmargin}{-0.2in}
\setlength{\headheight}{0.2in}
\setlength{\headsep}{0.2in}
\setlength{\textheight}{8.7in}
\setlength{\textwidth}{6.5in}
\setlength{\oddsidemargin}{0.0in}
\setlength{\evensidemargin}{0.0in}

%
% Fancy headings with the string ``DELVE'' on one side of the header,
% and the section number and name on the other. The page number is in
% the center of the footer.
%

\lhead[]{}
\rhead[]{}
\chead[]{}

\pagestyle{fancyplain}

%
% The level number of the least significant sectional unit listed in
% the table of contents. 2 = subsections
%

\setcounter{tocdepth}{2}

%
% Let floats take up *a lot* of space on a page.
%

\renewcommand{\topfraction}{0.8}
\renewcommand{\bottomfraction}{0.8}
\renewcommand{\textfraction}{0.2}

%
% \begin{Session} body \end{Session}
%
% An environment for displaying an example of a command session. Text
% is printed in typewriter font with a slightly smaller point size
% than the default point size.  Text formatting is unchanged as in
% ``verbatim'', but everything is indented by the current listing
% indentation. 
%

\newenvironment{Session}%
    {\begin{list}{}{}\item[]\small\begin{minipage}{\linewidth}\begin{alltt}}%
    {\end{alltt}\end{minipage}\end{list}}

%
% Strings that appear many times in the document.
%

\newcommand{\Dinfo}{{\rm\tt Dinfo}}
\newcommand{\OR}{{\em ~or~~}}
\newcommand{\delvepath}{{\rm\tt DELVE\_PATH}}
\newcommand{\delve}{DELVE}
\newcommand{\dinfo}{{\rm\tt dinfo}}
\newcommand{\dls}{{\rm\tt dls}}
\newcommand{\dmore}{{\rm\tt dmore}}
\newcommand{\dcheck}{{\rm\tt dcheck}}
\newcommand{\dgenproto}{{\rm\tt dgenproto}}
\newcommand{\dgenorder}{{\rm\tt dgenorder}}
\newcommand{\file}[2]{{\rm\tt #1}{\it .#2\/}}
\newcommand{\mgendata}{{\rm\tt mgendata}}
\newcommand{\mgendir}{{\rm\tt mgendir}}
\newcommand{\minfo}{{\rm\tt minfo}}
\newcommand{\mloss}{{\rm\tt mloss}}
\newcommand{\mls}{{\rm\tt mls}}
\newcommand{\mmore}{{\rm\tt mmore}}
\newcommand{\mstats}{{\rm\tt mstats}}
\newcommand{\mtable}{{\rm\tt mtable}}
\newcommand{\wwwhome}{http://www.cs.utoronto.ca/\maketilde{}delve/}
\newcommand{\maketilde}{\raisebox{0.4ex}{\scriptsize $\sim$}}

\def\beq{\begin{eqnarray}}
\def\eeq{\end{eqnarray}}
\def\eep{\end{eqnarray}}

\begin{document}

% ----------------------------------------------------------------------
% The title page
% ----------------------------------------------------------------------

\begin{center}
   \vspace*{1in}
   {\huge The DELVE Manual} \\[26pt]

   {\large C.~E.~Rasmussen, R.~M.~Neal, G.~E.~Hinton, D.~van~Camp,} \\[5pt]
   {\large M.~Revow, Z.~Ghahramani, R. Kustra, and R.~Tibshirani}   \\[20pt]

   {\large Version 1.1} \\
   December 1996
\end{center}

\vfill

\begin{list}{}{
\setlength{\leftmargin}{0.25in}\setlength{\rightmargin}{0.25in}}\item[]\em
    This manual describes the preliminary release of the DELVE environment.
    Some features described here have not yet implemented, as noted.
    Support for regression tasks is presently somewhat more developed than 
    that for classification tasks.\vspace{3pt}

    We recommend that you exercise caution when using this version of 
    DELVE for real work, as it is possible that bugs remain in the software.
    We hope that you will send us reports of any problems you encounter,
    as well as any other comments you may have on the software or manual, 
    at the e-mail address below.  Please mention the version number of
    the manual and/or the software with any comments you send.
\end{list}

\vfill\vfill

\begin{centering}

For the latest DELVE news, visit \wwwhome\\[4pt]
Send comments to delve@cs.utoronto.ca

\end{centering}\vspace{8pt}

\noindent This project was supported by grants from the Natural Sciences and
Engineering Research Council of Canada and the Institute for
Robotics and Intelligent Systems.

%
%  page style has to come after ``\maketitle'' because it tries to
%  make the style ``plain''.
%

\pagenumbering{roman}
\thispagestyle{empty}

\clearpage

% ----------------------------------------------------------------------
% The copyright notice
% ----------------------------------------------------------------------

\thispagestyle{plain}

\vspace*{\fill}
\vspace*{-0.8in}
\begin{centering}
Copyright \copyright\ 1995--1996 by The University of Toronto,\\
Toronto, Ontario, Canada.\\[1ex]
All Rights Reserved\\[2ex]
\end{centering}

\noindent
Permission to use, copy, modify, and distribute this software and its
documentation for {\bf non-commercial purposes only} is hereby granted
without fee, provided that the above copyright notice appears in all
copies and that both the copyright notice and this permission notice
appear in supporting documentation, and that the name of The
University of Toronto not be used in advertising or publicity
pertaining to distribution of the software without specific, written
prior permission.  The University of Toronto makes no representations
about the suitability of this software for any purpose.  It is
provided ``as is'' without express or implied warranty.\vspace{7pt}

{\bf\noindent
The University of Toronto disclaims all warranties with regard to this
software, including all implied warranties of merchantability and
fitness.  In no event shall the University of Toronto be liable for
any special, indirect or consequential damages or any damages
whatsoever resulting from loss of use, data or profits, whether in an
action of contract, negligence or other tortious action, arising out
of or in connection with the use or performance of this software.  
}\vspace{0.4in}

\noindent If you publish results obtained using \delve{}, please cite
this manual, and mention the version number of the software that you
used.

\vspace*{\fill}
\clearpage 

% ----------------------------------------------------------------------
% The table of contents and list of figures.
% ----------------------------------------------------------------------

\renewcommand{\contentsname}{\hfill CONTENTS \hfill}
\pagestyle{plain}
\thispagestyle{plain}
\setlength{\parskip}{3.3pt}
\vfill
\tableofcontents
\vfill
\clearpage

% ----------------------------------------------------------------------
% The body of the document.
% ----------------------------------------------------------------------

%
% Change paragraphs so the first line is *not* indented, and each
% paragraph is separated by 8 points. Don't forget to turn off the
% extra vertical space that would normally be added to the start of an
% environment if it starts its own paragraph (\partopsep).
%

\setlength{\parskip}{8pt}	% Extra vertical space between paragraphs.
\setlength{\parindent}{0em}	% Width of paragraph indentation.
\setlength{\partopsep}{0pt}	% Extra vertical space, in addition to 
				% \parskip and \topsep, added when user
				% leaves blank line before environment.
\setlength{\topsep}{0pt}
\raggedbottom

\pagenumbering{arabic}
\setcounter{page}{1}

\pagestyle{fancyplain}

%
% $Id: sec1.tex,v 1.45 1996/05/15 15:39:54 radford Exp $
%
\newpage

\section{INTRODUCTION}\label{sec-intro}
\thispagestyle{plain}
\setcounter{figure}{0}
\chead[\fancyplain{}{\thesection.\ INTRODUCTION}]
      {\fancyplain{}{\thesection.\ INTRODUCTION}}

\delve{} --- Data for Evaluating Learning in Valid Experiments --- is
a collection of datasets from many sources, an environment within
which this data can be used to assess the performance of methods for
learning relationships from data, and a repository for the results of
such assessments.

Many methods for learning relationships from empirical data have been
developed by researchers in statistics, pattern recognition,
artificial intelligence, neural networks, and other fields.  Methods
in common use include simple linear models, nearest neighbor methods,
decision trees, multilayer perceptron networks, and many others of
varying degrees of complexity.  Properly comparing the performance of
these learning methods in realistic contexts is a surprisingly
difficult task, requiring both an extensive collection of real-world
data, and a carefully-designed scheme for performing experiments.

The aim of \delve{} is to help researchers and potential users to
assess learning methods in a way that is relevant to real-world
problems and that allows for statistically-valid comparisons of
different methods.  Improved assessments will make it easier to
determine which methods work best for various applications, and will
promote the development of better learning methods by allowing
researchers to easily determine how the performance of a new method
compares to that of existing methods.

This manual describes the \delve{} environment in detail.  First,
however, we provide an overview of \delve's capabilities, describe
briefly how \delve{} organizes datasets, methods, and learning tasks,
and give an example of how \delve{} can be used to assess the
performance of a learning method.


\subsection{What \delve{} can do for you}\label{intro-can-do}

\delve{} can help you assess the performance of learning methods
in three major ways:\vspace{-4pt}
\begin{enumerate}
\item The \delve{} archive contains a collection of many datasets that 
      are appropriate for developing and assessing learning methods.
\item The \delve{} software helps you use this data to assess
      learning methods. \delve{} also provides guidelines on how such 
      assessments should be done.
\item The \delve{} archive also records the results of assessing many
      other learning methods on the same datasets, performed in the same way, 
      along with detailed descriptions of these methods.\vspace{-4pt}
\end{enumerate}
All these components of the \delve{} environment are freely available
via the Web, at URL
\begin{center}
  \texttt{\wwwhome}
\end{center}

If you are interested in using a learning method for a particular
application, you may find \delve{} useful in determining which methods
might be appropriate for you to use.  To do this, you would
look at the results of various learning methods on problems that
appear similar to your application.  For those methods that seem 
promising, you could refer to the detailed descriptions recorded in
the \delve{} archive.

If you are a researcher developing new learning methods, you will no
doubt wish to know how well the methods you develop compare in
performance to existing methods. \delve{} can help you answer this
question by providing a large number of datasets to test on, by
providing standard conventions for conducting experiments that
facilitate comparisons, by providing the results of other methods on
the same datasets, and by performing appropriate tests of the
statistical significance of the observed differences in performance.

After using \delve{} to assess a novel learning method, you can submit
your results for inclusion in the \delve{} archive. You should provide a
detailed description of your method, and include results of applying
your method to a selection of datasets. In this way other researchers
and users of learning methods will be able to benefit from your work.

 
\subsection{The \delve{} hierarchy of data, methods, and 
            results}\label{intro-hierarchy}

\delve{} organizes data, learning methods, and experimental results in
a hierarchical fashion.  This section provides an informal description
of this hierarchy, sufficient for you to follow the example in the next
section.

The \delve{} hierarchy is contained within one or more top-level
directories, each of which has a name starting with the five letters
``\texttt{delve}''.  All such \texttt{delve} directories contain two
sub-directories, corresponding to the two main divisions of the
\delve{} hierarchy.  The \texttt{data} sub-directory contains information
on datasets, and on learning tasks defined for these datasets.  The
\texttt{methods} sub-directory contains information on learning methods,
and on the results of applying these methods to various learning
tasks.

By using more than one top-level \texttt{delve} directory, you can keep
datasets and results that come from the \delve{} archive separate from
datasets and results that you are working on yourself, but have not
yet submitted to the archive.  Some research groups may also find
it convenient to maintain a group \texttt{delve} directory, in addition
to the private \texttt{delve} directories of the group members.

When you use \delve{}, you will see information on data and methods
from all such \texttt{delve} directories that are currently active,
merged into a single hierarchy.  In the rest of this section, we will
for simplicity describe this hierarchy as if was contained in a single
directory.  

The data part of the hierarchy begins with a number of {\it datasets},
each of which has its own sub-directory within the \texttt{data}
directory.  A dataset is a list of cases, with each case consisting
of values for a number of attributes.  Some additional information is
also specified at the dataset level, such as names and ranges for
attributes.

\emph{Prototasks} are the next lower level in the data hierarchy.  A
prototask defines which cases in the dataset are relevant to the
learning task, which attributes of a case we wish to predict (the
target attributes), and which attributes we wish to predict the targets
from (the input attributes). There may be several different prototasks
for a dataset, each of which has a sub-directory within the dataset's
directory.

At the \emph{task} level the size of training set for use in learning
is specified, along with whatever prior information is available
(which can be used, for example, to select encodings for the
attributes).  The task level specifies enough information that a
learning method will have a well-defined expected performance with
respect to any particular loss function.  A \emph{task instance} is a
particular training set and test set for a task, to which we can
actually apply a learning method.  The performance of a method on
several task-instances is used to estimate its expected performance on
the task.

Tasks do not have directories of their own in the data part of the
\delve{} hierarchy.  However, the results of applying a particular
method to a particular task are contained in a directory in the
methods part of the \delve{} hierarchy.

The methods part begins at the \texttt{methods} directory of a top-level
\delve{} directory.  Within this directory are sub-directories for the
various learning methods that have been assessed, each of which will
contain a description of the method, and perhaps the programs
implementing it.  The directory for a method will also contain
sub-directories for every dataset that has been used in assessing the
method, within which will be sub-directories for each prototask to
which the method has been applied.  Inside the directories
corresponding to prototasks will be task directories, containing the
results of applying the method to the various task instances.

A \delve{} hierarchy is illustrated in Figure~\ref{fig-my_method}.
The top-level \texttt{delve} directory could reside anywhere on your file
system, but it's name must start with \texttt{delve}, and it must contain
two sub-directories called \texttt{data} and \texttt{methods}.

\begin{figure}[t]
  \vspace{2mm}\par
  \centerline{\epsfxsize 1.0\hsize \epsfbox{dd.eps}}
  \caption[Schematic diagram of the structure of a \delve{} directory.]
  {Schematic diagram of the structure of a \delve{} directory.
  Names of directories are in bold font, names of files are in normal
  font. Files in brackets may not be present, as they can be
  generated by \delve{} if needed. A dotted line indicates parts of the
  directory that are left out.}\label{fig-my_method}
\end{figure}

In Figure~\ref{fig-my_method}, the data part of the hierarchy contains
two datasets, \texttt{demo} and \texttt{kin-8nh}, each with its own
sub-directory within the \texttt{data} directory.  Inside each dataset
directory there are two files:\ \ \texttt{Dataset.data} which contains
the cases, and \texttt{Dataset.spec} which contains information about the
data.  There is also a sub-directory called \texttt{Source}, which
contains all the data and information used to build the dataset, as it
was originally obtained.

The \texttt{demo} dataset in Figure~\ref{fig-my_method} has two
prototasks, \texttt{age} and \texttt{income}, which differ in the
attribute that is to be predicted.  These prototask directories
contain the files needed to specify both the prototask itself and the
tasks that are defined for it.

Back at the top level, the \texttt{methods} directory in
Figure~\ref{fig-my_method} contains sub-directories for two methods:\
\ \texttt{lin-1} (a linear regression model), and \texttt{knn-cv-1} (a
$k$-nearest-neighbor method).  The descriptions and program source for
these methods are contained in their \texttt{Source} directories.  The
results of applying the methods to various tasks are contained in
directories whose names combine the name of the prior for the task ---
\texttt{std}, for ``standard'', in these examples --- and the size of the
training set.  These task directories may contain files with
names such as \texttt{guess.*}, \texttt{prob.*}, and \texttt{loss.*} files, 
which record the final results of learning and prediction (here, ``\texttt{*}''
indicates several possible endings, which specify a particular
task-instance and loss function).


\subsection{Using \delve:~~A tutorial example}\label{intro-tutorial}

This section is a walk-through tutorial, which introduces \delve{} by
showing how you can test a simple learning method using the \delve{}
utilities.

This tutorial assumes that you (or someone else) have installed
\delve{} on your computer.  For details on how to do this, see
Appendix~\ref{app-install}.  It also assumes that the \delve{}
utilities are somewhere in your shell's search path.  This will likely
be true if \delve{} has been installed in its usual place in
\texttt{/usr/local/bin}, but if it has been installed elsewhere, you
may have to set your \texttt{PATH} environment variable appropriately.

The roles of the \delve{} commands used in this tutorial are described
in more detail in later chapters.  For detailed descriptions of 
command syntax and options, refer to Appendix~\ref{app-commands}.


\subsubsection*{Telling \delve{} where to look for information --- setting your
                \delvepath{}}

First you need to know how \delve{} looks for information in one or
more active \texttt{delve} directories. Formally, a \texttt{delve} directory
must have a name starting with the five letters ``\texttt{delve}'' and
{\em must\/} have two sub-directories called \texttt{data} and {\tt
methods}. You tell \delve{} where to look for these directories by
setting the \delvepath{} shell environment variable. This path works
analogously to the shell search path, except that \delve{} looks in
all the directories in \delvepath{} rather than stopping as soon as
the first match is found.  It therefore makes little difference what
order the directories in \delvepath{} come in.

You can create your own \texttt{delve} directory, and tell \delve{} to
use both it and a \texttt{delve} directory that holds data, methods, and
results from the \delve{} archive.  Assuming that \delve{} has been
installed on your machine in directory \texttt{/usr/local/lib/delve}, you
might do this as follows, if you use a shell program like \texttt{csh}:

\begin{Session}
unix> cd $HOME
unix> mkdir delve delve/data delve/methods
unix> setenv DELVE_PATH /usr/local/lib/delve:$HOME/delve
\end{Session}

If you use a shell like \texttt{sh}, you would instead say:

\begin{Session}
unix> cd $HOME
unix> mkdir delve delve/data delve/methods
unix> DELVE_PATH=/usr/local/lib/delve:$HOME/delve
unix> export DELVE_PATH
\end{Session}

In either case, you would probably want to put the commands that set
\delvepath{} in your shell start-up file (either \texttt{.cshrc} or 
\texttt{.profile}), so that \delvepath{} will be set again when you next
log in.

Setting up your \delvepath{} in this way lets you keep the material
distributed with \delve{} separate from the results of your own
experiments.  You could also have several \texttt{delve} directories of
your own, or include other users' \texttt{delve} directories in your
\delvepath{} in order to access their results.  Also, whenever your
current directory is within a valid \texttt{delve} directory, that
directory will be temporarily added to the list of active \texttt{delve}
directories, in addition to those in \delvepath{}.  This lets you
easily look in a \texttt{delve} directory that you don't usually access.

\subsubsection*{Listing information 
    --- \dls{} and \mls{}, \dinfo{} and \minfo{}, \dmore{} and \mmore{}}

Once you have set your \delvepath{} to a list of \texttt{delve}
directories, you can use various \delve{} commands to look at
information in the \delve{} hierarchy that is contained in these
directories.  These commands come in two flavours --- ``\texttt{d}''
commands that look in the \texttt{data} part of the hierarchy, and ``{\tt
m}'' commands that look in the \texttt{methods} part.  You can, for
instance, find out what files are in the directory for a particular
dataset using \dls{}, or get a formatted display of various information
about a dataset using \dinfo{}.

You can specify what you want to look at with these commands in two
ways.  One way is to give a \emph{dpath} or \emph{mpath} that
specifies the location of a file or directory in the \texttt{data} or
\texttt{methods} part of the hierarchy.  Such dpaths and mpaths start
with ``\texttt{/}'', and are translated by \delve{} into one or more
Unix path names within the active \texttt{delve} directories.  The
other way is to specify a relative Unix path name (which doesn't start
with ``\texttt{/}'' or ``\texttt{\maketilde{}}'') of a file or directory
in the \delve{} hierarchy.

The \dls\ and \mls\ commands are analogous to the Unix \texttt{ls}
command.  They let you look at what files and sub-directories exist in
the \texttt{data} and \texttt{methods} parts of the \delve{} hierarchy.  For
example, we can use the \dls\ command with a dpath of ``\texttt{/}'' to
list the datasets found in all the active \texttt{delve} directories:

\begin{Session}
unix> dls /
demo kin-8nh kin-8nm
\end{Session}

There might be many more than these three datasets, of course,
depending on what you have installed, and on how your \delvepath{} is
set.  Note that the three datasets shown would not necessarily be
located in the same Unix directory.  

We can list the files and sub-directories for the \texttt{demo} dataset
as follows:

\begin{Session}
unix> dls /demo
Dataset.data Source       age          income       siblings                  
Dataset.spec Summary      colour       sex
\end{Session}

Once again, these files and sub-directories might not all be in the
same Unix directory, though in this case, these files and
sub-directories are in fact all present in the directory for
\texttt{demo} from the \delve{} archive.  To see exactly where things
exist, you can use the \texttt{-l} option.  (By the way, you can find
out about options for any \delve{} command by executing the command
with the \texttt{-h} option.)

Here we use \texttt{-l} with the corresponding \mls{} command, to see
what methods are available, in what places:

\begin{Session}
unix> mls -l /
/usr/local/delve/methods/:
knn-cv-1 lin-1
\end{Session}

We see that there are just two methods, \texttt{knn-cv-1} and
\texttt{lin-1}, and that files for both methods are found only in the
directory holding information from the \delve{} archive.  If we had
tested one of these methods on new datasets of our own, however,
results for that method could exist in our private \texttt{delve}
directory as well, in which case directories for the method would
exist in both places.

Two \delve{} commands similar to the Unix \texttt{more} command also
exist, called \dmore\ and \mmore.  Here we use \dmore\ to look at the
summary description for the \texttt{demo} dataset:

\begin{Session}
unix> dmore /demo/Summary
The "demo" dataset was invented to provide an example for the DELVE
manual, and to test the DELVE software and software that implements
learning methods.  To those ends, it has a variety of numerical and
categorical attributes.  Cases for the "demo" dataset were artificially 
generated from a distribution based on simple demographic assumptions
and various stereotypical notions concerning the relationships between
people's sex, age, number of siblings, income, and favourite colour.
Prototasks are defined for predicting each of these attributes given
the others.
\end{Session}

We could also use \dmore{} to look at the specification file for a
dataset, but we would not usually do so, since \delve{} provides a
command \dinfo{} for conveniently displaying this and other
information about datasets.  We can ask about information for the
\texttt{demo} dataset as follows:

\begin{Session}
unix> dinfo /demo
Dataset: /demo
Origin: artificial
Usage: development
Order: uninformative
Number of attributes: 5
Prototasks: 
        age
        colour
        income
        sex
        siblings
\end{Session}

We would see more details if we used the \texttt{-a} option (ie, 
the command `\texttt{dinfo -a /demo}').  Similarly, we can ask for
information about the \texttt{age} prototask from the \texttt{demo}
dataset with the command ``\texttt{dinfo /demo/age}'', and so on.
There is a corresponding \minfo{} command for getting information on
learning methods, and on their application to learning tasks.


\subsubsection*{Applying your learning method to a task --- \mgendir{} and 
                \mgendata{}}

Now that you have seen how to obtain information about datasets and
methods in \delve, we will see how we can go about testing a simple
learning method, which we will call {\tt mymethod}.  First, we need to
create a directory for the method, with a structure of sub-directories
similar to that for the \texttt{lin-1} method depicted in
Figure~\ref{fig-my_method}.  These sub-directories will hold the
results of applying the method to the \texttt{demo/age} prototask.  We
could create all these sub-directories using the Unix \texttt{mkdir}
command, but it is more convenient to use the \delve{} \mgendir{}
command:

\begin{Session}
unix> cd delve/methods
unix> mkdir mymethod
unix> cd mymethod
unix> mgendir demo/age
demo
demo/age
demo/age/std.32
demo/age/std.64
demo/age/std.128
demo/age/std.256
demo/age/std.512
\end{Session}

Now that the \mgendir{} command has created the appropriate
directories, we can proceed to put files containing training and test
data into the sub-directory for one of the tasks (with the standard
prior, and 256 training cases) using the \mgendata{} command:

\begin{Session}
unix> cd demo/age/std.256
unix> mgendata
  segmenting cases...
  splitting test inputs and targets...
  encoding instance 0 training data...
  encoding instance 0 test inputs...
  encoding instance 0 test targets...
  encoding instance 1 training data...
  encoding instance 1 test inputs...
  encoding instance 1 test targets...
  encoding instance 2 training data...
  encoding instance 2 test inputs...
  encoding instance 2 test targets...
  encoding instance 3 training data...
  encoding instance 3 test inputs...
  encoding instance 3 test targets...
\end{Session}

This command creates files in the current directory pertaining to four
tasks instances.  The \file{train}{n} files contain the inputs and
targets for the training cases in instance \textit{n}, the
\file{test}{n} files the inputs for the test cases, the
\file{targets}{n} files the true targets for the test cases, and the
\file{normalize}{n} files the normalization constants used in encoding
the data.  Files called \texttt{Coding-used} and
\texttt{Test-set-stats} are also created; they hold information used
by later commands.  

You can get information about the way this method is being applied to
this task using the \minfo{} command.  When called with no arguments,
this command will give information about the method and task
associated with the current directory, as illustrated below:

\begin{Session}
unix> minfo
Task: /demo/age/std.256
Training set size: 256
Inputs: 
 col attr name          type   relevance  coding  options
   1   1  SEX           binary   nlmh     -1/+1        -
   2   3  SIBLINGS      integer  nlmh     nm-abs       -
   3   4  INCOME        real     nlmh     nm-abs       -
   4   5  COLOUR:pink   nominal  nlmh     1-of-n       -
   5   5  COLOUR:blue                  ...
   6   5  COLOUR:red                   ...
   7   5  COLOUR:green                 ...
   8   5  COLOUR:purple                ...
Targets: 
 col attr name          type   relevance  coding  options
   1   2  AGE           real     nlmh     nm-abs       -
\end{Session}

This shows things such as the way the various attributes have been
encoded in the data files to be used by the method (in this case, the
default encodings were used).  Similar information would be displayed
by the \dinfo{} command, but \minfo{} will show any information specific
to how this learning method is being applied to this task, whereas \dinfo{}
shows only information about a dataset itself, and its associated tasks.

Each of the \file{train}{n} files that were created above contains one
line for each training case.  With the default encoding used above,
the four input attributes are encoded as eight numbers, which appear
at the beginning of the line. (The \texttt{COLOUR} attribute is
encoded in \texttt{1-of-n} form, which uses five numbers to represent
which of its five possible values the attribute has.)  The target is
encoded as a ninth number, at the end of the line.

You can now train your model using the data in each of the four
\file{train}{n} files.  This is to be done separately for each file,
as the four training files are for four instances of the task, which
are to be handled completely independently of each other. You then use
the results of this training to make guesses for the targets in the
test cases that go with each task instance, given the inputs for these
cases in the \file{test}{n} files.  Your method should write its
guesses in the files \texttt{cguess.S.0} though \texttt{cguess.S.3}, one
guess per line.  Here, the prefix `\texttt{c}' indicates that the
guess are for the coded form of the attribute, not the original form
in which it appears in the dataset file.  The suffix \texttt{S}
indicates that the guesses are designed for use with the squared error
loss function.

For this tutorial, we don't want to get into the complexities of
writing a realistic learning method, so we'll use as an example a
method that simply predicts the constant zero for every test case.
Notice that, as seen in the output from \minfo{} above, the (default)
encoding of the targets used here is \texttt{nm-abs}, which means that
they are shifted and re-scaled so that the median of the target values
in the training cases is zero, and the average absolute deviation from
the median in the training cases is one.  Because of this, always
predicting zero, while not very sophisticated, is at least not wholely
unreasonable.  This method can be implemented by the following
\texttt{awk} commands:

\begin{Session}
unix> awk ' \{ print "0.0" \} ' test.0 > cguess.S.0
unix> awk ' \{ print "0.0" \} ' test.1 > cguess.S.1
unix> awk ' \{ print "0.0" \} ' test.2 > cguess.S.2
unix> awk ' \{ print "0.0" \} ' test.3 > cguess.S.3
\end{Session}

Notice that the training data is ignored here (though it is implicitly
used through the use of a normalized encoding), and the test data is
looked at only in order to determine how many test cases there are.
However, this is certainly not typical behaviour for a learning
method!


\subsubsection*{How well did it do? --- \mloss{} and \mstats{}}

Once our method has produced \file{cguess}{n} files containing its
guesses for targets, we can use the \mloss{} command to evaluate the
``loss'' suffered when using each of these guesses.  The loss is based
on the difference between the guess and the actual target value.  To
find the losses as judged by the squared difference between guess and
target value, we would use \mloss{} with the `\texttt{-l S}' option:

\begin{Session}
unix> mloss -l S
  decoding cguess.S.0...
  decoding targets.0...
  creating loss.S.0...
  decoding cguess.S.1...
  decoding targets.1...
  creating loss.S.1...
  decoding cguess.S.2...
  decoding targets.2...
  creating loss.S.2...
  decoding cguess.S.3...
  decoding targets.3...
  creating loss.S.3...
\end{Session}

The \mloss{} command transforms the guesses in the \file{cguess.S}{n}
files back to the original domain, storing these transformed guesses
in the files \file{guess.S}{n}.  It then computes the loss for each
test case and writes these losses to the \file{loss.S}{n} files.

We can now use the \mstats{} command to get a summary of the predictive
performance of our method.  Here, we give the `\texttt{-l S}' option
to \mstats{} to say we are only interested in the squared error loss
function:

\begin{Session}
unix> mstats -l S
/mymethod/demo/age/std.256
Loss: S (Squared error)
                                                    Raw value   Standardized

                         Estimated expected loss:     520.43       1.06461
                     Standard error for estimate:       41.7     0.0853028

     SD from training sets & stochastic training:    49.1004      0.100441
SD from test cases & stoch. pred. & interactions:    1078.63       2.20648

    Based on 4 disjoint training sets, each containing 256 cases and
             4 disjoint test sets, each containing 256 cases.
\end{Session}

The first line of this summary gives an estimate for the expected loss
when using this method on this task; the next line gives a standard
error for this estimate.  The lines below these give the standard
deviations for the variation in performance due to various causes.
For a more detailed discussion of these statistics refer to
Section~\ref{sec-analysis}.  The second column gives the same
quantities rescaled to a standardized domain, which makes
interpretation easier.  In the case of squared error, the losses are
standardized by dividing by the sample variance of the targets in all
the test cases.

The \mstats{} command can also be used to compare the performance of
different learning methods.  In the \texttt{methods} part of the
\delve{} hierarchy are descriptions and results for a selection of
learning methods on some of the \delve{} tasks.  If you have obtained
the results of the linear regression method called \texttt{lin-1} from
the \delve{} archive, you will be able to compare your method to
the \texttt{lin-1} method as follows (again, with respect to squared
error loss):

\begin{Session}
unix> mstats -l S -c lin-1
/mymethod/demo/age/std.256
Loss: S (Squared error)
                                                    Raw value   Standardized

            Estimated expected loss for mymethod:     520.43       1.06461
              Estimated expected loss for /lin-1:     397.82      0.813792
                   Estimated expected difference:     122.61      0.250815
          Standard error for difference estimate:    26.9735     0.0551778

     SD from training sets & stochastic training:    44.5182     0.0910678
SD from test cases & stoch. pred. & interactions:     487.52      0.997285

    Significance of difference (t-test), p = 0.0199425

    Based on 4 disjoint training sets, each containing 256 cases and
             4 disjoint test sets, each containing 256 cases.
\end{Session}

This shows that the linear method has a smaller expected loss that our
more trivial method. Notice that the expected difference between the
methods is approximately $4$ times greater than the standard error on
this estimate. The p-value from the $t$-test indicates that the
difference should be considered significant at the $2\%$ level.  The
methods used to compute such p-values are described in
Section~\ref{sec-analysis}.

It could happen that when you tried to compare our method with
\texttt{lin-1}, as shown above, \mstats{} could fail to find
\texttt{loss} files for the linear method in any of the active
\texttt{delve} directories.  If this were to happen, you could
generate the \texttt{loss} files needed by \mstats{} yourself
(assuming that the \texttt{guess} for \texttt{lin-1} were available).
However, you would probably need to generate these files in your own
\delve{} directory, since you likely don't have permission to write in
the directory that holds information from the \delve{} archive.  To
achieve this, you could do the following:

\begin{Session}
unix> cd $HOME/delve/methods
unix> mkdir lin-1
unix> cd lin-1
unix> mgendir demo/age
unix> cd demo/age/std.256
unix> mloss
\end{Session}

The \mstats{} command will now be able to use these \texttt{mloss}
files, as long as the \texttt{delve} directory they are stored within
is mentioned in your \delvepath, or you are currently inside this
\texttt{delve} directory.  

In similar fashion, you put things in your own \texttt{delve}
directory that extend what is in the the \delve{} archive by adding
new datasets, new prototasks for old datasets, new methods and results
for new methods, and new results for old methods.  However, to avoid
confusion, \delve{} will not allow you to use names for new things
that are the same as the names for things that already exist in the
\delve{} archive directory.


\subsection{What to read next}\label{intro-what-next}

Section~\ref{sec-aims} contains a more detailed specification of the
scope and aims of the \delve{} project; this section may be of general
interest.  Sections~\ref{sec-data} and~\ref{sec-task} contain detailed
descriptions of how datasets, prototasks, and tasks are specified in
\delve{}. These sections may be of some interest to all users, but are
primarily intended for people who wish to include new datasets in
\delve{}, or who wish to create new prototasks and tasks based on
existing datasets.  Section~\ref{sec-loss} describes the standard loss
functions supported by \delve, and discusses how other loss functions
can be incorporated.  Section~\ref{sec-scheme} discusses the schemes
for learning experiments used in \delve{}, and compares these to more
traditional schemes such as cross-validation.  

Users who want to get straight into using \delve{} to test their
learning methods may wish to just skim these initial sections, and
start serious reading with Sections~\ref{sec-assess}
and~\ref{sec-analysis}, which describe the methodology for \delve{}
assessments, and the \delve{} commands required to perform them.

Appendix~\ref{app-install} tells you how to get software, data,
and results from the \delve{} archive, while Appendix~\ref{app-submit}
tells you how to contribute things to the \delve{} archive.  Detailed
descriptions of \delve{} commands are found in
Appendix~\ref{app-commands}, and a glossary of \delve{} terminology is
found in Appendix~\ref{app-glossary}.

%
% $Id: sec2.tex,v 1.12 1996/05/10 22:45:37 radford Exp $
%
\newpage

\section{THE SCOPE OF THE DELVE PROJECT}\label{sec-aims}
\thispagestyle{plain}
\setcounter{figure}{0}
\chead[\fancyplain{}{\thesection.\ THE SCOPE OF THE DELVE PROJECT}]
      {\fancyplain{}{\thesection.\ THE SCOPE OF THE DELVE PROJECT}}

The aim of the DELVE project is to promote the development and use of
empirical learning methods by providing a well-designed environment in
which the performance of such learning methods can be assessed on data
that is relevant to the real world.  This is a broad objective, which
we can hope only to partially fulfill.  This section outlines the
scope of the DELVE project at present --- the sorts of learning
methods that DELVE can handle, the sorts of assessments that DELVE
supports for these methods, and the kinds of dataset on which these
assessments are performed.

As researchers ourselves, we of course have ideas about which learning
methods are most promising, but we have tried to keep such prejudices
from affecting the design of DELVE.  We have also tried to minimize
the extent to which DELVE constrains the sorts of questions that
researchers can investigate.  Inevitably, however, we have had to use
our own judgement in making tradeoffs between different design goals,
some of which are mentioned below.


\subsection{Learning methods that DELVE can handle}\label{scope-range}

At present, DELVE supports only methods for {\em supervised
learning\/} --- that is, methods that aim to predict one or more {\em
target attributes\/} using the information provided by some set of
{\em input attributes\/}.  The relationship between the inputs and the
targets is learned from a number of {\em training cases\/}, in which
both the inputs and targets are known.  These training cases are
modeled as if they were generated more-or-less independently from some
source.  The goal of learning is to predict the target in a {\em test
case\/}, generated from the same source as the training cases, but for
which only the inputs are known.  For some datasets, the cases are not
truly independent, but the primary goal is always to learn the
relationship of targets to inputs, not to learn the nature of any
dependencies between cases.

We distinguish between \emph{regression} tasks, in which the targets
(usually one, but sometimes more) are real-valued, and
\emph{classification} tasks, in which there is a single target, the
\emph{class} of the item in question, which takes on values from a
small set.  We also provide some limited support for other supervised
learning tasks, such as those in which the target is an integer, or an
angular value.

The DELVE facilities presently treat the attributes in a case as an
unstructured collection of values.  In some applications, such as
image processing, the attributes (eg, pixel values) are known to have
certain relationships to each other (eg, spatial adjacency), which can
be of great help in learning.  Although data from such application
areas could be included in DELVE, assessments using this data may be
of limited interest, since DELVE provides no scheme for informing
learning methods about such structure in the data.

In future, we hope to also support \emph{unsupervised learning}
methods and related statistical methods such as density estimation, in
which attributes are not characterized as inputs or targets.  As well,
we may someday add facilities for assessing \emph{time series}
methods, in which the aim is to characterize the sequential
dependencies between cases.


\subsection{Aspects of performance that can be assessed using 
            DELVE}\label{scope-aspects}

DELVE is aimed primarily at assessing the \emph{predictive
performance} of learning methods --- that is, their ability to make
predictions in previously unseen cases by generalizing from the
information contained in the data used for training.
\emph{Computational performance} --- the amount of time and
space needed for training and subsequent use of the methods --- is
also of concern.  There will often be a tradeoff between predictive
performance and computational performance.  However, DELVE does not
include any datasets where computational considerations appear
paramount, as might be the case, for example, when the amount of data
is extremely large.

Other characteristics of learning methods are also of interest,
such as ease of use by both expert and inexpert users, and the degree
to which the results of learning can be interpreted, but DELVE does
not support any formal evaluation of such characteristics.


\subsection{How DELVE encourages meaningful assessments}\label{scope-req}

The DELVE environment is designed to encourage and assist users to
produce meaningful assessments that are \emph{faithful},
\emph{comparable}, and \emph{reproducible}.  

To be \emph{faithful}, an assessment of a learning method must be
indicative of how well it would perform on an actual task that is of
some interest.  One must, for example, avoid any inadvertent
``cheating'', such as would occur if parameters of the learning method
were set on the basis of performance on the test cases.  Arbitrary
restrictions on how learning methods may be used must also be avoided,
if better performance might be obtained in a real application by doing
things differently.

For assessments of different learning methods to be \emph{comparable},
they must all have been applied in the same context --- for instance,
with training sets of the same size, and with equivalent attention
being paid to prior information.  It is perhaps in this respect that a
standard environment such as DELVE is most useful.

One requirement for an assessment to be \emph{reproducible} is that
the method used be adequately documented.  To encourage this, we
have provided guidelines for proper documentation, and examples of
their use.  Reproducibility is most easily achieved if the method
is fully automatic.  This is not always possible, however, so we
suggest ways of improving the reproducibility of methods that
involve human decisions.

Furthermore, DELVE is designed to provide assessments that are as {\em
accurate\/} as is practical, and for which the degree of accuracy is
known.  DELVE also supports comparisons of learning methods that
provide indications of the statistical significance of any observed
differences.  The power of these comparisons is increased by using the
same training and test sets for different methods, which is another
advantage of a standard environment.


\subsection{Kinds of datasets included in \delve{}}\label{scope-data}

Obtaining data is one of the most crucial, and most difficult, parts
of building an assessment environment.  We have drawn datasets for
DELVE from four sources, each of which has its advantages.

\emph{Natural} datasets come from real-world sources, and were at one
time used to address questions of real interest that are similar to
those addressed by the supervised learning methods we would like to
assess.  \emph{Cultivated} datasets also come from the real world, but
do not represent real supervised learning problems.  Such cultivated
data was instead gathered or selected specifically for the purpose of
assessing learning methods.  We also include real-world datasets that
have been altered (eg, by adding noise) in this category.

\emph{Simulated} datasets are generated by a computer simulation of a
real-world phenomenon.  To qualify for this category, the simulation
should be reasonably realistic, and of a complexity that makes it
difficult to see what form the relationships in the data will take.
\emph{Artificial} datasets are randomly generated from a distribution
defined by a relatively simple mathematical formula.

Natural datasets have the advantage of being arguably representative
of the problems we are actually interested in.  For example, a
statistical consultant might reasonably conclude that it would be
worthwhile to learn more about a learning method that has been found
to perform better than others on such real-world problems.  Relevance
to the real world is more doubtful for cultivated, simulated, and
artificial datasets.  As the datasets become less natural, it also
becomes more likely that a researcher may bias the assessment of a
learning method by unconsciously selecting problems on which that
method can be anticipated to do well.

Why, then, do we include any other than natural datasets?  One reason
is that the number of readily-available natural datasets is limited,
and those that are available are usually not as large as we would
like.  In the real world, the cost of collecting data is often high,
and we must try to obtain the most information possible from a small
dataset.  To properly assess the performance of a learning method in
such a context, however, we need much more data, in order to reduce
the uncertainty in our estimate of expected performance.  Simulated
and artificial datasets can easily be made as large as required
(limited only by storage space); this can greatly improve the accuracy
of performance estimates.

Another reason for using non-natural datasets is that they can be
designed to address certain questions that would otherwise be
difficult to answer, such as what the effect is of adding extra noise
to the input attributes, or of adding extra irrelevant inputs.  In
particular, we can design families of tasks that are related in
interesting ways --- eg, that have more or less noise, or a larger or
fewer number of input attributes --- and see how these dimensions of
variation affect the performance of various learning methods.

When we began collecting datasets for use in assessing supervised
learning methods, we had hoped to confine ourselves to datasets
where the cases were truly independent, as independence of cases is an
assumption behind many existing supervised learning methods.  We
found, however, that in many otherwise-interesting datasets, there is
at least a possibility of dependencies between cases.  We therefore
decided to include such datasets, both in order to increase the
variety of datasets available, and because it seems to us that the
possibility of such dependencies is a common feature of real-world
problems, which designers of supervised learning methods may be
well-advised to accommodate.  We have, however, avoided datasets in 
which the dependencies themselves are the primary focus of interest.

%
% $Id: sec3.tex,v 1.31.2.2 1996/06/13 21:57:15 carl Exp $
%
\newpage

\section{DATASET FILES AND SPECIFICATIONS}\label{sec-data}
\thispagestyle{plain}
\setcounter{figure}{0}
\chead[\fancyplain{}{\thesection.\ DATASET FILES AND SPECIFICATIONS}]
      {\fancyplain{}{\thesection.\ DATASET FILES AND SPECIFICATIONS}}

A dataset is a collection of \emph{cases}.  For each case, the values
of certain \emph{attributes} are recorded.  \delve{} stores these
attribute values in a file with a standard format that is general
enough that a wide variety of datasets can be represented without loss
of information.  For each dataset, \delve{} also keeps a specification
file, which records basic information such as the number of attributes
and their theoretical ranges.  Finally, the original files or programs
from which the dataset was derived are retained in the \delve{}
archive, along with any original documentation.

Files relating to a dataset are kept in a directory with the same name
as the dataset, located in the {\tt data} sub-directory of a top-level
{\tt delve} directory.  Some of the files that may appear in such a 
dataset directory are listed in Figure~\ref{fig-dataset-dir}.

\begin{figure}[b]

\rule{\textwidth}{0.5pt}

\hspace*{-4pt}\begin{tabular}{ll} \\[-6pt]
{\tt Summary} & A brief description of the dataset \\
{\tt Dataset.data} 
  & The actual data, in the format described in Section~\ref{data-format} \\
{\tt Dataset.spec}
  & Specifications for the dataset, usually accessed using the \dinfo\ command 
    \\[5pt]
{\tt Source} 
  & A sub-directory with files relating to the source of the dataset, such as:\\
\hspace{13pt}{\tt Notes}    
  & \hspace{13pt}Documentation on the dataset \\
\hspace{13pt}{\tt original}
  & \hspace{13pt}The original data file (but sometimes there will be more 
    than one) \\
\hspace{13pt}{\tt gen.c}    
  & \hspace{13pt}C program for generating dataset (or {\tt gen.f} for a Fortran
    program, etc.) \\[5pt]
$\!\!\!\left.\begin{array}{l} 
\mbox{\em Prototask-1} \\ 
\mbox{\em Prototask-2} \\ 
\mbox{\em Prototask-3} 
\end{array}\ \right\}$
  & Sub-directories for prototasks based on the dataset 
    (see Section~\ref{sec-task}) 
\end{tabular}

\caption{Some files and sub-directories that may appear within a \delve{}
         dataset directory.}

\label{fig-dataset-dir}
\end{figure}


\subsection{Specifications for datasets:~~The \dinfo{} command}\label{data-spec}

The specifications for a dataset include information about the dataset
as a whole, such as its origin and usage within \delve{}, plus
information about each attribute in the dataset, such as its range of
legal values.  This information is stored in the dataset's
specification file, \texttt{Dataset.spec}.  However, the only time you
will need to directly access this specification file is when you
create a new dataset, using the procedure described in
Section~\ref{data-prepare}.

Usually, it is more convenient to view the specifications for a
dataset using the \dinfo{} command, as was illustrated in the tutorial
in Section~\ref{intro-tutorial}.  For instance, to see the
specifications (as well as some other information) for the
\texttt{demo} dataset, you would use the command\vspace{-5pt}
\begin{Session}
   dinfo /demo 
\end{Session}\vspace{-5pt} 
Further details on individual attributes of the dataset can be
obtained by using the \texttt{-a} option with \dinfo{}, as is
illustrated in Figure~\ref{fig:dataset-dinfo-a}.

Note that dataset specifications contain only very basic information,
which is not likely to be wrong unless the data has been totally
misinterpreted.  More debatable prior information may be specified as
part of a task description (Section~\ref{task-prior}).

The following characteristics of a dataset as a whole are recorded
as part of its specification, and displayed by \dinfo{}:\vspace{-3pt}%
\begin{list}{}{%
\setlength{\leftmargin}{1.1in}%
\setlength{\labelwidth}{0.7in}%
\setlength{\labelsep}{0.1in}%
}
\item[{\tt Origin:}\hfill]
   {\tt natural} \OR {\tt cultivated} \OR {\tt simulated} \OR
   {\tt artificial}

A \emph{natural} dataset was originally gathered for some real-world
application; a \emph{cultivated} dataset comes from a real-world
source, but was never used to solve a real problem; a \emph{simulated}
dataset was generated by a simulator, but is believed to resemble real
data --- as opposed to an \emph{artificial} which is generated
according to some mathematical formula and does not pretend to
resemble any real dataset. These distinctions are discussed further in
Section~\ref{scope-data}.

\item[{\tt Usage:}\hfill]
  {\tt development} \OR {\tt assessment} \OR {\tt historical}
  \OR {\tt ?}

A \emph{development} dataset is recommended for use in developing new
learning methods, but to avoid bias, should not be used for formal
assessments.  An \emph{assessment} dataset is intended for use in
formal assessments; use for development should be minimized.  A
\emph{historical} dataset is included in \delve{} because it has been
used for assessing learning methods in the past, but is not
recommended for general use.  A `?' indicates that a recommended usage
has not yet been decided on.

\item[{\tt Order:}\hfill] 
  {\tt informative} \OR {\tt uninformative} \OR {\tt ?}

A dataset has an \emph{informative} ordering if the order of cases
may convey information that is not already present in the attribute
values.  The order is recorded as \emph{uninformative} if it is
random, or has some basis that is not related to any matter of
interest.  The order is recorded as `?' if the order appears to be
non-arbitrary, but the basis of the ordering cannot be determined from
the available documentation.

\item[{\tt Commonality indexes are present}] ~

If this line is displayed by \dinfo, {\em commonality indexes\/} are
associated with some or all cases in the dataset.  Cases with the same
commonality index share something in common, as is described further
in Section~\ref{data-dependencies}.  If this line is not displayed,
the cases in the dataset do not have commonality indexes.\vspace{-3pt}
\end{list}

If the ordering of a dataset is informative, or if commonality indexes
are present, the issue of possible dependencies between cases must be
addressed, as is discussed in Section~\ref{data-dependencies}.

\begin{figure}[t]
\begin{Session}
Dataset: /demo
Origin: artificial
Usage: development
Order: uninformative
Number of attributes: 5
Attributes: 
     #  name     c/u range        description
     1  SEX       u  male female  Sex of the person
     2  AGE       u  [0,Inf)      Age of the person in years
     3  SIBLINGS  u  0..Inf       Number of siblings the person has
     4  INCOME    u  [0,Inf)      The person's annual income (dollars)
     5  COLOUR    u  pink blue red green purple 
                                  The person's favourite colour
Prototasks: 
        age
        colour
        income
        sex
        siblings
\end{Session}\vspace{-4pt}
\caption{Output of the command: \texttt{dinfo -a /demo}.}
\label{fig:dataset-dinfo-a}
\end{figure}

Each dataset has a specified number of attributes associated with each
case.  Datasets in which the number of attributes varies from case to
case are not handled by \delve, though it is possible for the values
of some attributes to be missing in some cases (see
Section~\ref{data-format}).  The attributes for a dataset are numbered
from 1 on up.  Attributes can also have short {\em names}, which can
be used in place of numbers to identify them.  For the \texttt{demo}
dataset illustrated in Figure~\ref{fig:dataset-dinfo-a}, the attributes
have names of \texttt{SEX}, \texttt{AGE}, etc.

The dataset specification also records whether each attribute was
\emph{controlled} or \emph{uncontrolled} (abbreviated to `{\tt c}'
or `{\tt u}' in the output of \dinfo).  The values of a controlled
attribute were fixed for each case by the investigator who gathered
the data; the values of an uncontrolled attribute were not fixed,
though the investigator will often have had some influence on the
mechanism by which they were generated.  For example, in a dataset
concerning the growth of plants under various conditions, the amount
of fertilizer applied to a plant would usually be a controlled
attribute, whereas the amount of rainfall would be an uncontrolled
attribute.  This field will be recorded as `?' if it is not clear from
the available documentation whether or not the attribute was
controlled.

Each attribute in the dataset also has a specified \emph{range},
consisting of a list of items, each of which defines a set of allowed
values for the attribute.  Such an item can specify a single permitted
value (which could be a \emph{missing value}, as discussed in
Section~\ref{data-format}), or a set of permitted numerical values
having the form of an open, closed, or half-open interval of real
numbers, or a range of integers.  The bounds of a real interval can be
ordinary numbers, or one of `{\tt Inf}', `{\tt -Inf}', or `{\tt
+Inf}', with `{\tt Inf}' representing infinity; these bounds are
enclosed by round or square brackets, indicating whether the bound
itself is included.  For example, {\tt [0,1)} represents the interval
from 0 to 1, including 0, but not including 1, and {\tt (0,Inf)}
represents the set of positive real numbers.  An integer range
extending from \emph{low} to \emph{high}, inclusive, is written as
\emph{low{\tt ..}high} (with no enclosing brackets); \emph{low} and
\emph{high} can be infinite, as for real intervals.  For example, {\tt
1..Inf} represents the positive integers.

Several items can be combined, as in the following range:\vspace{-5pt}
\begin{Session}
(-Inf,0)  (0,+Inf)  ?
\end{Session}\vspace{-5pt}
This specifies that the attribute can take on any numerical value
other than zero, as well as the missing value indicator,
`\texttt{?}'.

Note that the range specified for an attribute is the full set of
conceivable values, regardless of whether all of these values actually
occur.  For example, the range {\tt [0,100]} would be appropriate for
an attribute that represents the percent by weight of water in a
sample of some substance, since it is inconceivable that the value
could ever fall outside this range, but any more narrow range would
not be appropriate, even if the actual values in the dataset 
never exceed 10\%.  Similarly, for an attribute representing a person's
birth sign, the appropriate range would be all twelve signs of the
zodiac, even if no Scorpios happen to be included in the dataset.

Finally, an attribute may be accompanied by a short \emph{description},
which is ignored by the \delve{} software, but may help users keep track
of which attribute is which.


\subsection{Datasets with dependencies between cases}
\label{data-dependencies}

Dependencies between cases in a dataset are of significance for two
reasons.  First, a learning method may take account of such
dependencies in order to improve learning.  For example, a method that
adapts its behaviour based on the size of the training set might
consider the effective size of the training set to be reduced when
training cases are dependent (since the information in one case may
largely duplicate the information in other cases).  Second, \delve{}
itself must be aware of possible dependencies in order to avoid
assessing learning methods using test cases that are dependent on the
cases included in the training set, and in order to properly compute
standard errors for performance figures.

Whenever a dataset has an informative ordering, there is the
possibility of \emph{sequential dependencies} between the cases.  In
some circumstances, however, this possibility may be remote enough
that it is reasonable to ignore it --- for example, if the cases are
ordered by the time when their attributes were measured by some
machine, it is possible that dependencies are present as a result of
temporal variation in the machine's accuracy, but this possibility
may be too remote to be worth worrying about.

Dependencies between cases may also exist whenever \emph{commonality
indexes} are present.  Cases with the same commonality index have
something in common of a nature that may produce dependencies.  For
example, suppose the problem is to classify cars by make, given an
image of the car.  If several cases were obtained by viewing the
\emph{same} car from different angles, the whole group of cases should
be used either for training or for testing, but not for a mixture of
these.  Otherwise, a test case might be correctly classified based on
some idiosyncratic feature of a training case in the same group (eg, a
scratch on the car's bumper).  Similarly, in a dataset of spoken
words, all the words spoken by one person would share a commonality
index.

The presence of commonality indexes or of an informative ordering is
merely an indication of the possibility of dependencies, and even if
dependencies exist, they may or may not be of significance in the
context of a particular learning task.  More specific information
concerning dependencies may be given in prototask and task
specifications.  When significant dependencies do exist, they are
dealt with in \delve{} in one of two ways.  One is to properly
accommodate the dependencies, as would be necessary in a real-world
learning task.  The other is to randomly select cases so as to produce
an internally-consistent task without dependencies.  Such tasks can be
useful for assessing learning methods even though they no longer
correspond to a real-world situations.  These issues are discussed
further in Section~\ref{sec-task}.

{\em Note: Currently, commonality indexes are not really implemented ---
you can include them in \delve{} dataset files, but they will be ignored.
Also, the only way of dealing with sequential dependencies at present 
is to randomize the ordering.}


\subsection{The \delve{} format for dataset files}\label{data-format}

DELVE datasets are stored in a standard format that is designed to
preserve as much relevant information from the original data as
possible, even if some of this information is not currently used by
DELVE.  Users may occasionally wish to look at these dataset files,
but programs implementing learning methods do not read these files
directly.  Instead, a learning method will work with data files that
have been appropriately encoded for a given task, as described in
Section~\ref{sec-assess}.

A dataset in the \delve{} standard format consists of an ordered list of
\emph{cases}, each of which consists of values for an ordered list of
\emph{attributes}.  A case may optionally be accompanied by a
\emph{comment}, which may be anything, and by a \emph{commonality
index}, a number that identifies several cases as having a common
origin.  \emph{Note: Commonality indexes aren't implemented\nolinebreak{} yet.}

The number of attributes is a characteristic of the dataset, and all cases 
have values (of some sort) for all attributes.  The value of an attribute
may be any of the following:\vspace{-5pt}
\begin{itemize} 
\item A string that represents a number in any of the common forms
      --- that is, with syntax
      \[
         [\ \mbox{\tt +}\ |\ \mbox{\tt -} \ ]\ 
         [\ digit\ldots\ ]\ [\ {\tt .}\ [\ digit\ldots\ ]\ ]\
         [\ (\ \mbox{\tt e}\ |\ \mbox{\tt E}\ )\ 
         [\ \mbox{\tt +}\ |\ \mbox{\tt -}\ ]\ digit\ldots\ ]
      \]
      with the restriction that at least one digit must appear, not 
      counting digits after an `{\tt e}' or `{\tt E}'.
\item A number as above, preceded or followed by `{\tt :}',
      representing a \emph{censored value}.  If the colon is at the end,
      the actual value of the attribute is known only to be greater than
      or equal to the given number; if the colon is at the beginning, the
      actual value is less than or equal to the given value. \emph{Note:
      Support of censored values is not yet implemented.}
\item The character `?', perhaps followed by other non-space characters.
      This represents a \emph{missing value}.
      The other characters may indicate the reason for the value being 
      missing.  Just `?' is used for values that are missing due to
      a random mechanism unrelated to the relationship of inputs to targets. 
      \emph{Note: Missing values are not really implemented yet. About the only
      thing useful that can be done at present with cases having missing 
      values is to ignore them.}
\item Any other string of non-space characters that does not begin with
      `{\tt $\backslash$}', `{\tt @}', `{\tt \#}', `{\tt (}', 
      `{\tt [}', `{\tt +}', `{\tt -}', `{\tt .}', `{\tt :}', 
      or a digit. These strings represent values from a discrete set of 
      categories.
\end{itemize}\vspace{-5pt}
Numerical values are represented in as close to their original form as
possible --- for example, `5.0' is \emph{not} converted to `5' or to
`5.00'.  This preserves any information that might be contained in the
original choice of the number of significant digits.

A dataset in standard format is encoded as a ASCII file, in which the
cases appear in order, with each case being represented by a group of
lines.  All lines in a group except the last end with a space followed
by the character `{\tt $\backslash$}'.  The whole group of lines for a
single case should be thought of in terms of the single line that
would result if the `{\tt $\backslash$}' and the following newline were
removed.  Within the line (or group of lines) representing a case, the
attribute values appear in order, separated by one or more spaces.

If a case has a commonality index associated with it, it appears after
all the attributes.  This index consists of the character `@'
followed by one or more digits.  

If a case has a comment associated with it, it appears at the end of
the line, preceded by `{\tt \#}'.  These comments are ignored by 
all \delve{} programs.


\subsection{Preparing a new dataset:~~The \dcheck{} command}\label{data-prepare}

When a dataset is obtained, the original data files, documentation,
programs, and any other possibly relevant material should be saved in
as close to its original form as possible.  This archived information
may be of interest if, for example, doubts should arise as to whether
the original data format was properly interpreted, or questions are
raised regarding the real-world relevance of the data.  This information
goes in the {\tt Source}\ sub-directory of the dataset's directory.

The dataset should then be converted to the standard \delve{} format,
and stored in the {\tt Dataset.data} file in the dataset's directory.
The aim in doing this should be to retain all information that could
be relevant to some use of the data, discarding only fields such as
redundant case numbers.  Converting a dataset will often be simply a
matter of mechanically reformatting it.  However, difficulties of
interpretation may arise if there are peculiar aspects to the original
data, or if it is inadequately documented.  In such cases, the
rationale for the decisions made should be documented, in the {\tt
Notes} file in the {\tt Source} directory for the dataset.

As well as the data file itself, you must create a specification file
for the dataset, with the name \texttt{Dataset.spec}, which describes how the
dataset is to be interpreted and used.  The specification file is
meant to be machine readable, and, as such, has a very strict format.
The file may have zero or more initial comment lines (lines where the
first character is a \verb+#+).  Immediately after the comments lines
there should appear the three lines (in any order):\vspace{-5pt}
\begin{Session}
Origin: {\rm \em origin}
Usage:  {\rm \em usage}
Order:  {\rm \em order}
\end{Session}\vspace{-5pt}
These lines specify the information discussed in Section~\ref{data-spec}.
Specifically:\vspace{-5pt}
\begin{list}{}{%
\setlength{\leftmargin}{0.9in}%
\setlength{\labelwidth}{0.5in}%
\setlength{\labelsep}{0.18in}}
\item[{\em origin\hfill}]
    should be one of the strings \texttt{natural},
    \texttt{cultivated}, \texttt{simulated}, or 
    \texttt{artificial}.\vspace{-2pt}
\item[{\em usage\hfill}]
    should be one of the strings \texttt{development},
    \texttt{assessment}, \texttt{historical}, or \texttt{?}.\vspace{-2pt}
\item[{\em order\hfill}]
    should be one of the strings \texttt{informative},
    \texttt{uninformative}, or \texttt{?}.\vspace{-5pt}
\end{list}\vspace{-5pt}
In addition to the above lines, you may include the optional line:
\begin{Session}
Title: {\rm \em title}
\end{Session}\vspace{-5pt}
where \textit{title} is a string describing the dataset.  It is not
used directly by \delve{}, but it is available to users via \texttt{dinfo}.

The string \texttt{Commonality indexes are present} may appear on the next
line.  If there are no commonality indexes, this line should be omitted.
{\em Note: Currently, this line must be omitted.  You can always include
commonality indexes, but they will be ignored.}

Following these lines should be a line contain the single string
\texttt{Attributes:}.  Each remaining line in the file will be
interpreted as an attribute description, with the format:
\[
\mbox{{\em i name control range} [ {\tt \#} {\em comment} ]}
\]
The fields above have the following meanings:\vspace{-3pt}
\begin{list}{}{%
\setlength{\leftmargin}{0.9in}%
\setlength{\labelwidth}{0.5in}%
\setlength{\labelsep}{0.18in}}
\item[{\em i\hfill}] is the integer index for the
    attribute. Indices should start at one and increment by one for
    each line.
\item[{\em name\hfill}] is a mnemonic name that can be used in place
    of the attribute's index.  The names must be unique (within a
    dataset).  They may not contain spaces, and may not look like 
    integers.
\item[{\em control\hfill}] is one of the characters \texttt{c} or
    \texttt{u}, depending on whether the attributes was controlled or
    uncontrolled.
\item[{\em range\hfill}] is the range for the attribute, a list of items of
    the form described in Section~\ref{data-spec}.
\end{list}\vspace{-3pt}
The range for an attribute may optionally be followed by `{\tt \#}'
and a comment describing the attribute.

The specification file for the \texttt{demo} dataset is shown in
Figure~\ref{fig:dataset-spec}.

\begin{figure}[t]
\begin{Session}
Origin: artificial
Usage: development
Order: uninformative
Attributes:
 1 SEX      u male female  # Sex of the person
 2 AGE      u [0,Inf)      # Age of the person in years
 3 SIBLINGS u 0..Inf       # Number of siblings the person has
 4 INCOME   u [0,Inf)      # The person's annual income (dollars)
 5 COLOUR   u pink blue red green purple  # The person's favourite colour
\end{Session}
\caption{Dataset specification file for the \texttt{demo} dataset.}
\label{fig:dataset-spec}
\end{figure}

Once you have created both \texttt{Dataset.data} and
\texttt{Dataset.spec}, you should check that the two are legal and
consistent using the \dcheck{} command, which will verify that each
case has the right number of attributes, and that they are in the
specified ranges. Note that missing values are allowed in
\texttt{Dataset.data} only if they are listed as allowed in
\texttt{Dataset.spec}.  A censored value for an attribute (specified
using `:') is allowed only if it includes at least one possible
value that is within the attribute's range.
\emph{Note: The \dcheck{} command is not implemented yet.}

%
% $Id: sec4.tex,v 1.27.2.3 1996/11/29 21:33:27 revow Exp $
%
\newpage

\section{FROM DATASETS TO TASKS}\label{sec-task}
\thispagestyle{plain}
\setcounter{figure}{0}
\chead[\fancyplain{}{\thesection.\ FROM DATASETS TO TASKS}]
      {\fancyplain{}{\thesection.\ FROM DATASETS TO TASKS}}

A dataset does not, by itself, define a problem to be solved.  To get
a well-defined learning task, we must specify additional information,
such as what part of the data we are concerned with, what we hope to
predict about this data, and what contextual information is available
to assist learning.  In the \delve{} environment, these specifications
have a hierarchical form, in which specificity increases as we go from
a \emph{dataset}, to a \emph{prototask}, to a \emph{task}, and finally
to a \emph{task instance}.

A \emph{prototask} fixes only the most basic aspects of the learning
task --- just enough so that it makes sense to compare the performance
of various learning methods on the various tasks that derive from the
prototask.  Specifically, a prototask will define the following:\vspace{-5pt}
\begin{itemize}
\item The \emph{subset of cases} that a learning method is expected to handle.
\item The set of \emph{target attributes} that the method is supposed to 
      predict, and the set of \emph{input attributes} that it may refer to
      when making these predictions.\vspace{-5pt}
\end{itemize}

A \emph{task} is derived from a prototask by specifying the additional
information required so that each learning method will have a
well-defined \emph{expected performance} on the task, with respect to some given
\emph{loss function} (see Section~\ref{sec-loss}).  In particular, to
define a task, we must supplement the specifications for the prototask
by specifying the following:\vspace{-5pt}
\begin{itemize} 
\item The \emph{number of training cases} in the training set
      that will be provided to the learning method, and (if applicable)
      whether this training set will be \emph{stratified} by target value.
\item The \emph{prior information} that the method may
      use to assist the learning.
\end{itemize}\vspace{-5pt}
Note that expected performance is estimated using \emph{task instances},
for which particular training cases are specified, as
discussed in Sections~\ref{sec-scheme} and~\ref{sec-assess}.

Specifications and other information relating to a prototask and its
tasks are kept in a sub-directory associated with the prototask,
located within the directory for the dataset.  Some of the files that
may appear within such a prototask directory are listed in
Figure~\ref{fig-prototask-dir}.

\begin{figure}[b]

\vspace*{-10pt}
\rule{\textwidth}{0.5pt}

\hspace*{-4pt}\begin{tabular}{ll} \\[-7pt]
{\tt Summary} & A brief description of the prototask \\
{\tt Prototask.data} 
  & Data relevant to the prototask, a subset of that in \texttt{Dataset.data} \\
{\tt Prototask.spec}
  & Specifications for the prototask and associated tasks, usually accessed \\
  & using the \dinfo\ command \\[5pt]
{\tt std.prior} & The ``standard'' prior information for the prototask \\[5pt]
$\!\!\!\left.\begin{array}{l} 
\mbox{\textit{Prior-1}\texttt{.prior}} \\ 
\mbox{\textit{Prior-2}\texttt{.prior}} \\ 
\mbox{\textit{Prior-3}\texttt{.prior}} \\ 
\end{array}\ \right\}$
  & Other specifications of prior information
\end{tabular}\vspace*{-1pt}

\caption{Some files that may appear within a \delve{} prototask 
         directory.}\vspace*{-10pt}

\label{fig-prototask-dir}
\end{figure}


\subsection{Specifications for prototasks and tasks:~~More on 
            \dinfo}\label{task-proto}

A supervised learning prototask is derived from a dataset by specifying
the set of attributes that are available for use as inputs, the set of
attributes that constitute the targets to be predicted, and any
restrictions on the types of cases for which the learning method is
expected to work.  It is possible to define many prototasks based on the
same dataset, involving different sets of inputs, targets, and cases.

Such prototask specifications are contained in files named {\tt
Prototask.spec} in the prototask directories.  Usually, users will not
look at such files directly, however, but will instead view the
information using \dinfo{}.  For example, the information displayed by
\dinfo{} for the \texttt{age} prototask of the \texttt{demo} dataset
is shown in Figure~\ref{fig:task-dinfo}.

\begin{figure}[t]
\begin{Session}
Prototask: /demo/age
Origin: artificial
Cases: all
Order: retain
Test set size: 1024
Training set sizes: 32 64 128 256 512
Test set selection: hierarchical
Maximum number of instances: 8
Inputs: 
     #  name     c/u range        description
     1  SEX       u  male female  Sex of the person
     3  SIBLINGS  u  0..Inf       Number of siblings the person has
     4  INCOME    u  [0,Inf)      The person's annual income (dollars)
     5  COLOUR    u  pink blue red green purple 
                                  The person's favourite colour
Targets: 
     #  name     c/u range        description
     2  AGE       u  [0,Inf)      Age of the person in years
Tasks: 
        std.32
        std.64
        std.128
        std.256
        std.512
\end{Session}\vspace{-4pt}
\caption{Output of the command: \texttt{dinfo /demo/age}.}
\label{fig:task-dinfo}
\end{figure}

The meaning of the prototask specifications displayed by \dinfo{} is as 
follows:\vspace{-5pt}
\begin{list}{}{%
\setlength{\leftmargin}{1.1in}%
\setlength{\labelwidth}{0.7in}%
\setlength{\labelsep}{0.1in}%
}
\item[{\tt Origin:}\hfill]
   {\tt natural} \OR {\tt cultivated} \OR {\tt simulated} \OR {\tt artificial}

The origin of a prototask and the tasks derived from it is usually the
same as that of the dataset on which the prototask is based.  For a
natural dataset, however, there will generally be only one or a few
natural prototasks, those that were of actual interest to the original
investigators.  Any substantially different prototasks that are based
on the same natural dataset are classified as cultivated. In
particular, all prototasks based on natural datasets in which the
effect of possible sequential dependencies among the cases has been
suppressed by random re-ordering are classified as cultivated.

\item[{\tt Cases:}\hfill]
   {\tt all} \OR {\tt no missing} \OR filename

This specifies which cases are to be included in the prototask.  The
special string \texttt{all} specifies that all cases are included in
the prototask.  The string \mbox{\texttt{no missing}} specifies that
all cases are included except those for which the values of one or
more attributes used by the prototask are missing.  Otherwise, the
cases to include are listed in the given file, as described in
Section~\ref{task-design}.


\item[{\tt Order:}\hfill]
   {\tt retain} \OR filename

The order in which cases for the prototask are to be used in
constructing training and test sets.  The specification may say to
\texttt{retain} the order in \texttt{Dataset.data}.  Alternatively,
the order may be as specifed in the given file; often this is
a file called \texttt{Random-order} containing a random re-ordering of 
cases. Section~\ref{task-design} for more details.

\item[{\tt Inputs:}\hfill] list

A list of indexes or names for attributes of the dataset that the
learning method is allowed (but not obliged) to use as inputs.

\item[{\tt Targets:}\hfill] list

A list of indexes or names for attributes of the dataset that the learning
method will attempt to predict.

\item[{\tt Test-Set-Size:}\hfill] size

The number of cases to be set aside for testing in the standard \delve{} set of
task instances.

\item[{\tt Training-Set-Sizes:}\hfill] list

A list of sizes for the training sets for the standard \delve{} set of tasks
associated with this prototask.

\item[{\tt Test-Set-Selection}:\hfill]
   {\tt hierarchical} \OR {\tt common}

Specifies how the test sets should defined for the standard set of task
instances. In the \texttt{heirarchical} scheme test sets for the different
instances are disjoint; in the {\tt common} scheme the same test set 
is used for all task instances.  See Section~\ref{sec-scheme} for further
details.

\item[{\tt Maximum-Number-Of-Instances:}\hfill]
   number

Specifies the maximum number of task instances used in the standard
\delve{} scheme.  This upper limit is used to prevent a very large number 
of instances being generated for the tasks with small training sets.

\end{list}\vspace{-5pt}

Note that the last four items above are not, strictly speaking,
specifications for the prototask, but rather for the standard set of
tasks and task instances that \delve{} defines for the prototask.

Attributes in the \texttt{Inputs:} and \texttt{Targets:} list may be
identified by number, starting with `1' for the first attribute in
the dataset, or by name.  An additional attribute, identified by
`0', is allowed for datasets with an informative order; its value is
the index of the case in the original ordering, starting with one for
the first case. (This index attribute is usually not an appropriate
input, but provision for its use is included for completeness.)
\emph{Note: Attribute `0' is not yet supported by the implementation.}

The ordering of cases in a prototask determines which cases will make
up the training and test sets of the various task instances for the
standard \delve{} set of tasks.  Most typically, we will want this
ordering to be random, to ensure that cases are effectively
independent (even if, in reality, there were dependencies between
cases as originally ordered).  This can be ensured by using a random
re-ordering, though one can also choose to \texttt{retain} the
ordering if it is certain that the original ordering is random (as
will often be the case for simulated or artificial datasets).

When the dataset is in an informative order, one may instead define a
\emph{sequential prototask}, in which this order is retained.  To
avoid certain complications, sequential prototasks are not allowed
when the cases also have commonality indexes.  In order to allow an
appropriate selection of training and test sets, the prototask
specification must include a maximum range over which there may be
non-negligible sequential dependences that are relevant to the
supervised learning task.  Note that this may be less than the range
over which there are dependencies in the input attributes, as it is
only dependences in the noise in the relationship between inputs and
targets that are relevant.  This maximum range should be set on the
high side, to ensure that the performance assessments are not biased.
A sequential prototask should not be defined if it is thought that the
range of relevant dependencies may be comparable to the number of
cases available.  \emph{Note: Sequential prototasks are not yet
supported by the implementation.}

\subsection{The size and nature of the training set for a task}
\label{task-training-size}

Potentially, a researcher might wish to assess the the performance of
learning methods on a task with any number of training cases, up to
the maximum that is feasible given the number of cases in the dataset.
It is unrealistic, however, to expect all researchers to test their
methods on training sets of all possible sizes. \delve\ therefore
defines a relatively small set of training set sizes for each
prototask, which we hope will be adequate for most purposes.

The smallest standard training set size is chosen to be the smallest
that the designer of the prototask believes might be sufficient for a
learning method to learn something interesting.  The larger standard
training set sizes are bigger than this smallest size by powers of
two, up to a maximum size limited by the need to reserve an adequate
test set.

For non-sequential prototasks with a single target taking values from
a finite set, \delve{} also provides the option of specifying that the
training set for a task should be \emph{stratified} by target value
--- that is, that the training set will contain equal numbers of cases
with each target value.  The size of a stratified training set must be
a multiple of the number of target values.  Stratification is natural
in applications such as handwritten digit recognition, for which
training data would often be collected in a fashion that ensured that
there were equal numbers of cases for each digit.  The expected
performance of a task with a stratified training set will be based on
a distribution of test cases in which all values of the target are
equally likely. \emph{Note: Support for stratification is not yet
implemented.}


\subsection{Prior information available for a task}\label{task-prior}

Learning can be (some would say, must be) assisted by the provision of
prior information about the relationship to be learned.  For real
applications, all available prior information should be used to
improve performance, to the extent that it can be accommodated by the
learning method.  But for research into the performance of learning
methods, it is not desirable for each researcher to employ whatever
prior knowledge they may happen to have about the problem, as the
results obtained by different researchers would then not be
comparable.

Each DELVE task specification therefore includes an explicit
specification of the prior information that is to be regarded as
available for use by a learning method.  Researchers who happen to
know something about the real-world context of the problem beyond what
is specified should \emph{not} use such additional information to
improve the performance of their learning methods.  Indeed, if they
happen to know that some of the prior information specified for the
task is incorrect, they should still use this information as if they
believed it to be true, despite any bad effects this might have on
performance. (They could, however, create a new prior specification
that reflects their knowledge, and apply their method to tasks based
on this new prior.)

Although prior information for real tasks can take many forms,
\delve{} supports only prior information that is specified in the
semi-formal form described below.  Most of this prior information is
associated with the various input and target attributes for the
prototask, and is used to determine the default encodings of
attributes, as discussed in Section~\ref{assess-encodings}.  A
learning method that uses the default encodings will therefore
implicitly be making use of the prior information.  A learning method
may employ some other way of selecting encodings based on the prior
information, however, and may also use prior information in
other ways.

A prototask will typically come with a ``standard'' prior
specification, stored in the file \texttt{std.prior}, which generally
will be fairly unspecific (eg, will be vague about how relevant the
various inputs are).  Other specifications of prior information may
also be defined, stored in other files ending in \texttt{.prior}.  A
learning task within a prototask is specified by giving both the name
of a prior specification and the number of training cases used, for
instance, \texttt{std.128}.  The prior for a task can be viewed using
\dinfo{}, as illustrated in Figure~\ref{fig:task-prior}.  The output
also shows the default encodings derived from this prior information,
as explained in Section~\ref{assess-encodings}.

\begin{figure}[t]
\begin{Session}
Task: /demo/age/std.128
Training set size: 128
Inputs: 
 col attr name          type   relevance  def coding  options
   1   1  SEX           binary   nlmh       -1/+1        -
   2   3  SIBLINGS      integer  nlmh       nm-abs       -
   3   4  INCOME        real     nlmh       nm-abs       -
   4   5  COLOUR:pink   nominal  nlmh       1-of-n       -
   5   5  COLOUR:blue                  ...
   6   5  COLOUR:red                   ...
   7   5  COLOUR:green                 ...
   8   5  COLOUR:purple                ...
Targets: 
 col attr name         type    noise-lev def coding  options
   1   2  AGE           real     nlmh       nm-abs       -
\end{Session}\vspace{-4pt}

\caption{Output of the command: \texttt{dinfo /demo/age/std.128}
        }\label{fig:task-prior}
\end{figure}

Note that the explanations of prior specifications given below are
meant only as rough guides to their meanings.  The precise,
quantitative representation of prior knowledge is, after all, a topic
for ongoing research in learning.  Note also that none of these prior
specifications should be taken as indicating absolutely certain
knowledge; they mean only that it is considered very likely that the
true situation conforms to the specification.

{\bf Noise in targets.\/} The amount of inherent noise that is thought
to affect the values of a target is specified using one or more of the
characters `N', `L', `M', and `H', representing no, low,
medium, or high noise.  If more than one character is specified, the
amount of noise is uncertain.  For example, a specification of `NLM'
indicates that there might be no noise at all, or there might be a low
or medium amount of noise, but it is thought that there is not a high
amount of noise.

If a target is noise-free, its value will be the same in all cases
where the input attributes are the same.  This does not imply that the
target can be always be predicted with certainty on the basis of
information from a finite training set, since there may be no training
case with inputs that match a particular test case.  It means, rather,
that it \emph{would} be possible to predict the target with certainty
if we had enough training data.  For many prototasks, the inputs will
be different for every case that is actually available, so that the
characterization is hypothetical in nature (as is the case below as
well).

A real-valued target is said to have a low amount of inherent noise if
the spread in the distribution of target values over cases where the
inputs are all the same is roughly 1\% or less of the spread of target
values for all cases.  For a target with a medium amount of noise, the
spread for particular values of the inputs is roughly 10\% of the
overall spread.  For targets with a high amount of noise, the figure
is substantially higher, perhaps approaching 100\%.  Here, the spread
is assumed to be measured in a unit such as standard deviation, but
the term is left deliberately vague, as it could be, for example, that
the standard deviation is not defined for a target that takes on
occasional extreme values.  The intent is that the rough figures of
1\% and 10\% should be interpreted with respect to some intuitively
appropriate notion of spread.

For discrete targets, a low amount of noise means that the target
value differs from that which is most common for the given inputs
about 1\% or less of the time, with the corresponding figure for
medium noise being about 10\%, and for high noise something
substantially greater than that.

{\em Note: At present, the noise level specified does not affect the
default encoding, but this may soon change.  For the moment, it is
probably best to always specify a noise level prior of `NLMH', as it
is expected that the default coding with this specification will not
change in the future.}

{\bf Dependencies between cases.\/} For a sequential prototask, or a
prototask based on a dataset containing cases with commonality
indexes, the prior specification for a task must include information
on the anticipated strength of any dependencies between cases with the
same commonality index, or which are close to each other in sequential
order.  This specification will consist of one or more of the
characters `N', `L', `M', and `H', representing the possibility of no,
low, medium, or high dependencies.

If there is a high degree of dependence between such cases, knowing
the true target for one case would, if the true nature of the
relationship were known, permit one to predict the target in another
case that is nearby, or has the same commonality index, with an
accuracy that is better than would be possible without knowing the
true target for such another case, by a factor of around 100 or more
(in terms of some intuitively appropriate measure of ``spread'' such
as discussed above for noise levels).  For a medium degree of
dependence, the corresponding factor would be around 10, and for a low
degree of dependence, much less (perhaps around 2).  If there is
``no'' dependence, very little or no improvement in predictions would
be possible from knowing the true target in another case that is close
in sequential order, or that has the same commonality index.

For a sequential prototask, the maximum range over which it is thought
that non-negligible dependencies may occur will also be specified as
part of the prior information.  This maximum range will often be the
same as that specified in the prototask specification, but might
differ if the effect of changing this aspect of the prior information
is being investigated.  Note that it is possible for dependencies to
persist over a long range even if the magnitude of these dependencies
is low.  It is usually reasonable, however, to expect that the
strength of the dependencies will likely decline at least somewhat
with increasing range, even before the maximum is reached.

{\em Note: These prior specifications regarding dependencies between
cases have not yet been implemented.}

{\bf Relevance of inputs.\/} The degree of relevance that an input
attribute is thought to possess is specified using one or more of the
characters `N', `L', `M', and `H', representing no, low,
medium, or high relevance.  If more than one character is specified,
this indicates that the degree of relevance is uncertain, except that
it is likely to be in one of the categories mentioned.

The meaning of degree of relevance can be explained in terms of the
variation in target valuess, after the component of the variation due
to inherent noise is eliminated.  An input is considered to be of high
relevance if as it varies over the range of values that may actually
occur in combination with the other input values (which are kept
fixed), the target attributes often vary over close to their full
range (discounting variation that is due to inherent noise).  The
effects of some inputs may depend on the values of other inputs.  To
be considered highly relevant, it is not necessary that the input
always have a big effect; only that it does so in many of the cases.
Note the mention above of the range of values for the input that
actually occur in conjunction with the other inputs.  It may sometimes
be known that an input would have a big effect if it were to take on
an extreme value, but this does not make the input highly relevant
unless such extreme values are likely to actually occur.

An input is considered to be of medium relevance if it can have a
somewhat smaller effect on the targets --- say, changing them by about
10\% of their range.  Variation in inputs of low relevance might
affect the targets to the extent of about 1\% of their range.  Inputs
of ``no'' relevance have substantially less effect (perhaps none).

Learning methods may use prior information about relevance in various
ways.  A Bayesian method might use this information to set up a prior
distribution for model parameters.  A method prone to ``overfitting''
might reduce the number of model parameters when the training set is
small by looking only at inputs thought to be highly relevant .

{\em Note: At present, the relevance specification does not affect the
default encoding, but this may soon change.  For the moment, it is
probably best to always specify a relevance prior of `NLMH', as it
is expected that the default coding with this specification will not
change in the future.}

{\bf Binary attributes.\/} An input or target attribute that takes on
only two possible values (not counting missing values) can be
specified to be either {\em symmetric\/} or {\em active-passive}.  

For a symmetric binary attribute, nothing is known about the two
possible values that would justify treating one differently from
another.  The actual significance of the two values may be quite
different, however --- we just have no prior knowledge of which way
around the effects might go.

For an active-passive attribute, one of the two values is specified to
be \emph{passive}; the other is then \emph{active}.  Exactly what this
means will depend on the problem; the general concept is best defined
by an example.  In a medical diagnosis task, binary input attributes
indicating whether the patient has fever, chest pain, and yellow
toenails are active-passive, with the presence of the symptom being
the active value.  We expect that the presence of such a symptom will
have specific diagnostic implications, pointing to a relatively small
class of diseases.  In contrast, the absence of fever does not in
itself suggest a diagnosis.  For a binary target, the ``passive''
value is considered to be the ``default'', though this does not
necessarily mean that it occurs more often than the ``active'' value.

What, if anything, the distinction between symmetric and
active-passive attributes should mean for the proper treatment of
binary inputs and targets is a matter for researchers developing
learning methods to judge.  However, the default DELVE encodings
(see section~\ref{assess-encodings}) do treat symmetric inputs
symmetrically, and active-passive inputs asymmetrically.

{\bf Categorical attributes.\/} An input or target attribute that
takes on a finite number of possible values (three or more, not
counting missing values) may be specified to be \emph{nominal} or
\emph{ordinal}.  This distinction affects the default encoding
of the attribute, as discussed in Section~\ref{assess-encodings}.

The values of a nominal attribute are significant only in that they
are distinct from one another, except that one of the values may
optionally be singled out as the \emph{passive} value.  The meaning
of such a passive specification is analogous to that described above for
binary attributes.

The values of an ordinal attribute have a defined ordering, which must
be specified, if it differs from the order in which the possible
values are listed in the dataset specification.  The first value 
in this ordering may optionally be specified to be \emph{passive}.
\emph{Note: There is currently no way of overriding the ordering of
attribute values in the dataset specification.}

{\bf Real-valued attributes.\/} Currently, no specific prior
information pertaining to real-valued attributes is recorded, other
than the noise level and degree of relevance, as discussed above.
Formal specification of prior information regarding promising
transformations of real-valued input and target attributes may be
allowed in future.  The expected degree of smoothness in the
relationship between a real-valued input attribute and the targets
might also be useful prior information, but this also has not been
standardized.  In the absence of such information, it is appropriate
to assume that relationships are often smooth, or at least continuous,
but that discontinuities are not impossible.

{\bf Integer attributes.\/} At present, no special special prior 
specifications are defined for integer attributes.  The relevance
and noise level priors apply, however.

{\bf Angular attributes.\/} Numeric attributes interval can be
specified to be \emph{angular}.  These attributes are thought to have
a circular meaning, for which all that matters is the modulus of the
value with respect to some unit.  For instance, a attribute giving the
time of day could be considered to be angular, with a modulus of 24
hours.

Angular attributes are by default encoded in terms of the sine and
cosine of the angle they define (see Section~\ref{assess-encodings}).
This representation respects the assumed continuity as values wrap
around.

 
\subsection{Defining prototasks:~~The \dgenorder{} and \dgenproto{} 
            commands}\label{task-design}

Before a new dataset can be used to assess learning methods in
\delve{}, at least one prototask must be defined for it.  Researchers
may also wish to define new prototasks for existing datasets.  This
section describes how to do these things, as well as the approach
taken in defining the standard \delve{} prototasks.

The purpose of defining a prototask is to support interesting
experiments, which say something significant about the learning
methods that are assessed.  For some datasets, such interesting
prototasks may need to have special features. For example, if a
potential input attribute is very highly correlated with a target
attribute, it may be best to leave it out of the allowed set of input
attributes, in order to prevent the prototask from being so easy that
it is uninteresting.  If the inputs in a few cases differ greatly from
those in the other cases, it might be of interest to define a
prototask that excludes cases with these extreme inputs, in order to
assess learning methods that do not purport to handle such
extrapolation well.  The documentation for a prototask with unusual
features should include a statement of the research questions the
prototask is meant to address, and a justification for its
specifications in terms of these objectives.

Most standard \delve{} prototasks are defined with no specialized
objectives in mind, however, and include all attributes and all cases.
Complications due to \emph{missing data} arise fairly often, however.
Since many of the supervised learning methods we would like to assess
do not naturally handle missing data, we hope to obtain a good
collection of \delve{} prototasks in which the values of input
attributes are never missing.  We expect that this will require
creating some such prototasks by excluding a few input attributes
whose values are missing in many cases, or by excluding a few cases
for which the values of one or more attributes are missing, or by
doing a bit of both.  

The designer of a prototask decide how to deal with any dependencies
between cases that may be present.  We take two approaches to this for
the standard \delve{} prototasks.  For some prototasks, we accommodate
the dependencies in a proper fashion (\emph{or rather, we will do so
once the required facilities are implemented}).  In particular, we
ensure that there are no significant dependencies between training and
test cases, as this would invalidate the results.  Other times,
however, we instead circumvent sequential dependencies by randomly
reordering the dataset.  This second approach allows us to define
tasks for which ignoring dependencies gives internally consistent
results, although such tasks no longer correspond to real-world
situations.

When a non-sequential prototask is defined it is recommended that the
cases always be randomly re-ordered, unless it is known for certain
that the existing order is random.  Certainly this must be done if the
ordering is \texttt{informative}, or is sorted by some attribute
value.  It should also be done even if it is thought that the order is
arbitrary, in order to provide greater certainty that assessments
based on the assumption of no sequential dependence will be internally
valid. When cases have commonality indexes, this random re-ordering
must keep cases with the same index grouped together (in random
order), while randomly ordering the groups themselves.

To create a prototask, you first must create a directory for the
prototask within the \delve{} hierarchy.  This directory must have the
same name as the new prototask, and be located within one of the
directories for the dataset in the \delve{} hierarchy.  Within this
prototask directory, you must create a \texttt{Prototask.spec} file,
containing the specifications for the prototask and the standard set
of tasks associated with it, and also one or more files containing
prior specifications, usually including \texttt{std.prior}, which
contains the ``standard'' prior information.

These files have formats paralleling the output of \dinfo{} for a
prototask and for a task.  A \texttt{.prior} file should have one line
per attribute, specifying the attribute number, the noise level or
relevance prior, the type of the variable, and any additional options.
For example, the line for a nominal attribute, numbered 2, thought to
be of at least medium relevance, and which has a passive value of
\texttt{none}, would be be\vspace{-5pt} \begin{Session}
  2 MH nominal passive=none
\end{Session}\vspace{-5pt}
An angular attribute must be accompanied by a \texttt{unit={\rm\em modulus}}
specification.

The \texttt{Prototask.spec} and \texttt{std.prior} files for the
\texttt{/demo/age} prototask are shown in Figure~\ref{fig:task-spec}.

\begin{figure}[t]
\begin{Session}
        Prototask.spec                                std.prior

Cases: all                                         1 NLMH binary
Inputs: 1 3 4 5                                    3 NLMH integer
Order: retain                                      4 NLMH real
Origin: artificial                                 5 NLMH nominal
Targets: 2                                         2 NLMH real
Test-Set-Size: 1024                        
Training-Set-Sizes: 32 64 128 256 512      
Test-Set-Selection: hierarchical           
Maximum-Number-Of-Instances: 8             
\end{Session}\vspace{-4pt}
\caption{Prototask specification (\texttt{Prototask.spec}) and standard 
         prior specification (\texttt{std.prior}) for the \texttt{age} 
         prototask of the \texttt{demo} dataset.}
\label{fig:task-spec}
\end{figure}

The \texttt{/demo/age} prototask includes \texttt{all} cases in the
dataset.  Another built-in option is \mbox{\texttt{no missing}}, which
specifies that all cases should be included except those for which one
or more of the attributes used in the prototask are missing.  One can
also give for \texttt{Cases} the name of a file that contains an explicit
list of case numbers to include, one case per line, with numbers
starting at one.  The order of lines in this file does not matter.
This case file should be located in the \delve{} hierarchy, within the
prototask directory.

The order for the \texttt{/demo/age} prototask is specified as
\texttt{retain}.  This prototask is non-sequential, but the data is
artificially generated in a fashion that guarantees that the original
data file is in random order.  For a natural or cultivated dataset,
one would normally randomize the ordering explicitly (assuming that
the prototask is not meant to be sequential).  One does this by
specifying a file that contains such an ordering.  This file must be
located in the \delve{} hierarchy, within the prototask directory.
The order file should have one line per case, with each line
containing the index of a case.  Indexes start at one, and go up to
the number of cases that are used in the prototask.  Note that if any
cases were left out of the prototask, these will \emph{not} be the
indexes of the cases in \texttt{Dataset.spec}.

Most often, this ordering file will be called \texttt{Random-order},
and will be generated automatically using the \dgenorder{} command.
This command will also take care of the complications involved in
handling commonality indexes.  \emph{Or at least it will once
commonality indexes have been properly implemented}

Another command that you will often wish to use after creating a new
prototask is \dgenproto{}, which creates the intermediate file
\texttt{Prototask.data}, containing the portion of
\texttt{Dataset.data} relevant to this prototask.  This intermediate
file will be created ``on-the-fly'' by other commands, as needed, but
creating a single permanent copy will save time.  You will also
want to use the \dcheck{} command, in order to check that the 
prototask specifications are consistent with the dataset specifications.

Here is how you would go about creating a non-sequential prototask for
a natural dataset:\vspace{-4pt}
\begin{Session}
unix> cd \textit{dataset}           # Change to a delve directory for the dataset 
unix> mkdir \textit{prototask}      # Create a directory for the new prototask
unix> cd \textit{prototask}         #   and change into it
unix> edit Prototask.spec  # Create the prototask specification file
unix> edit std.prior       # Create the standard prior specification 
unix> dcheck               # Check that it's all consistent 
unix> dgenorder            # Generate the Random-order file
unix> dgenproto            # Generate the Prototask.data file
\end{Session}\vspace{-4pt}

The \dgenproto{} step is optional, but usually advisable; if it is
done, it must be after \dgenorder{} has been done.  For a simulated or
artificial dataset, where the cases are already in random order, the
ordering would usually be \texttt{retain}, and the \dgenorder{} step
would be omitted.  \emph{Note: The \dcheck{} command is not 
implemented yet, so you will have to leave out that step at present.}

%
% $Id: sec5.tex,v 1.15 1996/05/02 18:26:42 radford Exp $
%
\newpage

\section{PREDICTIONS AND LOSS FUNCTIONS}\label{sec-loss}
\thispagestyle{plain}
\setcounter{figure}{0}
\chead[\fancyplain{}{\thesection.\ PREDICTIONS AND LOSS FUNCTIONS}]
      {\fancyplain{}{\thesection.\ PREDICTIONS AND LOSS FUNCTIONS}}

Together, the specifications for a prototask and for one of its tasks
determine what is to be learned and what information will be available
on which to base learning.  To complete the specification of a
learning problem, we need to say what form the output of a learning
method should take, and how the performance of a method on a task will
be judged.

DELVE supports assessments only of the predictive performance of
learning methods --- the degree to which the relationships learned
can be used to predict attributes in previously unseen cases.  For
this purpose, the relevant output of a supervised learning method
is a set of {\em predictions\/} for the target attributes in a set of
{\em test cases\/} for which only the input attributes are known.  The
accuracy of these predictions is judged by how well they match the
actual values of the targets, as measured by some {\em loss function}.

For some methods, learning, making predictions, and judging the
loss from these predictions may be sequential activities, with the
nature of the predictions required having no effect on the learning
itself, and with the loss function by which these predictions will be
judged having no effect on the predictions themselves.  In general,
however, this need not be so.  A learning method may be designed to
behave quite differently depending on the predictions that it will be
required to produce, or on the loss function by which these
predictions will ultimately be judged.


\subsection{Types of predictions}\label{loss-pred}

DELVE expects learning methods to produce predictions in the form
of either {\em guesses\/} or {\em predictive distributions}.  A real
application might require either type of prediction, and many learning
methods will be able to produce predictions of both types.

A {\em guess\/} for a target in a test case is a value of the same
type as the target itself --- that is, if the target is categorical,
the guess will be one of the possible target values, and if the target
is numerical, so will the guess be (though a guess for an integer
target need not be an integer).  If there is more than one target
attribute, a separate guess is made for each target.  One might
sometimes wish to allow a learning method to decide to make no
guess for a target (at a penalty); provisions for this are described
in Section~\ref{loss-specialized}.

The accuracy of a guess is judged by a loss function that measures how
close the guess is to the true value, as described below in
Section~\ref{loss-standard}.

A {\em predictive distribution\/} is a probability distribution for
the targets in a test case, conditional on the known values of the
inputs for the test case.  In theory, a learning method that
produces predictions of this form should output a complete
representation of the predictive distribution for each test case.
Given this distribution and the actual value, a loss could then be
computed using one of the loss functions described below
(Section~\ref{loss-standard}).

However, the predictive distribution for a target produced by a
learning method could be arbitrarily complex (at least for real-valued
targets).  When there is more than one target, the predictive
distribution might in general involve dependencies between targets.
Due to the difficulty of defining a standard representation for
predictive distributions that is both convenient and sufficiently
general, \delve\ does not require learning methods to actually output
their predictive distributions.  Instead, the computation of the loss
based on the predictive distribution and the actual target values is
left for the learning method itself to compute, using its internal
representation of the predictive distribution.

Tasks with a single categorical target are an exception to this
general procedure.  In this case only, a learning method may output
an explicit representation of the predictive distribution for each
test case, as described in Section~\ref{assess-mloss}, leaving the
computation of losses to \delve.  This is in fact the preferred
procedure, since it makes the predictive distributions available for
examination, and avoids the possibility that the learning method
will compute the losses incorrectly.


\subsection{Standard loss functions supported by DELVE}\label{loss-standard}

The accuracy of a prediction for a test case is measured by a {\em
loss function\/}, which takes two arguments:\ \ The prediction output
by the method for a particular test case, and the true values of
the targets for that case.  The value of the loss function is a single
real number that represents the ``loss'' suffered when the given
prediction is used in a situation where the true values of the targets
are as given.  

Note that the loss function is defined in terms of a single test case,
not a set of test cases.  The goal of prediction is to minimize the
expected value of this loss on a test case that is randomly drawn from
the distribution of cases defined for the prototask.  In assessing the
performance of a method, we will of course use test sets with many
cases, taking the average loss over many test cases as an estimate of
the expected loss on a single test case.

The loss function for an actual application might sometimes be quite
complex and specialized.  DELVE does not attempt to assess methods for
producing predictions in such a context, but concentrates instead on a
predictions that will be judged using a few simple loss functions.
These loss functions have been selected because they are already in
common use, and because they emphasize somewhat different aspects of
predictive performance.  The performance of a learning method with
respect to these standard loss functions can be compared to that of
the many other methods that will have been assessed with the same
loss functions.  More specialized loss functions may be of interest
for some prototasks, however, and DELVE does provide some support for
them, as is described in Section~\ref{loss-specialized}.

Each of the standard loss functions has a one-letter abbreviation.
This abbreviation is used to specify a loss function, and occurs in
the standard names for files holding predictions and losses on a task
instance, as is described further in Section~\ref{sec-assess}.  The
standard loss functions are summarized in Figure~\ref{fig-losses}.

\begin{figure}[t]

\begin{center}\begin{tabular}{lccccc}
& {\em abbrev.} 
& {\em categorical?} & {\em integer?} & {\em ~real?~} & {\em angular?} \\[-4pt]
{\em For guesses:} \\
~~~Squared-error loss      & S &         & $\surd$ & $\surd$ &         \\
~~~Absolute-error loss     & A &         & $\surd$ & $\surd$ &         \\
~~~0-1 loss                & Z & $\surd$ & $\surd$ &         &         \\[6pt]
{\em For predictive distributions:} \\
~~~Log-probability loss    & L & $\surd$ & $\surd$ & $\surd$ & $\surd$ \\
~~~Squared-probability loss& Q & $\surd$ &         &         &         \\
\end{tabular}\end{center}

\caption{Standard loss functions, their abbreviations, and the
types of targets for which they can be used.}
\label{fig-losses}
\end{figure}

For predictions that take the form of guesses, the standard loss
functions are all based on a ``distance'' of some kind between a guess
and the true value of a target.  For tasks with more than one target,
the total loss is simply the sum of the losses based on the distance
of each target guess from the true target value.

For guesses of targets that take on integer or real values, DELVE
supports two loss functions, based on squared and absolute distance.
The {\em squared-error loss\/} is the square of the difference between
the guess and the true target value.  Those who take a probabilistic
approach to learning should note that the expected squared-error loss
is minimized by guessing the mean of the predictive distribution for
the target.  The {\em absolute-error loss\/} is the absolute value of
the difference between the guess and the true target value.  The
expected absolute-error loss is minimized by guessing the median of
the predictive distribution.

For guesses of integer and categorical targets, DELVE supports {\em
0-1 loss\/}, in which the loss is zero if the guess is correct, and
one if it is incorrect.  The optimal strategy for minimizing 0-1 loss
is to guess the target value with greatest probability (the mode of
the predictive distribution).

DELVE does not currently support any loss functions for guesses of
targets that take on angular values.  There is also no provision for
using different loss functions for the various targets in a case.

For predictions that take the form of a predictive distribution, one
may use {\em log probability loss}, which is minus the log (to base
$e$) of the probability or probability density of the true target
values under the predictive distribution.  Log probability loss may be
used with targets of any kind.  Note that if all targets are integer
or categorical, the predictive distribution will consist of
probabilities for the various combinations of target values.  If
instead the targets are all real or angular, the predictive
distribution will take the form of a probability density (which must
be finite if log probability loss is to be used).  If some targets are
integer or categorical and others are real or angular, the log
probability loss will be computed from the hybrid probability/density
of the true target values.

{\em Squared-probability loss\/} may be used with predictive
distributions for a single categorical target.  In this case, the
prediction takes the form of a list of probabilities, $p_1, \ldots,
p_n$, for the possible target values, which are labeled $1$ to $n$,
with $t$ being the true target value for the case in question.  (As
mentioned in Section~\ref{loss-pred}, in this case only, the learning
method may produce the predictive distribution explicitly.)  The
squared probability loss is the square of one minus the probability
assigned to the true target value, plus the squares of the
probabilities assigned to all the other possible target values ---
that is, $(1\!-\!p_t)^2\ +\ \sum\limits_{i\ne t}p_i^2$.\vspace{-3pt}

Note that the expected value of both the log probability loss and the
squared probability loss is minimized by a distribution matching the
true probabilities.  The log probability loss will be infinite if
the probability or probability density for the true target is zero,
but the squared-probability loss is never greater than two.


\subsection{Using a specialized loss function}\label{loss-specialized}

{\em Note: The facilities described in this section have not yet been
implemented.}

In addition to the standard predictions and loss functions described
above, DELVE supports specialized predictions in which guessing is
optional, and specialized loss functions defined by an arbitrary loss
matrix.  These facilities are intended for use with natural prototasks
that come from application areas where such specialized predictions
and loss functions are appropriate, or with cultivated or synthetic
prototasks that are intended to mimic such actual applications.  For
example, an automatic postal code recognition system may have the
option of referring hard-to-recognize postal codes to a human worker,
and in a medical testing application, we might wish to treat a false
positive as less serious than a false negative.

Guessing can be made optional by specifying a {\em no-guess penalty},
which is the loss suffered when the learning method decides to make no
guess --- presumably because the method is so uncertain of the value
of the target that it expects the loss produced with its best guess to
be greater than the no-guess penalty.  This form of prediction and
loss function is specified by appending the value of the no-guess
penalty followed by ``N'' to the abbreviation of any of the loss
functions for guesses in Figure~\ref{fig-losses}.  For example,
``Z0.2N'' specifies 0-1 loss with a penalty of 0.2 for not making a
guess.

A non-standard loss function for guesses of a single categorical target
can be specified by means of a {\em loss matrix}, in which the
loss for every possible combination of a guess and a true value for
the target is explicitly specified, with the restriction that the
losses must be non-negative, and be zero when the guess is correct.
A loss for making no guess may also be specified, separately for
each true target value.  

Use of a loss matrix is specified by giving ``M'' followed by a file
name wherever you would otherwise use an abbreviation for a standard
loss function.  This file should be located in the {\tt data} part of
the \delve{} hierarchy, in the directory for the corresponding
prototask.  The file should contain as many lines as there are
possible values for the target attribute, plus one additional line if
the method is to be allowed to make no guess.  The lines correspond to
possible guesses, according to the ordering of possible attribute
values in the dataset specification.  Each such line should contain
numerical values for the losses suffered for each possible true value
of the target, again in the order given by the dataset specification.


% For reasons that are not apparent, the following has to go here in order to
% produce a page break after Section 6.

\addtocontents{toc}{\protect\newpage\protect\vspace*{1pt}} 

%
% $Id: sec6.tex,v 1.9.2.1 1996/11/25 14:34:08 revow Exp $
%
\newpage

\section{SCHEMES FOR LEARNING EXPERIMENTS}\label{sec-scheme}
\thispagestyle{plain}
\setcounter{figure}{0}
\chead[\fancyplain{}{\thesection.\ SCHEMES FOR LEARNING EXPERIMENTS}]
      {\fancyplain{}{\thesection.\ SCHEMES FOR LEARNING EXPERIMENTS}}

Tasks are sufficiently well-defined that each learning method has a
well-defined expected performance on each task, which is the expected
value of some specified loss function on a randomly selected test
case, when using the predictions produced by the learning method based
on the given prior information and a random training set of the
specified size.  We do not, and never will, know this expected
performance exactly, but we can estimate the expected performance by
performing experiments in which we apply the learning method to
several {\em task instances}, each of which has a particular set of
training cases, and a particular set of test cases.

There are many possible schemes for defining task instances, with
different advantages and disadvantages.  This section describes the
standard schemes used in \delve{}, and discusses why we chose this
scheme for experiments.


\subsection{Issues in designing learning experiments}\label{scheme-issues}

For research purposes, we are usually interested not so much in the
numerical value of the expected loss for a method applied a task, but
rather in the relative performance of several learning methods on the
same task.  Such performance comparisons can be done more accurately
if the various performance estimates are all based a common set of
task instances, in which the training and test sets contain the same
cases.

The statistical benefits of such a common set of task instances are
discussed further in Section~\ref{sec-analysis}.  In this section, we
describe the standard scheme used in \delve{} to define a common set
of task instances for each task.  This scheme has been designed not
only to allow for good estimates, and an indication of their accuracy,
but also to limit the number of task instances, and hence the number
of times that a learning method must be applied to a training set in
order to obtain a performance estimate for a task.  Minimizing the
number of applications is important if sophisticated learning methods
are to be evaluated, which may, at least in early stages of research,
be computationally intensive.  It is even more important for learning
methods that involve decisions made by a human analyst.  In order to
achieve these goals, we have been willing to forgo the use tasks based
on small datasets, as we believe that any research questions that
these datasets might be useful in answering can equally well be
addressed using larger datasets.

Two different situations arise depending on whether we are dealing
with real or synthetic datasets. For real datasets the number of
available cases is often a limiting factor, and it therefore seems
best to use large a single common test set for all instances --- what
is referred to in \delve{} as a \emph{common} testing scheme.  On the
other hand, if we are dealing with synthetic data, it is usually
possible to generate an unlimited amount of data for testing, and in
this case the limiting factor will be the disk space needed to store
the prediction and loss files for all the applications of methods to
task instances.  In this case it seems more profitable to use disjoint
test sets for different instances, allowing a much larger number of
test cases in total for a given amount of disk storage. This is what
we call the \emph{hierarchical} testing scheme.

The standard \delve{} schemes for defining task instances are
certainly not the only possible ways of estimating expected
performance, however.  Some researchers may prefer to use some other
scheme, such as leave-one-out cross-validation.  One may also wish to
evaluate the performance of a new method on exactly the same task
instances as were used to evaluate some older method.  For these
reasons, we allow users to specify non-standard task instances, which
will enable them to perform such evaluations using the \delve{}
facilities described in Section~\ref{sec-assess}.  \emph{Note: This
facility isn't implemented yet, however.}


\subsection{DELVE's standard set of task instances}\label{scheme-standard}

In the standard set of tasks for each prototask, the training set size
is one of a series of numbers that differ by factors of two.  The
designer of a prototask based on some dataset might, for example, have
specified standard tasks with training set sizes of 20, 40, and 80.
The same range of training set sizes, and the same actual training and
test sets, are used for all specifications of prior information, and
for all loss functions.  The designer of a prototask also specifies
how many cases should be reserved for use in testing.

To obtain the standard set of task instances that go with a task,
\delve{} first reserves the specified number of cases for use in
testing.  Training sets of the desired sizes are then obtained by
successively dividing the set of remaining cases (whose number will
usually have been arranged to be a multiple of the largest standard
training set size).  In the above example, suppose that the prototask
was applicable to 500 cases in the dataset.  We could reserve 340
cases testing, leaving 160 cases for inclusion in training sets.  For
the task with a training set of size 80, this allows for two task
instances, obtained by partitioning the 160 cases not in the test set
into two subsets.  Similarly, four instances can be created of the
task with a training set of size 40, obtained by dividing each of the
training sets of size 80 in half, and eight instances of the task with
20 training cases, obtained by subdividing the 40-case training sets
yet again.  (It would also be possible to define a single instance of
a task with a training set of size 160, but with only a single
training set, no empirical assessment could be made of the variability
of performance on this task with respect to random choice of training
set.)

In the above example, the generation of test sets for each task
instance would depend on the type of {\tt Test-Set-Selection}
specified for the prototask, as explained in the previous section. If
the {\tt Test-Set-Selection} is {\tt common} then all test cases will
be included in a single common test set, used for every instance. If
the {\tt Test-Set-Selection} is {\tt hierarchical} then the test cases
will also be divided into smaller disjoint subsets, one for each
instance of a particular size.

The successive partitioning described above is performed using the
order of cases as defined in the prototask specification.  For
prototasks without any complications, the test set consists of the
first so-many cases in this ordering, and the training sets are taken
from the later part of the ordering.  The training sets of different
sizes are obtained by successively dividing the full set of potential
training cases into contiguous blocks.  Recall that the ordering of
cases will be random unless the prototask is intended to be
sequential.

The above scheme becomes a bit more complicated if the the prototask
has special features.  For a sequential prototask, a gap of unused
cases will be left between the cases used for testing and those used
for training.  The size of this gap will be the maximum range of
dependencies given in the prototask specification.  For data with
commonality indexes, the ordering will group cases with the same
commonality index together, and a gap will be left if necessary to
ensure that no cases used for testing have the same commonality index
as a case in some training set.  These provisions to eliminate
dependencies between the training and test sets are needed for the
performance on the test set to be a faithful indication of real
performance.  Note, however, that for sequential prototasks and for
prototasks where there are commonality indexes, there may still be
dependencies between the training sets for different instances of a
task (though we try to avoid this with commonality indexes).  This
could reduce the accuracy of the performance estimates, but does not
introduce any bias.

\emph{The provisions in the above paragraph have not been implemented
yet. Special provisions will also be needed to handle tasks whose
training sets are specified to be stratified.}


\subsection{Using non-standard task instances}\label{scheme-non-standard}

\emph{The facilities in this section have not been implemented yet.}

A non-standard task instance (perhaps for a task with a non-standard
size of training set) can be specified by giving an explicit list of
the indexes of the cases making up the training and test sets.  These
indexes are with respect to the ordering of the original dataset, but
must be among those included in the prototask.

For a sequential prototask, the list for the training set must be a
sub-sequence of the prototask ordering, and the test cases must be
further from the training cases than the maximum range of dependencies
specified for the prototask.  It is the user's responsibility to
ensure that the manner in which cases were selected for the training
and test sets is valid in other respects.

%
% $Id: sec7.tex,v 1.32.2.3 1996/11/25 14:34:08 revow Exp $
%
\newpage

\section{ASSESSING A LEARNING METHOD}\label{sec-assess}
\thispagestyle{plain}
\setcounter{figure}{0}
\chead[\fancyplain{}{\thesection.\ ASSESSING A LEARNING METHOD}]
      {\fancyplain{}{\thesection.\ ASSESSING A LEARNING METHOD}}

This section and the following explain the details of how to use
\delve{} to assess learning methods. We start here with guidelines on
documenting your method, and then discuss how you can apply your
method to a set of task instances.

The information relating to a method and its application various tasks
is organized into files and directories in the \texttt{methods} part
of the \delve{} hierarchy.  This organization is illustrated in
Figure~\ref{fig-my_method}, and some of the files involved are listed
in Figure~\ref{fig-method-dir} below.


\begin{figure}[b]

\rule{\textwidth}{0.5pt}

\hspace*{-4pt}\begin{tabular}{ll} \\[-6pt]
{\tt Summary} & A brief description of the method \\
{\tt Source} 
  & A sub-directory with files that document the method, and perhaps \\
  & programs that implement it \\[5pt]
\textit{dataset/prototask/task\hspace{-4pt}} 
  & A sub-directory holding results for one task, with files such as: \\[5pt]
\hspace{13pt}{\tt Test-set-stats} & \hspace{13pt}Statistics 
    from the test data used to standardize losses \\
\hspace{13pt}{\tt Coding-used} & \hspace{13pt}The attribute encoding that
    was used to generate the data files \\[5pt]
\hspace{13pt}\file{normalize}{n} & \hspace{13pt}Normalization constants 
    from training data for instance \emph{n} \\
\hspace{13pt}\file{train}{n} & \hspace{13pt}Training data (inputs and targets)
    for instance \emph{n} \\ 
\hspace{13pt}\file{test}{n} & \hspace{13pt}Test inputs for instance 
    \emph{n} \\ 
\hspace{13pt}\file{targets}{n} & \hspace{13pt}Test targets for instance 
    \emph{n} \\[5pt]
\hspace{13pt}\file{cguess}{n} & \hspace{13pt}Coded guesses for test targets
    for instance \emph{n} \\
\hspace{13pt}\file{guess}{n} & \hspace{13pt}Uncoded guesses for test targets 
    for instance \emph{n} \\
\hspace{13pt}\file{loss.S}{n} & \hspace{13pt}Squared error losses produced
    using the guesses for instance \emph{n} 
\end{tabular}

\caption{Some files and sub-directories that may appear within a \delve{}
         methods directory.}

\label{fig-method-dir}
\end{figure}


\subsection{Documenting the method to be assessed}\label{assess-define}

An essential part of reporting results for a learning method is to
document, as precisely as possible, what the method actually
does. These descriptions should be detailed enough to allow someone to
implement the method from the description and get results similar to
those reported.  The description should include a specification of how
data should be encoded for use by this method, on the basis of the
available prior information.  Without such a specification, it is
unclear how the method would be applied to a new task.  If the method
uses \delve{}'s default encodings, you can just say that.  The
description for a method should also specify such matters as how to
decide when an optimization procedure has converged.  You can get an
idea of the level of detail required in documentation by looking at
the existing documentation in the \texttt{methods} directory.

Precise specification of what a learning method does is easiest if the
method is fully automatic. However, there may be situations when it is
undesirable to formulate fully automatic methods. In these cases,
careful descriptions of the heuristics used, together with examples of
the human choices made on sample tasks may be useful.  Since our
overall goal is to evaluate how well learning methods can be expected
to work on novel tasks, when applied by people who are not necessarily
the designers of the learning method, the proper approach to assessing
a non-automatic method would be for the developers of the method to
get other people to apply the method following their
documentation. This method of evaluation may perhaps be too cumbersome
in practice, but it is useful to keep in mind while documenting a
non-automatic learning method.

In many cases it may be a good idea to supply the source of a computer
implementation as a part of the documentation, since the program
itself may be able to resolve important details of the methods. One
should not consider cryptic computer code to be a substitute for an
intelligible description, however.

It is also useful to include some rough estimates of the computational
costs associated with applying the learning method.  Some learning
procedures can use arbitrary amounts of computation time; in this case
a fully-specified method must indicate how the time is limited in
practice.  Different time allowances will define different (albeit
closely related) methods.

Learning algorithms often have parameters whose values need to be set
using empirical trials. \delve{} includes a suite of developmental
datasets that are intended for used in such trial runs. However, it is
possible that you will discover ways of improving your method as a
result of running it on one of \delve{}'s assessment datasets. This is
unfortunate, since modifying the method based on performance on these
datasets may introduce bias in the evaluations.  If a method was tuned
using the assessment datasets, you should therefore include in your
documentation a short description of what tests were done, on what
datasets, so that people can take account of this tuning when judging
the significance of the results obtained.

Documentation and programs relating to a method should be placed in
the \delve{} hierarchy in the \texttt{Source} sub-directory of the
method's directory.  A brief summary of the method should also be
placed in the \texttt{Summary} file in the method's directory.


\subsection{Creating directories for assessments:~~The \mgendir{}
            command}\label{scheme-mgendir}

For each dataset used to assess a method, a directory with the name of
the dataset will exist in the \delve{} hierarchy, within the directory
for the method.  These directories need not all be in the same actual
directory, but may instead be in located within various of the active
\texttt{delve} directories.  This allows you to assess existing
methods on new tasks without having to write into the directory
holding results from the \delve{} archive.

You can create such directories manually if you wish, but it is
usually easier to create an appropriate directory structure using the
\mgendir{} command.  This command will generate all the directories
associated with a given dataset, prototask, or task. If a task is
specified, only the directory for this task will be generated (along
with the directories needed to contain this task directory, if they do
not already exist). If a prototask is specified, then directories for
all the tasks associated with this prototask will be generated.
Typically there will be many tasks, with different training set sizes,
and perhaps with different prior information.  Similarly, if a dataset
is specified, directories for all prototasks defined for the dataset
will be created.

\texttt{Mgendir} creates these directories in or below the current
directory. If some of the directories already exist, \mgendir{} simply
makes sure that they are up to date.  An example will illustrate the
command:

\begin{Session}
unix> cd delve/methods; mkdir mymethod; cd mymethod
unix> mgendir demo/income/std.32
demo/income
demo/income/std.32
unix> mgendir /demo/income
demo/income/std.64
demo/income/std.128
demo/income/std.256
demo/income/std.512
\end{Session}

In this example we first generated the task named \texttt{std.32} of
the \texttt{demo/income} prototask. The \mgendir{} command created the
appropriate directories for that dataset, prototask and task. We then
asked to have the entire set of tasks for the \texttt{income}
prototask generated. In this case \mgendir{} skips the existing
directories and generates the new ones. Notice that the identity of
the current directory is important.  For example, if your current
directory is at the task level, you should not ask \mgendir{} to
generate directories for a new dataset --- this will cause mixing of
the different levels. Always issue the \mgendir{} command from the
correct level (or higher up, as in the above example).

Note that \mgendir{} just creates directories; it does not create the
data files needed to train and test your learning method.  That is
done by the \mgendata{} command.

The above discussion has focused on the most common usage of
generating directories according to existing specifications in the
corresponding \texttt{data} part of the \delve{} hierarchy. You may
sometimes want to generate tasks with different specifications.  For
example, you might want to use an existing prototask, but with a new
specification for prior information. In this case, you would create a
new prior specification file in your \texttt{data} directory, and
specify this name to \mgendir{} to generate the data.


\subsection{Specifying how attributes are to be encoded}
\label{assess-encodings}

Part of the definition of a learning method is the manner in which
attributes are encoded in a form suitable for the technique used.  For
example, inputs to a neural network must be numeric, so a method
based on neural networks that handles categorical inputs must include
a definition of how a categorical value is represented as one or more
numbers.

Some researchers may be interested in developing better encoding
methods, in which case they will of course employ whatever methods
they think are most promising.  \delve{} has facilities that support
a number of common encoding methods, but it is of course possible
that you will have to implement the encoding you want to use yourself.

For researchers who are not especially interested in encoding methods,
\delve{} supplies \emph{default encodings} for attributes, selected on
the basis of the prior information for the task.  If you have no
reason not to, it is probably best for you to stick with the default
encodings, as that will make it easier to isolate the reasons for any
differences in performance between your method and other methods
that also uses the default encodings.

An encoding specification consists of a name for the encoding, perhaps
followed an additional \texttt{passive}, \texttt{unit}, or
\texttt{center} argument. The possible encodings are as follows:\vspace{-4pt}

\begin{list}{}{\setlength{\labelwidth}{0.7in} \setlength{\labelsep}{0.1in}%
\setlength{\leftmargin}{1.1in}}

\item[{\tt ignore}\hfill] 
Ignore the attribute.

\item[{\tt copy}\hfill] 
Copy the raw attribute value unmodified from the dataset file.

\item[{\tt 0/1}\hfill] 
Encode a binary attribute as `0' or `1', with `0' being the passive
value.  An argument of \texttt{passive={\rm\em value}} is mandatory.

\item[{\tt -1/+1}\hfill] 
Encode a binary attribute using a symmetric encoding of `$-1$' for the
first value and `$+1$' for the second value (as ordered in the dataset
specification).

\item[{\tt 1-of-n}\hfill] 
Encode a categorical attribute as a list of zeros and ones.  If
the attribute has $n$ possible values, and no \texttt{passive} argument
is specified, values will be encoded using $n$ numbers, exactly one 
of which is `1', with the others being `0'.  If an argument of 
\texttt{passive={\rm\em value}} is given, the $n$ possible
values will be encoded as $n\!-\!1$ numbers, with the passive value
being encoded by all the numbers being `0', and the non-passive values
being encoded as before, by setting exactly one of the numbers to `1'.

\item[{\tt therm}\hfill] 
Encode a categorical attribute by a thermometer code, using a list of
$n\!-\!1$ numbers with values of $-x$ or $+x$, where $n$ is the number
of categories for the attribute, and $x$ is a scaling factor
described below.  The lowest value of the attribute (according to the
ordering in the dataset specification) will be encoded by setting all
numbers to $-x$.  For the next higher value, the first number will be
$+x$ and the remaining ones $-x$, and so forth.  The scaling factor
$x$ is determined by the \texttt{scale=\textrm{\textsl{string}}}
option, where \textrm{\textsl{string}} is one of \texttt{none},
\texttt{linear}, or \texttt{sqrt}.  If it is \texttt{none}, then $x=1.0$.
If it is \texttt{linear}, then $x=(n-1)^{-1}$.  If it is
\texttt{sqrt}, then $x=(n-1)^{-1/2}$. The default value is
\texttt{sqrt}.

\item[{\tt nm-sqr}\hfill] 
Encode a numerical attribute by shifting and re-scaling its values so
that the distribution of these values in the training set has mean
zero and variance one.  If a \texttt{centre={\rm\em c}} argument is
specified, the values will be shifted to have \emph{c} as their mean
rather than zero.

\item[{\tt nm-abs}\hfill] 
Encode a numerical attribute by shifting and re-scaling its values 
so that the distribution of these values in the training set has 
median zero and average absolute deviation from the median of 
one.  If a \texttt{centre={\rm\em c}} argument is specified, the 
values will be shifted to have \emph{c} as their median rather than zero.

\item[{\tt 0-up}\hfill] 
Encode a categorical attribute as an integer, from zero and up to the
number of possible values minus one (using the ordering of values in
the dataset specification).

\item[{\tt 1-up}\hfill] 
Encode a categorical attribute as an integer, from one and up to the
number of possible values (using the ordering of values in the dataset
specification).

\item[{\tt rectan}\hfill] 
Encode a numerical value, $x$, as two numbers, $\sin(2\pi x/u)$ 
and $\cos(2\pi x/u)$, where $u$ is the value specified by a mandatory
argument of the form \texttt{unit={\rm$u$}}.

\end{list}\vspace{-4pt}

If you need to use encodings other than these, you will have to
specify a coding as close as possible from the list above, and then
modify the data files \delve{} produces using a program of your own.

When you generate data files using the \mgendata{} command (described
in the next section), \delve{} will by default use encodings from the
above list that are selected on the basis of the prior information
specified for the task (see Section~\ref{task-prior}).  The default
encoding for an attribute is based first of all on the type assigned
to the attribute in the prior specification, in the following
way:\vspace{-4pt}

\begin{list}{}{\setlength{\labelwidth}{0.7in} \setlength{\labelsep}{0.1in}%
\setlength{\leftmargin}{1.1in}}

\item[{\tt binary}\hfill] attributes with a \texttt{passive} value 
are coded as \texttt{0/1}; those without a passive value are coded as
\texttt{-1/+1}.

\item[{\tt nominal}\hfill] attributes are encoded as {\tt 1-of-n},
with a passive option if a passive value is specified in the prior.

\item[{\tt ordinal}\hfill] attributes are encoded using {\tt therm},
with the default scale option \texttt{sqrt}.

\item[{\tt real}\hfill] attributes are encoded using {\tt nm-abs}.

\item[{\tt integer}\hfill] attributes are also encoded using {\tt nm-abs}.

\item[{\tt angular}\hfill] attributes are encoded using the {\tt rectan}
code, with the \texttt{unit} argument as specified in the prior.

\end{list}\vspace{-4pt}

You can override the default encodings by giving the name of a file of
alternate encodings (typically called \texttt{encoding}) to the
\mgendata{} command, using the `{\tt -c}' option.  For the format of
this file, see the documentation for \mgendata{} in
Appendix~\ref{app-commands}.  This is useful if you wish to use other
than the default encodings, and also if the software your using has
built-in facilities that implement the default encodings, but expects
to receive attributes in some other format.

The manner in which choices of encodings are made is logically part of
the learning method and should be documented as part of the
description of the learning method being assessed.  If other than
default encodings are being used, you will probably have to manually
specify how attributes are to be encoded for a particular task,
according to the rules defined for the method.  In theory, however, a
method's encoding rules could be implemented automatically, using a
program that reads the relevant specification files.


\subsection{Creating data files for training:~~The \mgendata{}
            command}\label{assess-mgendata}

Once you have decided on the encodings to be used by a method on some
task (which may be just deciding to use the defaults), you can use the
\mgendata{} command to generate the training and testing data files to
be read by the program implementing the method.  These files must be
placed in the directory for the task within the \texttt{methods} part
of the \delve{} hierarchy, which you will usually have created earlier
using \mgendir{}.  The \mgendata{} command can also generate files for
all the task instances associated with a prototask or dataset, as
described in the detailed documentation for \mgendata{} in
Appendix~\ref{app-commands}.

For each task, \mgendata{} creates files pertaining to all task
instances.  These files all have the number of the instance (from 0 on
up) as a suffix.  Four files will be generated for task instance $n$:\
\ \file{train}{n}, \file{test}{n}, \file{targets}{n} and
\file{normalize}{n}.  The contents of the first three of these files
will depend on the encoding used, which can be left to default, or can
be specified using the `\texttt{-c}' option of \mgendata{}, which
should be followed by the name of the file containing the alternate
encodings.  If you type \texttt{minfo} (with no arguments) in the task
directory for a method after running \mgendata{}, you will see a
listing of all the numbers involved in encoding the attributes for the
present set of data files (as saved in the file \texttt{Coding-used}).
Typing \texttt{dinfo} (with no arguments) will show you what numbers
would be produced by the default encodings.  These commands can also
take explicit task specifications. Figure~\ref{fig:dinfo-encoding}
shows the display of the default encodings for the
\texttt{/demo/age/std.128} task produced by \dinfo{}.

\begin{figure}[t]
\begin{Session}
Task: /demo/age/std.128
Training set size: 128
Inputs: 
 col attr name          type   relevance  def coding  options
   1   1  SEX           binary   nlmh       -1/+1        -
   2   3  SIBLINGS      integer  nlmh       nm-abs       -
   3   4  INCOME        real     nlmh       nm-abs       -
   4   5  COLOUR:pink   nominal  nlmh       1-of-n       -
   5   5  COLOUR:blue                  ...
   6   5  COLOUR:red                   ...
   7   5  COLOUR:green                 ...
   8   5  COLOUR:purple                ...
Targets: 
 col attr name          type   noise-lev  def coding  options
   1   2  AGE           real     nlmh       nm-abs       -
\end{Session}\vspace{-4pt}
\caption{Output of the command: \texttt{dinfo /demo/age/std.128}.}
\label{fig:dinfo-encoding}
\end{figure}

The \texttt{train} files produced by \mgendata{} contain the training
cases, one per line. The encoded values of the input attributes for a
case appear first on the line, in the order they are mentioned in the
prototask specification (and in the output of \dinfo{} or \minfo{}).
The encoded values of the target attributes follow the inputs.  All
the numbers in a training data file are separated by spaces.  Note
that there may well be more numbers than attributes, since some
attribute encodings produce more than one number --- as is the case
with the \texttt{COLOUR} attribute in Figure~\ref{fig:dinfo-encoding}.

The \texttt{test} files contain only the input attributes of the test
cases.  The true targets for the test cases are not supplied, since
they should not normally be available to the learning method.  An
exception is allowed for a method that makes predictions to be
evaluated using the log probability loss functions (see
Section~\ref{assess-mloss}), since it is not practical for \delve{} to
evaluate these losses itself.  The true targets are available for this
use in the \texttt{targets} files.  

The \texttt{normalize} files contain the offset and scaling constants
that may have been used to encode the data (if \texttt{nm-abs} or
\texttt{nm-sqr} encodings were specified, or were the defaults).  You
will not normally have to look at the \texttt{normalize} files
yourself, but they are needed for \delve{} to interpret the
predictions produced by the method.

Once the training and testing data files for the various task
instances have been generated using \mgendata{}, you can run your
learning method.  This should be done completely independently for
each task instance, with the run for one instance making no reference
to any data files intended for another instance.  If your learning
method has a stochastic aspect, you should initialize the random seed
differently for each instance, for reasons discussed in
Section~\ref{sec-analysis}.


\subsection{Processing predictions on test cases:~~The \mloss{} 
            command}\label{assess-mloss}

The objective of running your learning method is to produce
predictions for the test cases.  These predictions will normally be
encoded, in the same way as the targets seen by the learning method
were encoded.  As discussed in Section~\ref{sec-loss}, predictions can
take two forms:\ \ point predictions or \emph{guesses} for the target
values, and \emph{predictive distributions} for the targets.  In most
circumstances, your method will not read the \texttt{targets} files
when producing predictions, and the losses with these predictions will
be calculated by \delve{}, not by the method itself.  However, since
there is no easy way of representing an arbitrary predictive
distribution for a target of real, integer, or angular type, the
predictive probability density for the true target must be evaluated
by the method itself, if log probability loss is of interest, with
reference to the true target values found in the \texttt{targets}
files.

The actual losses are in all cases evaluated by the \mloss{} command,
which will refer to files of predictions produced by the method.  In
general, a method may wish to make different predictions for use with
different loss functions.  Accordingly, the files to which a method
writes predictions may have names incorporating the abbreviation of
the loss function for which they are intended.  Prediction files have
one of three possibile root names, according to the form of
prediction:\ \ \texttt{guess}, for point predictions, \texttt{prob},
for predictive distributions, and \texttt{ptarg}, for probabilities
(or probability densities) of the true target value.  Prediction files
always have the instance number as a final suffix
(e.g.~\texttt{guess.3}).  If a specific loss function is specified, it
goes between the root and the instance number
(e.g.~\texttt{guess.S.0}).  The name of a \texttt{prob} or
\texttt{ptarg} file can be prefixed by ``l'' to indicate that it
contains the (natural) logs of the probabilities (or densities),
rather than the probabilities themselves.  Finally, names of
prediction files may optionally have a leading `\texttt{c}' to
indicating that they are for encoded data. For a more extensive
discussion of these conventions, refer to the discussion of \mloss{}
in Appendix \ref{app-commands}.

The \mloss{} command performs two tasks: it decodes predictions (in
the typical situation where the method's predictions were encoded),
and it evaluates losses.  When \mloss{} is invoked it looks to see if
it can find encoded prediction files. If so, it decodes the encoded
predictions in the files and writes these to files with the initial
`\texttt{c}' removed from their name. For example, \texttt{cguess.A.0}
would be decoded into \texttt{guess.A.0}.  After this, \mloss{} looks
for the decoded prediction files (which it may just have produced
itself), computes the losses using them, and writes them to files
called \file{loss}{l.n}, where \textit{l} is the loss function, and
\textit{n} is the instance number.  Section~\ref{sec-analysis}
discusses how these loss files are analysed.

After running \mloss{} you can remove the \texttt{train.*},
\texttt{test.*}, \texttt{targets.*}, and \texttt{normalize.*} files
(they can be regenerated using \mgendata{} if you should ever want
them again).  You should keep the \texttt{Coding-used} and
\texttt{Test-set-stats} files, as they are used by \mstats{} and
\minfo{}.  Usual practice is to also remove any encoded prediction
files, keeping only the decoded versions (\texttt{guess.*},
\texttt{prob.*}, etc.).  You can remove the \texttt{loss.*} files as
well, if you need to save disk space, as they can be regenerated from
the decoded prediction files using \mloss{}, but it is better if
possible to keep the loss files around so that performance comparisons
between methods can be made conveniently.


\subsection{Submitting your results to the \delve\ archive}
            \label{assess-prepare}

Once you have documented a learning method, and tried it out on a
number of tasks, you may submit the method and the results of applying
it for inclusion in the \delve{} archive.  Other people will then be
able to examine your method and results, and compare the results they
obtain with their methods to those that you obtained.

You submit a method to the archive by sending the complete directory
structure for the method, containing the documentation and tests on
all the datasets you have tried.  This directory will be placed in the
\texttt{methods} directory of the \delve{} archive.  It is also possible
to submit new results on existing methods, and new datasets and
prototask specifications.  For details on how to go about submitting
material to the \delve{} archive, see Appendix~\ref{app-submit}.

It should be understood that submission of a learning method to the
\delve{} archive constitutes a form of publication.  Once your method
has been incorporated into the archive, other researchers will start
publishing comparisons of their results with yours.  For these
comparisons to be intelligible to other researchers, it is necessary
for methods to remain in the archive once they have been submitted, in
their original form, though you will be able to submit new commentary
on the method, explaining any new developments.  When a bug is found
in the program implementing the method, or a substantial improvement
has been made to the learning method, a new updated version may be
submitted.

%
% $Id: sec8.tex,v 1.24.2.4 1996/11/25 14:34:09 revow Exp $
%
\newcommand{\MSa}{{\rm MS}_a}
\newcommand{\MSb}{{\rm MS}_b}
\newcommand{\MSe}{{\rm MS}_\varepsilon}
\renewcommand{\SS}{{\rm SS}}
\newpage

\section{ANALYSING THE RESULTS}\label{sec-analysis}
\thispagestyle{plain}
\setcounter{figure}{0}
\chead[\fancyplain{}{\thesection.\ ANALYSING THE RESULTS}]
      {\fancyplain{}{\thesection.\ ANALYSING THE RESULTS}}

Suppose we have applied several learning methods to one or more tasks,
and used the \mloss{} command to evaluate the losses for the
predictions they produced, as described in the previous section.  We
can now use the \mstats{} command to compute summaries of losses, and
perform tests of statistical significance on observed differences.  We
hope that in this way we will be able to draw interesting conclusions
about the relative performance of these learning methods on these
tasks.

The \mstats{} command addresses two basic questions.  First, how can
we compute an estimate of the performance of a method on some task,
together with an indication of uncertainly in the estimate?  Second,
how can we judge whether an observed difference in performance between
two methods is statistically significant?  This section will explain
the theory of the statistical procedures used to answer these
questions, and the commands that implement these procedures.

A task is the basic unit for which an expected loss can be defined.
However we cannot apply our learning methods directly to tasks, since
no specific training cases are associated with a task. Instead we
apply our methods to a set of task instances and use the observed
performance these particular instances to estimate the expected
performance on the task. Note, in particular, that we are generally
not interested in the performance on a certain set of test cases, nor
in the performance when using particular training sets. Rather, we
wish to estimate the expected performance on a new test case when the
learning method has been trained on a new training set, both of which
are randomly drawn from the same distributions as are the available
task instances.

In order to form estimates that are appropriate for this context, we
use a set of statistical techniques known as ANOVA (for ANalysis Of
VAriance).  In each experiment, we try several training sets and for
each training set we evaluate the losses for many test cases. The
appropriate analysis depends on whether a single common test set or a
hierarchical design with disjoint test sets was used.  The analysis
for the hierarchical model is simplest, so we begin with that.

\subsection{Analysing the hierarchical loss model}

In the hierarchical model, the losses for a particular set of task
instances and a particular learning method are modeled by:
\beq
y_{ij}=\mu+a_i+\varepsilon_{ij},
\label{eq-loss-model-h}
\eeq
where $y_{ij}$ is the loss on training set $i$ and test case
$j$ (from the $i$'th test set --- remember, in the hierarchical model
there is a separate test set for each training set). There are $I$
instances, each of which contain $J$ test cases. The overall mean loss
is given by $\mu$. The parameter $a_i$ is a random variable which
explains the variation in losses due to individual training sets. The
$\varepsilon_{ij}$ parameters account for the residual variation in
the losses which are unexplained by the model.

The loss model in eq.~(\ref{eq-loss-model-h}) captures the notion that
individual training sets may not be equally well suited to learn the
true relationship of the data. As an example, it may be that one
particular training set contains an outlier, which can be accounted
for by the corresponding $a_i$ taking on a large positive value. The
residuals $\varepsilon_{ij}$ account for the variability in losses
that are unexplained by the contributions from training set factor
$a$.  This variability may be due to variation in the ``difficulty''
of test cases (either in general, or when a particular training set is
used).  Any stochastic aspects of the learning method can also 
contribute to the variability in either the $a_i$ or the $\varepsilon_{ij}$,
as discussed below.

We propose using simple independent Gaussian assumptions about the
model parameters:
\beq
a_i\sim {\cal N}(0, \sigma_a^2) \hspace{4cm}
\varepsilon_{ij}\sim {\cal N}(0, \sigma_\varepsilon^2).
\label{eq-model-h}
\eeq
These assumptions are primarily based on simplicity requirements for
the following analysis of the results. For many loss functions the
distributions of the above variables may not be well approximated by
Gaussians. However, it is generally believed that the $t$-test which
will be used in the following are fairly robust to violations of
normality.  Finally, it seems that more sophisticated models become
very complicated to analyse, which is why we have settled for this
simple model as our standard recommendation in \delve{}. Note that the
loss files are available, so that a more ambitious analysis can be
performed if desired.

The parameter in which we are primarily interested is $\mu$, the
overall mean performance of the method.  An estimate for it, $\hat\mu$,
can be found as follows:
%
\beq
\hat\mu\ =\  {1 \over IJ} \sum_{i,\,j} y_{ij} \ =\  \bar y
 \hspace*{2cm}{\rm SD}(\hat\mu)\ =\ 
\Big(\frac{\sigma_a^2}{I}+\frac{\sigma_\varepsilon^2}{IJ}\Big)^{1/2}. 
\label{res:hier}
\eeq
%
This above standard error is in terms of the true values of the
$\sigma$ parameters.  In practice, we will have to substitute estimates
for the $\sigma$ parameters. 

We introduce the following partial means:
%
\beq
\bar y_i\ =\ \frac{1}{J}\sum_j y_{ij}\hspace*{2cm}
\eeq
%
and the ``mean squared error'' for $a$ and $\varepsilon$ and their
expectations
%
\beq
\MSa\ =\ \frac{J}{I-1}\sum_i(\bar y_i-\bar y)^2 & \hspace{2cm}
& E[\MSa]\ =\ J\sigma_a^2+\sigma_\varepsilon^2\\
\MSe\ =\ \frac{1}{I(J-1)}\sum_i\sum_j(y_{ij}-\bar y_{i})^2 &
& E[\MSe]\ =\ \sigma_\varepsilon^2.
\eeq
%
We can now use the following minimum variance unbiased estimators for
the $\sigma$ values
%
\beq
\hat\sigma_\varepsilon^2\ =\ \MSe
 \hspace*{2cm} \hat\sigma_a^2\ =\ \frac{\MSa-\MSe}{J}.
\eeq
%
Unfortunately, $\hat\sigma_a^2$ may be negative, in which case we set
it to zero. The estimated values may be substituted back into
eq.~(\ref{res:hier}) to estimate the uncertainty associated with the
average loss.

In order to compare two learning methods, the same model can be
applied to the \emph{differences} between the losses from two learning
methods, identified by $k$ and $k'$
%
\beq
y_{ij}\ =\ y_{ijk}-y_{ijk'}\ =\ \mu+a_i+\varepsilon_{ij},
\eeq
%
with similar Gaussian and independence assumptions as in
eq.~(\ref{eq-model-h}).  In this case $\mu$ is the expected difference
in performance and $a_i$ will be the difference effect due to training
sets. Since the overall estimate for the mean can be seen as arising
from $I$ independent estimates from each of the instances, we can test
whether the estimate for $\mu$ is should be considered significantly
different from zero using a $t$-test. This is effectively a paired $t$-test
for whether the expected performance of the two methods is different,
with the pairing being performed by modeling the differences in losses. 
The appropriate $t$ statistic to use is
%
\beq
t\ =\ \bar y\Big(\frac{1}{I(I-1)}\sum_i(\bar y_i-\bar y)^2\Big)^{-1/2}
\ =\ \bar y\Big(\frac{\MSa}{J(I-1)}\Big)^{-1/2},
\eeq
with $I-1$ degrees of freedom.

In cases where the methods to be analysed have stochastic elements,
these give rise to variation in the losses that are not explicitly
accounted for in the above analysis. There may be stochastic elements
in both the training of the method and in the predictions. For
example, many neural network methods are initialized with random
weights which gives rise to stochasticity in the training phase.

Although these stochastic elements are not explicitly modeled, the
additional variability that they lead to will still show up in this
model. Some training conventions should be followed so that it always
shows up in the same way.  If your learning method is stochastic, you
should use a different random number seed for every training set. This
will result in the variation due to stochastic training being lumped
together with the training set effects in the analysis. Similarly, if
your prediction procedure is stochastic, you should use random numbers
that are independent for each combination of training set and test
case, so that the effects of stochastic predictions will be lumped
together with the effects of test cases.  Following these conventions,
the present analysis will take these stochastic effects into account
in a consistent way, but you will not be able to separate the
stochastic training and prediction effects from the other sources of
variability. In future versions of \delve{} we may support explicit
evaluation of stochastic training effects, since it may often be of
interest to know how much performance varies with things such as
random initialization of model parameters.  However, this extra
information will come at the cost of having to run methods multiple
times on identical training and test sets, but with different random
number seeds.

\subsection{Analysis of experiments with common test sets}

When using a common test set the nested model described in the
previous section is no longer applicable. Instead, we model the losses
for a particular set of task instances and a particular learning
method by:
\beq
y_{ij}\ =\ \mu+a_i+b_j+\varepsilon_{ij},
\label{eq-loss-model-c}
\eeq
where $y_{ij}$ is the loss on training set $i$ and test case $j$.
The overall mean loss is given by $\mu$. The parameters $a_i$ and
$b_j$ are random variables that explain the variation in losses due to
individual training sets and test cases respectively. The
$\varepsilon_{ij}$ parameters are the residual variation in losses,
which are unexplained by the model.

The loss model in eq.~(\ref{eq-loss-model-c}) captures both effects of
training sets $a_i$ and test cases $b_j$. In the case of a common test
set, we have computed the loss for each test case using each of the
$I$ training sets, and can thus explicitly estimate the effects of the
different test cases in general (as opposed to their effect in
combination with a particular training set). As before, the residuals
$\varepsilon_{ij}$ account for the variability of the losses
unaccounted for by the model, such as interactions between training
sets and test cases.  If the methods being tested have stochastic
elements, variation due to this will also show up somewhere, as discussed 
below.

We again propose using simple independent Gaussian assumptions about the
model parameters:
\beq
a_i\sim {\cal N}(0, \sigma_a^2) \hspace{2cm} b_j\sim {\cal N}(0, \sigma_b^2)
\hspace{2cm} \varepsilon_{ij}\sim {\cal N}(0, \sigma_\varepsilon^2).
\label{eq-model}
\eeq
As before, these assumptions are primarily based on simplicity
requirements for the following analysis of the results.  For many loss
functions the distributions of the above variables may not be well
approximated by Gaussians. However, it is generally believed that the
F-test which will be used in the following is fairly robust to
violations of normality.

We wish to find an estimate, $\hat \mu$, for the expected loss of the 
learning method, as well as a standard error associated with
this estimate.  As our estimate for the expected loss, we can
use the average loss. We can estimate the standard deviation for this
estimate based on the model defined by eq.~(\ref{eq-loss-model-c}) 
and (\ref{eq-model}):
\beq
\hat \mu \ =\  \bar y \hspace{2cm}
\mbox{SD}(\hat \mu) \ =\  \left(\frac{\sigma^2_\varepsilon}{IJ}+
\frac{\sigma_b^2}{J}+\frac{\hat\sigma_a^2}{I}\right)^{1/2},
\label{eq-mean}
\eeq
where $I$ is the number of training sets and $J$ the number of test
cases. The expected mean is simply the overall average loss. To
evaluate the standard error we first need to estimate the values of
the $\sigma$ parameters. Here and in the following we will use the property
that when training sets are set up using the \delve{} standard scheme
(see section \ref{sec-scheme}), the training sets are disjoint subsets
of the entire data set. We introduce the overall mean and the marginal
means:
\beq
\bar y\ =\ \frac{1}{IJ}\sum_i\sum_j y_{ij} \hspace{2cm}
\bar y_i\ =\ \frac{1}{J} \sum_j y_{ij} \hspace{2cm}
\bar y_j\ =\ \frac{1}{I} \sum_i y_{ij},
\eeq
and the the ``mean squared error'' for $a$, $b$ and $\varepsilon$ and
their expectations:
\beq
\MSa\ =\ \frac{J}{I-1}\sum_i(\bar y_i-\bar y)^2 \hspace{4cm}
&E[\MSa]\ =\ J\sigma_a^2+\sigma_\varepsilon^2\\
\MSb\ =\ \frac{I}{J-1}\sum_j(\bar y_j-\bar y)^2 \hspace{4cm}
&E[\MSb]\ =\ I\sigma_b^2+\sigma_\varepsilon^2\\
\MSe\ =\ \frac{1}{(I-1)(J-1)}
\sum_i\sum_j\big((y_{ij}-\bar y)-(\bar y_i-\bar y)-(\bar y_j-\bar y)\big)^2 
&E[\MSe]\ =\ \sigma_\varepsilon^2
\eeq
Now we can use the empirical values of $\MSa$, $\MSb$ and $\MSe$ to
estimate values for the $\sigma$'s:
\beq
\hat \sigma_\varepsilon^2\ =\ \MSe
\ \ \ \ \ \hat \sigma_b^2\ =\ \frac{\MSb-\MSe}{I}
\ \ \ \ \ \hat \sigma_a^2\ =\ \frac{\MSa-\MSe}{J}
\eeq
These estimators are uniform minimum variance unbiased estimators.
Unfortunately however, the estimates for $\sigma_a^2$ and $\sigma_b^2$
are not guaranteed to be positive, so we set them to zero if they are
negative.  We can then substitute back these variance estimates in
eq.~\ref{eq-mean} to get an estimate for the standard error for the
estimated mean performance.

Note that the estimated standard error $\hat\sigma$ diverges if we
only have a single training set (as is common practise!).  This effect
is caused by the hopeless task of trying to empirically estimate a
variance based on a single observation. At least two training sets
must be used, and probably more if accurate estimates of uncertainty
are to be achieved.

Another important question is whether we have good evidence that
one learning methods is better than another. To settle this
question we again use the model from eq.~(\ref{eq-model}), only this
time we model the difference between the losses of the two models, $k$
and $k'$:
\beq
y_{ijk}-y_{ijk'}\ =\ \mu+a_i+b_j+\varepsilon_{ij},
\eeq
under the same assumptions as above. The question now is whether the
estimated overall mean difference $\hat\mu$ is significantly different
from zero. We can test this hypothesis using a quasi-F test [Lindman,
Harold R., ``Analysis of Variance in Experimental Design'',
Springer-Verlag, 1992], which uses the F statistic and degrees of
freedom:
\beq
F_{\nu_1,\nu_2}&\ =\ &(\SS_m+\MSe)/(\MSa+\MSb),
\mbox{\ \ \ where\ \ \ }\SS_m \ =\  IJ\bar y^2\\
\nu_1&\ =\ &(\SS_m+\MSe)^2/(\SS_m^2+\MSe^2/((I-1)(J-1)))\\
\nu_2&\ =\ &(\MSa+\MSb)^2/(\MSa^2/(I-1)+\MSb^2/(J-1)).
\eeq
The result of the F-test is a p-value, which is the probability that
given the null-hypothesis ($\mu=0$) is true, we would
observed a difference in average performance of this magnitude
(positive or negative), or of a more extreme magnitude. In general, a 
low p-value produces high confidence that
the learning method with better performance in this experiment 
actually has better performance.  If the p-value is not low (say,
greater than 0.05), it is not implausible that the method whose
performance appeared better in this experiment could actually
be worse in reality.

As is the case with the hierarchical model, the common test set model
will pick up the variability due to stochastic training and stochastic
predictions, even though they are not modeled explicitly.  Whenever
you apply a stochastic method you should initialize it with a
different random seed. The uncertainty due to stochastic training will
then be lumped together with the training set effects in the model,
and the effects of stochastic predictions will by lumped together with
the interaction effects.  Thus, the present analysis will take these
stochastic effects into account, but you will not be able to separate
the effects according to their causes. In future versions of \delve{}
we may support explicit evaluation of the stochastic training set
effect, if the method has been run several times on the same training set
with different random seeds.


\subsection{Obtaining performance statistics:~~The \mstats{} command}
\label{analysis-mstats}

The \mstats{} command implements the calculations described in the two
previous sections. When used to estimate the expected loss for a
particular task, \mstats{} is called from within the task directory,
or is given the path to such a directory in the \delve{} hierarchy.
The command will look for \file{loss}{l.x} files in this directory,
and produce the statistics derived from these files. You may specify
the desired loss functions with the `\texttt{-l}' option. As an
example, we can analyse the performance of a linear regression method
in the \texttt{demo/age/std.128} task, using absolute error loss:

\begin{Session}
unix> mstats -l A /lin-1/demo/age/std.128
/lin-1/demo/age/std.128
Loss: A (Absolute error)
                                                    Raw value   Standardized

                         Estimated expected loss:    15.0988      0.893246
                     Standard error for estimate:   0.667719     0.0395023

     SD from training sets & stochastic training:    1.49368     0.0883662
SD from test cases & stoch. pred. & interactions:    13.0755      0.773547

    Based on 8 disjoint training sets, each containing 128 cases and
             8 disjoint test sets, each containing 128 cases.
\end{Session}

Here, the values reported correspond to the parameters of the model in
eq.~(\ref{eq-loss-model-h}); the overall mean performance is followed by
the standard error on this estimate. Also the standard deviations for
different sources of variability are printed.

In the second column, the values have been standardized, an a manner
appropriate for the loss function. The standardized domain is designed
such that a simple \emph{baseline} method has a nearly pre-specified
performance. This makes the standardized losses easier to interpret
than the raw losses. Note however, that these standardizations are
obtained by imagining the baseline methods applied to the union of all
\emph{test cases}, such that applying the same simple methods to actual
training instances will typically yield a standardized error a little
larger than might be expected. (We are forced to accept this, since
the standardizations must be the same for all instances in a task,
whereas the training sets usually differ.)

For the `S', `A', `Z' and `Q' loss types, we obtain standardized
losses by division by the baseline loss. For squared error loss, the
baseline method is prediction of the mean. For absolute error loss,
the baseline is prediction of the median. For 0/1-loss, the baseline
method is to always predict the majority class, yielding a loss of
$1-p^*$, where $p^*$ is the frequency of the majority class.  The
baseline method for squared probability loss predicts the empirical
class probabilities as observed in the test cases, giving a loss of
$1-\sum_i p_i^2$, where $p_i$ is the frequency of the $i$'th class
over the test cases. Thus, these simple methods (which don't utilise
the inputs) will have a standardized losses of close to $1.0$ and
better methods will have losses closer to $0.0$.

For log probability loss, the baseline method depends on whether the
targets are discrete or continuous. For continuous targets the
baseline method produces a Gaussian predictive distribution with mean
and variance set to the empirical mean and variance of the test
cases. Thus, the standardized losses are obtained by subtracting
$\frac{1}{2}\log(2\pi\sigma^2)+\frac{1}{2}$ from the raw values.  For
discrete targets, the baseline method sets the class probabilities in
accordance with the test frequencies; the standardized values are
consequently obtained by subtracting $-\sum_i p_i\log(p_i)$ from the
raw values, where $p_i$ is the frequencies of class $i$ in the test
set. Thus, methods which perform as well as the baseline methods will
have a standardized loss of around $0$ and better methods will have
negative losses. The negative value of the standardized loss can be
interpreted as the amount of information (measured in nats) that the
method predicts about the targets relative to the baseline method.

Specifying the `\texttt{-c}' option to \mstats{} causes it to compare
losses with the method named after the `\texttt{-c}' option.  For
example we can compare the linear method with a $k$-nearest-neighbor
method with respect to absolute-error loss on the
\texttt{/demo/age/std.128} task as follows:

\begin{Session}
unix> mstats -l A -c knn-cv-1 /lin-1/demo/age/std.128
/lin-1/demo/age/std.128
Loss: A (Absolute error)
                                                    Raw value   Standardized

               Estimated expected loss for lin-1:    15.0988      0.893246
           Estimated expected loss for /knn-cv-1:    13.2854      0.785965
                   Estimated expected difference:     1.8134      0.107281
          Standard error for difference estimate:   0.350707     0.0207478

     SD from training sets & stochastic training:   0.505922     0.0299304
SD from test cases & stoch. pred. & interactions:    9.65323      0.571086

    Significance of difference (t-test), p = 0.00129409

    Based on 8 disjoint training sets, each containing 128 cases and
             8 disjoint test sets, each containing 128 cases.
\end{Session}



\appendix

%
% $Id: appA.tex,v 1.13.2.6 1996/12/04 19:04:28 revow Exp $
%
\newpage

\section{INSTALLING DELVE ON YOUR COMPUTER}\label{app-install}
\thispagestyle{plain}
\setcounter{figure}{0}
\chead[\fancyplain{}{\thesection.\ INSTALLING DELVE ON YOUR COMPUTER}]
      {\fancyplain{}{\thesection.\ INSTALLING DELVE ON YOUR COMPUTER}}

\delve{} consists of a set of utility programs for assessing learning
methods, a number of datasets that can be used for such assessments,
and the results of assessing various learning methods on these
datasets.  To use \delve{} you must at least install the utility
programs.  You will no doubt wish to install some of the datasets as
well (unless you wish to use \delve{} only on your own data).  If you
want to compare your learning methods with others, you will also need
to install the relevant results.

\subsection*{Requirements}

Currently, \delve{} requires you to be running some variant of
Unix. It has been tested under IRIX 5.3 and Sun-OS 5.4, but should run
under other variants without problems.

The datasets and method results have no requirements beyond a Unix
file system.  The utilities currently require that you have an
ANSI-compliant C compiler and an installed copy of \textbf{Tcl} (Tool
Command Language).  \textbf{Tcl} is freely available on the Internet
and are extremely portable (i.e.~it has almost certainly been ported
to whatever variant of Unix you are running).  If you do not already
have \textbf{Tcl} installed, copies of the source are available at the
ftp site \texttt{ftp.smli.com} in the directory \texttt{/pub/tcl}.

\subsection*{Obtaining \delve}

The best way to obtain \delve{} is to visit our web site: \wwwhome{}.
You'll find full instructions on getting and building delve there, as
well as the latest news on the software, results, and datasets.

If you don't have access to a web browser, the \delve{} distribution
is available via anonymous ftp, in multiple compressed \texttt{tar}
files (Unix tape archive format).  Currently the files are available
on the machine \texttt{ftp.cs.toronto.edu} in the directory
\texttt{/pub/neuron/delve}.

The files are broken down as follows:
\begin{enumerate}
\item 
The source code for the \delve{} utilities and documentation is
available in one file: \texttt{delve-*.tar.gz}.
\item 
Each dataset is in its own file, where the name of the file is the
same as the dataset (with the appropriate suffix added),
e.g. \texttt{demo.tar.gz}.  The easiest way to obtain datasets is from the
Delve web site at \texttt{http://www.cs.toronto.edu/\maketilde{}delve} or they
can be obtained by ftp from  \texttt{ftp.cs.toronto.edu} in the directory
\texttt{/pub/neuron/delve/data/tarfiles} 

\item
The complete results for each method that has been run on \delve{} is
in its own file, named in a manner similar to the datasets, but with
an  \texttt{all} appended, e.g. \texttt{lin-1-all.tar.gz}.
Results for a particular method and dataset are stored in files with
the \texttt{-all} suffix replaced with the dataset name:
\texttt{lin-1-demo.tar.gz}.  The source code and description for the
methods are stored in another tar file with \texttt{-all} replaced
with \texttt{-Source}: \texttt{lin-1-Source.tar.gz}.  These files are only
available from the Delve web site at 
\texttt{http://www.cs.toronto.edu/\maketilde{}delve}.
\end{enumerate}

A sample \texttt{ftp} session for obtaining \delve{} might be as
follows\footnote{This example illustrates the process for version 1.1; for
other versions replace ``1.1'' by the version/patch number you wish.}:

\begin{Session}
ftp ftp.cs.toronto.edu
cd /pub/neuron/delve
binary
get software/delve-1.1.tar.gz
get data/tarfiles/demo.tar.gz
quit
\end{Session}

\subsection*{Installation}

Before installing the datasets and method results, you must build and
install the \delve{} utilities as follows:

\begin{enumerate}
\item
Obtain the distribution file from our ftp site:
\begin{Session}
ftp ftp.cs.toronto.edu
get /pub/neuron/delve/software/delve-1.1.tar.gz
bye
\end{Session}
%
\item
Uncompress and untar the distribution using the \texttt{gunzip}
utility:
\begin{Session}
gunzip delve-1.1.tar.gz
tar xvf delve-1.1.tar
\end{Session}
%
\item
Run the configuration script:
\begin{Session}
cd delve-1.1
./configure
\end{Session}
or, for systems that don't recognize \texttt{\#!} in shell scripts:
\begin{Session}
cd delve-1.1
/bin/sh ./configure
\end{Session}
By default, the configuration script will set things up to be
installed in \texttt{/usr/local}. You can change this by specifying a
different \texttt{prefix} in the \texttt{configure} command:
\begin{Session}
./configure --prefix=/your/install/path
\end{Session}
You can also add options for a particular \texttt{cc} compiler and
compiler flags:
\begin{Session}
./configure --with-cc=gcc --with-cflags=-g
\end{Session}
For a full list of the options \texttt{configure} takes, type:
\begin{Session}
./configure --help
\end{Session}
The \texttt{configure} script generates new Makefiles from their
respective templates (Makefile.in). If \texttt{configure} can't find
something, you can make changes to the intermediate
\texttt{config.status} script, and invoke this script to reconfigure
the Makefiles:
\begin{Session}
vi config.status
./config.status
\end{Session}
As a last resort, you can edit the Makefiles in the current directory
and \texttt{doc/} by hand and insert the proper paths.
%
\item
Build the libraries and the executables. From the top-level directory
type:
\begin{Session}
make all
\end{Session}
%
\item
Install the executables, libraries, documentation, and script
files. From the top-level directory type:
\begin{Session}
make install
\end{Session}
If you have problems with the installation, you can use a subset of
the commands:
\begin{Session}
make install-binaries
make install-libraries
make install-doc
make install-man
\end{Session}
\end{enumerate}

Once you've installed the utilities, you can install the
datasets. This involves simply extracting the files from their tape
archives into the proper directory: the installed top-level \delve{}
data directory. By default this directory is
\texttt{/usr/local/lib/delve/data}. If you specified a
\texttt{--prefix} to the \texttt{configure} command, replace the
\texttt{/usr/local} prefix with the path of that directory.

Each tape archive will create a directory with the same base name as
the archive file. This directory will contain all the data and
specification files \delve{} needs to generate the tasks.
\begin{Session}
mv demo.tar.gz /usr/local/lib/delve/data
cd /usr/local/lib/delve/data
zcat demo.tar.gz | tar xvf -
\end{Session}

If you want to install a dataset in a private directory, you can do
the following
\begin{enumerate}
\item
Create a directory called \texttt{delve} in your home directory (or
anywhere else, for that matter).
\item
In that directory create two more directories: \texttt{data} and
\texttt{methods}.
\item
In the \texttt{delve/data} directory, untar the data file as described
above.
\end{enumerate}
Once you've done that, you can work in your own private delve
directory and you will have access to the datasets you've downloaded,
as well the ones installed in \texttt{/usr/local/lib/delve/data}.

Once you've extracted the data, you can safely remove the tar file.

\subsection*{Setup}

Once the software has been installed you can run any of the \delve{} commands
without further setup. There are, however, 2 environment variables that make the
software more flexible

\begin{enumerate}
\item
DELVE\_PATH - (see also appendix C) allows multiple delve directories to be
active. It is similar in flavour to the normal Unix PATH environment variable.

\item
DELVE\_UNCOMPRESS - Set this environment variable to the name of the Unix
utility that will {\em uncompress} files on the ``fly'', ie it can read
compressed files and uncompress them to stdout. If this environment variable
is not set, {\em zcat} is assumed.
\end{enumerate}

%
% $Id: appB.tex,v 1.4.2.2 1996/06/14 02:37:51 carl Exp $
%
\newpage

\section{CONTRIBUTING TO THE DELVE ARCHIVE}\label{app-submit}
\thispagestyle{plain}
\setcounter{figure}{0}
\chead[\fancyplain{}{\thesection.\ CONTRIBUTING TO THE DELVE ARCHIVE}]
      {\fancyplain{}{\thesection.\ CONTRIBUTING TO THE DELVE ARCHIVE}}

The ultimate aim of the \delve{} project is to collect datasets,
implementations of learning methods, and the results of learning
experiments from a wide variety of sources. If you have datasets,
methods or results which might be of interest to other users you can
submit these to the \delve{} archive. To make contributions, you can
put files on our ftp-server \texttt{ftp.cs.utoronto.ca} in the directory
\texttt{/pub/incoming}, and notify us by email to 
\texttt{delve@cs.utoronto.ca}. The submitted files should preferably
conform to the usual \delve{} conventions, and be in the form of a
compressed \texttt{tar} file.

We welcome contributions of datasets to \delve{}. We are particularly
seeking large real-world datasets, and realistic simulation programs
that can be used to create large datasets. Contributions of datasets
should be accompanied by descriptions of the data. For real datasets
both the data in its original form and in \delve{} format should be
supplied, as well as descriptions of the relevant context and the
attributes recorded. Also suggestions for prototasks together with
specifications of prior information should be included. Naturally,
proprietary data cannot be included in \delve{} without permissions.
For simulated and artificial datasets, programs to generate the data
should be supplied (if possible) as well as descriptions of the data
attributes and suggestions for prototasks and priors, etc.

You may also contribute new learning methods to the archive.
Typically, you would also provide results of running your method on
various \delve{} datasets. You can conveniently submit the whole
methods directory pertaining to your method. Also you need to supply a
detailed description of your method. Remember, that the description
should be detailed enough that someone else can re-implement the
method and get comparable results to the ones you might get for any
dataset to which the method is applicable.  The easiest way to attain
this, is if your method is fully automatic. In particular, you should
make sure that your description includes:

\begin{itemize}
\item implemetational details allowing someone else to re-implemet your
method with similar results
\item discussion of the role of all parameters of the method
\item discussion of the heuristic rules for setting all parameters of the
method on the basis of a particular application, including convergence
criteria for iterative methods
\item detailed discussion of how attributes should be encoded for the
method
\end{itemize} 

Finally, it would be convenient if source code of the program
implementing you method can be included in \delve{}. This may help
clarify details of the implementation, help other researches to
easily use the methods and help with identifying possible
bugs. Authors should take care not to submit implementations
containing any parts whose copyrights prohibit public distribution.

For all contributions it should be considered that submission to
\delve{} is a form of publication, and once contributions are released
with \delve{} they cannot in general be retracted, since other people
may have used them in their research. Therefore, care should be taken
to avoid submissions of erroneous material. If a bug should be
discovered in a learning method a new corrected version can be
submitted under a different name; eg.~a buggy version of {\tt loess-1}
could be succeeded by a corrected version named {\tt loess-2} --- but
the original method and its results would be retained in the archive.

You may also submit experimental results using new combinations of
methods and datasets that are already in \delve{}. If you repeat
experiments for which results are already in the archive, it is of
interest whether your results were comparable to the earlier results.
Notes of such confirmations can be included in \delve{}, but for
practical reasons only one set of results can be maintained for each
method.

All submitted material will be presented in \delve{} with the date,
name and address (or email) of the contributor(s) allowing further
clarifications and collaboration.

%
% appC.tex - Appendix containing a reference to the DELVE utilities.
%
% $Id: appC.tex,v 1.48.2.6 1996/11/29 21:33:26 revow Exp $
%

%
% Some commonly used strings.
%

\newcommand{\tcl}{\textbf{Tcl}}
\newcommand{\dpath}{dpath}
\newcommand{\mpath}{mpath}
\newcommand{\lossfile}[2]{\texttt{loss\textit{.#1.#2}}}

%
% \synopsis{command}{description}
%
% A command for formatting ``synopsis'' descriptions consistently.
% First arg contains the command name, second arg contains the command
% description.
%

\newcommand{\synopsis}[2]{%
\newpage%
\subsection*{\rule{0pt}{20pt}\texttt{\textbf{#1}} --- #2}%
\addcontentsline{toc}{subsection}{%
\texttt{\textbf{\protect\makebox[11.5ex][l]{#1}}} #2}%
\rhead[\texttt{#1}]{\texttt{#1}}%
\lhead[\texttt{#1}]{\texttt{#1}}%
}

%
% \Usage{command}{args}
%
% Prints a command usage description. 

\newcommand{\oo}{$[\,$}
\newcommand{\oc}{$\,]$}
\newcommand{\alt}{\hspace{-1pt}{\tt |}}
\newcommand{\AND}{\hspace{8pt}}
\newcommand{\litr}[1]{\hspace{2pt}{\tt #1}\hspace{0pt}}

\newcommand{\Usage}[2]{%
	\begin{list}{}{}\item[]%
	{\tt #1}\hspace{8pt}{\em #2}%
	\end{list}}

%
% \begin{ttdescription} item_list \end{ttdescription}
%
% An environment like ``description'', but it typesets the label in
% typewriter font, and starts the description on a new line if the
% label is longer than than the label width.
%

\newlength{\Optionlen}
\newcommand{\Optionlabel}[1]{%
	\settowidth{\Optionlen}{\texttt{#1}}%
	\ifdim \Optionlen >\labelwidth%
		\parbox[b]{\labelwidth}{\makebox[0pt][l]{\texttt{#1}}\\\mbox{}}%
	\else%
		\texttt{#1}%
	\fi%
	\hfil\relax}
\newenvironment{ttdescription}%
	{\begin{list}{}{%
		\settowidth{\leftmargin}{\texttt{xxxxxxxx}}%
		\setlength{\labelwidth}{\leftmargin}%
		\addtolength{\labelwidth}{-\labelsep}%
		\renewcommand{\makelabel}{\Optionlabel}}}%
	{\end{list}}

%
% The body of the appendix
%

\newpage
\section{DESCRIPTIONS OF DELVE COMMANDS}\label{app-commands}
\thispagestyle{plain}
\setcounter{figure}{0}
\chead[\fancyplain{}{\thesection.\ DESCRIPTIONS OF DELVE COMMANDS}]
      {\fancyplain{}{\thesection.\ DESCRIPTIONS OF DELVE COMMANDS}}

This appendix is a detailed reference for the commands that make up
the \delve{} working environment.  You will not necessarily need to
use all these commands every day, as some of them are needed only by
people creating new datasets or prototasks.

\subsection*{Introduction to \delve{} commands}
\addcontentsline{toc}{subsection}{Introduction to \delve{} commands}
Before describing the individual \delve{} commands, we will describe
the common aspects that they all share. 

\subsubsection*{Command syntax}
All \delve{} utilities have a common calling syntax, along the lines of:

\Usage{command}{\oo option \oc \AND argument~\ldots}

Portions enclosed by ``$[$'' and ``$]$'' are optional; things before
``\ldots'' can be repeated several times.  A vertical bar, ``\texttt{|}'',
separates alternatives, only one of which should be present.

The \texttt{command} is the name of the \delve{} utility, for example
\dinfo{}.  Commands are named so that those that act on dataset
directories have names that begin with the letter `\texttt{d}', while
those that act on method directories have names beginning with
`\texttt{m}'.

The options are used to modify the behaviour of the \texttt{command}.
They take the usual Unix form --- a dash followed by a single
character, for example: ``\texttt{-h}''.  Some options also take a
single argument.  In this case the argument must immediately follow
the option, separated by one or more blank spaces, for example:
\mbox{``\texttt{-i foo}''}.  If the argument contains spaces it must be
quoted: \mbox{\texttt{-i "this arg has spaces"}}.

The arguments differ in number and meaning for each command.  However,
all commands recognize the two following options: 
\begin{ttdescription}
\item[-h]
    This causes the command to print a short help message describing
    its usage and options, after which it exits normally without
    doing anything else.
\item[--]
    This marks the end of the options for the command.  The arguments
    following this one will be treated as regular command arguments
    even if they start with a \texttt{-}.
\end{ttdescription}

\subsubsection*{Data and method path names}
Throughout this appendix, we refer to data path names, or \emph{dpaths} and
method path names, or \emph{mpaths}.  These paths look just like normal
unix path names, but they differ in two important aspects:\vspace{-4pt}
\begin{itemize}
\item
    Dpaths and mpaths are defined only for files that
    exist inside the \delve{} directory hierarchy.  Dpaths
    point into the \texttt{data} part of the hierarchy; mpaths
    point into the \texttt{methods} part.
\item
    A dpath or mpath may identify a file or directory in any
    of the active \texttt{delve} directories.  Dpaths and mpaths
    for directories can even resolve to several locations
    within different \texttt{delve} directories (though this is
    not supposed to happen for dpaths and mpaths identifying files).
\end{itemize}\vspace{-4pt}

The \delve{} hierarchy is the collection of all the active
\texttt{delve} directories. A \texttt{delve} directory must have a
name that starts with the five characters ``\texttt{delve}'', and it
must have two subdirectories called \texttt{data} and
\texttt{methods}.  \delve{} decides on the set of active
\texttt{delve} directories as follows.  First of all, if your current
working directory is inside a \texttt{\delve} directory, that
\texttt{delve} directory will be considered active, for as long as you
remain in it.  In addition, you may provide a list of \texttt{delve}
directories in your \texttt{DELVE\_PATH} environment variable.  If you
do not have such an environment variable, \delve{} will use a default
list of directories that was fixed when \delve{} was installed.

The \texttt{DELVE\_PATH} environment variable, if set, should
contain a colon separated list of \delve{} directories. You can use the
command ``\texttt{dinfo -k delve\_path /}'' to find out which directories
are currently in your \texttt{DELVE\_PATH}, or what the default list
of directories is, if you have not set your \texttt{DELVE\_PATH}.

All the files relating to \delve{} datasets, methods, and the results
of applying methods to data are kept in the \delve{} hierarchy. Files
relating to datasets, but not to any particular method, are stored in
the \texttt{data} part of the hierarchy, and hence have a \dpath.
Methods and the results of applying methods are stored in the
\texttt{methods} part of the hierarchy, and hence have an \mpath.  An
mpath that points to a file or directory relating to the results of a
method on a particular dataset, prototask, or task can also be used to
identify the corresponding information in the \texttt{data} part
of the hierarchy.

Dpaths and mpaths may be ``absolute'', starting with a ``/'' character,
or they may be specified relative to the current directory.

Some examples may clarify these naming conventions. Consider the case
where you have a directory called \texttt{/usr/local/lib/delve}.
Inside this directory are the directories \texttt{data} and
\texttt{methods} (and any number of other files and directories).
This is a valid \delve{} directory.

Inside the \texttt{data} directory there is another directory called
\texttt{demo}.  Its absolute unix path name is:
\texttt{/usr/local/lib/delve/data/demo}; however, its \dpath{} is
simply \texttt{/demo}.  It does not have a \mpath{} because it is not
in the \texttt{methods} directory.  If your current directory were
\texttt{demo}, its relative Unix path would be ``\texttt{.}'', as would
its relative \dpath.

Assume the \texttt{methods} directory contained
\texttt{lin-1/demo}. Its absolute unix path name would be:
\texttt{/usr/local/lib/delve/methods/lin-1/demo}; however, its
\mpath{} would be \texttt{/lin-1/demo}.  It would not have a \dpath{}
because it is not in the \texttt{data} directory.  If your current
directory were \texttt{lin-1/demo}, its relative unix path would be
``\texttt{.}''. This would also be its relative \mpath. Similarly, the
relative path name of \texttt{lin-1} in both schemes would be
``\texttt{..}''.

Finally, note that commands that create files inside a directory need
to know the \emph{true} pathname for the directory, \emph{not} just a
dpath or mpath, since the latter might resolve to more than Unix
directory.

\newpage

%
% The ``dcheck'' command
%

\synopsis{dcheck}{Validate \delve{} data files}

{\em Note: This command is not yet implemented.}

The \dcheck{} command is used to verify that the data and specifications
for a dataset and its prototasks are legal and consistent.

\subsubsection*{Command Summary}
\Usage{dcheck}{\oo \litr{-l}\oc \AND \oo dpath \alt mpath \oc}

The path given \dcheck{} must identify a dataset, a prototask for a
dataset, or a prior file for a prototask.  The default is
``\texttt{.}'', the current directory, which must identify a 
dataset or prototask.  If a dataset is specified, its
\texttt{Dataset.spec} and \texttt{Dataset.data} files are checked for
errors.  Unless `\texttt{-l}' is specified, all the prototasks for the
dataset are also checked.  A single prototask can be checked by giving
a path to that prototask.  When a prototask is checked, the
\texttt{Prototask.spec} file is checked for consistency with the
\texttt{Dataset.spec}, and, unless `\texttt{-l}' is specified, all the
\texttt{.prior} files for the prototask are also checked for errors.
A single \texttt{.prior} file can be checked by giving its pathname.

The \dcheck{} command recognizes the \texttt{-h} ``help'' option described in
the introduction to this section, as well as:
\begin{ttdescription}
\item[-l] 
    This causes \dcheck{} to run locally.  If a dataset is specified,
    only information on the dataset itself is checked, not information
    on its prototasks.  If a prototask is specified, only information on
    the prototask itself is checked, not information in prior files for the
    prototask.
\end{ttdescription}

%
% The ``dgenorder'' command
%

\synopsis{dgenorder}{Generate random order for a prototask}

The \dgenorder{} command is used to set up a random ordering of cases
in a prototask.  This will usually be necessary only for natural or
cultivated datasets, not for simulated or artificial ones, for which
the order will presumably already be random.

\subsubsection*{Command Summary}
\Usage{dgenorder}{\oo path\oc}

The argument given must be the true path name of a prototask directory
(\emph{not} the dpath for a prototask); the default is `.', the
current directory.  The \dgenorder{} command creates a file called
\texttt{Random-order} within this directory that contains a random
ordering of cases in the prototask.  This file has one line for each
case used by the prototask, with each line containing a number from
one up to the total number of cases in the prototask.

This \texttt{Random-order} file is meant to be used as the ordering
file in the specification for a prototask.  Use of a random ordering
is advisable whenever the prototask is not sequential (where the
ordering is meaningful), unless the ordering is already known for
certain to be random (as would often be the case for simulated and
artificial data).

{\em Note: For prototasks with commonality indexes, or for which
training sets are to be stratified, \dgenorder{} will have to do
something cleverer, but such things are not implemented yet.}

\subsubsection*{Files Used}
\begin{ttdescription}
\item[/{\rm\em dataset\/}/Dataset.spec]
    contains specifications describing the dataset with dpath
    \texttt{/{\rm\em dataset}}.
\item[/{\rm\em dataset\/}/Dataset.data]
    contains all the data for the dataset with dpath
    \texttt{/{\rm\em dataset}} in a \delve{} standard format.  
\item[/{\rm\em dataset\/}/{\rm\em prototask\/}/Prototask.spec]
    contains specifications describing the prototask with dpath {\tt
    /{\rm\em dataset\/}/{\rm\em prototask}}.
\item[/{\rm\em dataset\/}/{\rm\em prototask\/}/Random-order]
    this is the output file produced by the command.
\end{ttdescription}

%
% The ``dgenproto'' command
%

\synopsis{dgenproto}{Generate prototask data files}

Prototasks are composed of a subset of the cases in a dataset, and a
subset of the attributes in each case.  The prototask data file is an
intermediate file between the dataset data file and the task data
files.  It will be generated ``on the fly'' if it doesn't exist, but
some time will be saved each time it is needed if the creator (or
installer) of a prototask creates it once and for all, using the
\dgenproto{} command.  (On the other hand, keeping such prototask data 
files around permanently takes up disk space.)

\subsubsection*{Command Summary}
\Usage{dgenproto}{\oo \litr{-i}\oc \AND \oo path\oc}

The \dgenproto{} command generates prototask data files (called
\texttt{Prototask.data}) inside prototask directories found within the
\texttt{data} part of the \delve{} hierarchy.  The data put in these
files is taken from the corresponding dataset data files
(called \texttt{Dataset.data}), which are also found in the \delve{}
hierarchy.  

The \textit{path} argument is the true path of the directory to
generate the data in (\emph{not} its dpath, as the dpath might not
specify a unique directory).  If \textit{path} is a prototask
directory --- that is, a subdirectory of a \delve{} dataset directory
--- the prototask data file for that prototask alone is generated.  If
\textit{path} points to a dataset directory, data for all prototasks
in the dataset will be generated.  If \textit{path} points to the root
of the \texttt{data} part of the \delve{} hierarchy, data files for
all prototasks for all datasets will be generated.

The \dgenproto{} command recognizes the \texttt{-h} ``help'' option
described in the introduction to this section, as well as:
\begin{ttdescription}
\item[-i] 
    This option causes the command to ignore errors when multiple data
    files are being generated.  The command will continue even if one
    or more of the files cannot be created.
\end{ttdescription}

\subsubsection*{Example}

After obtaining the dataset \texttt{/demo} from the \delve{} archive
and placing it in the archive \delve{} directory
\texttt{/usr/local/lib/delve}, the installer will probably wish to run
the following command:

\begin{Session} 
unix>dgenproto -i /usr/local/lib/delve/data/demo generating:
/usr/local/lib/delve/data/demo/age/Prototask.data
  extracting cases...
  creating file...
generating: /usr/local/lib/delve/data/demo/colour/Prototask.data
  extracting cases...
  creating file...
generating: /usr/local/lib/delve/data/demo/income/Prototask.data
  extracting cases...
  creating file...
generating: /usr/local/lib/delve/data/demo/sex/Prototask.data
  extracting cases...
  creating file...
generating: /usr/local/lib/delve/data/demo/siblings/Prototask.data
  extracting cases...
  creating file...
\end{Session}

\subsubsection*{Files Used}
\begin{ttdescription}
\item[/{\rm\em dataset\/}/Dataset.spec]
    contains specifications describing the dataset with dpath
    \texttt{/{\rm\em dataset}}.
\item[/{\rm\em dataset\/}/Dataset.data]
    contains all the data for the dataset with dpath
    \texttt{/{\rm\em dataset}} in a \delve{} standard format.  
\item[/{\rm\em dataset\/}/{\rm\em prototask\/}/Prototask.spec]
    contains specifications describing the prototask with dpath {\tt
    /{\rm\em dataset\/}/{\rm\em prototask}}.
\item[/{\rm\em dataset\/}/{\rm\em prototask\/}/Prototask.data]
    this is the output file produced by the command.
\end{ttdescription}

\newpage

%
% The ``dinfo'' command
%

\synopsis{dinfo}{Get information about datasets}

Although \dls{} and \dmore{} can be used to browse through the
directories and files that define a dataset, the information the files
contain is not presented in a very useful format.  The \dinfo{}
command takes all of the information available and puts it into a more
accessible format.

\subsubsection*{Command Summary}
\Usage{dinfo}{\oo \litr{-a} \alt \litr{-k} keys\oc \AND
              \oo \litr{-q}\oc \AND
              \oo \litr{-t}\oc \AND
              \oo dpath \alt mpath\oc}

The \dinfo{} command prints human readable information summaries about
the \delve{} dataset, prototask or task whose data path is
\textit{dpath}.  If \dinfo{} is given a method path name instead of a
data path name, it converts it to a data path name by removing the
method prefix.  If not specified, the path defaults to `\texttt{.}',
which must be in the \delve{} heirarchy.

For different types of paths, \dinfo{} returns different types of
information. The \textit{dpath} argument may specify one of the
following:
\begin{itemize}
\item
    The root data directory `\texttt{/}'; for which the information
    available includes: the \delvepath{} and a list of all installed
    datasets.
\item
    A \textbf{dataset}, for which the available information includes:
    the name of the dataset; its origin; its recommended usage; the
    order cases occur in it; the number of attributes each case
    contains; a description of these attributes; and a list of all
    prototasks in the dataset. An example of a dataset path is
    `\texttt{/demo}'.
\item
    A \textbf{prototask}, for which the available information
    includes: the name of the prototask; its origin; the number of
    cases it contains; the ordering of these case; the number of cases
    in each test set; the sizes of the training sets for each task;
    the scheme used for generating test sets; the maximum number of
    training instances a task may contain; a list of the attributes to
    be used as inputs for tasks; a list of the attributes to be used
    as targets; and a list of the available tasks.  An example of a
    prototask path is `\texttt{/demo/age}'.
\item
    A \textbf{task}, for which the available information includes: the
    name of the task, the number of cases in each training set; a list
    of the attributes to be used as inputs; a list of the attributes
    to be used as targets; the type, relevance, and default coding
    method for each attribute.  An example of a task path is
    `\texttt{/demo/age/std.128}'.
\end{itemize}

The \dinfo{} command recognizes the \texttt{-h} ``help'' option described in
the introduction to this section, as well as:
\begin{ttdescription}
\item[-a] 
    Causes the command to print out all the information it knows about
    the path you are querying. By default, it only prints
    ``interesting'' information.
\item[-k \textit{keys}] 
    Print only information about fields in the \texttt{keys} list. Keys 
    can be obtained with the \texttt{-q} option. This is useful, for
    example, if you are only interested in what prototasks a dataset
    contains.
\item[-q] 
    Print, instead of the information normally printed, the keys for
    the information. For example the command \verb+dinfo -q /demo+
    would print:
\begin{Session}
dataset origin usage order number-of-attributes prototasks
\end{Session}
    These keys can be used as arguments for the \texttt{-k} option. Note
    that the \texttt{-k} and \texttt{-a} options affect the behaviour of
    this option, i.e.~\texttt{-q} causes the command to print the keys
    for the information it would print given all other options.
\item[-t] 
    Print information in a terse format: no headings are printed, and
    the format is more suitable as input to another program than to a
    human. The \texttt{-t} and \texttt{-k} options can conveniently be used
    together in scripts.
\item[-v] 
   Print the software version number.
\end{ttdescription}

\subsubsection*{Example}

An example of a command to obtain information about the \texttt{demo}
dataset is:
\begin{Session}
unix> dinfo /demo
Dataset: /demo
Origin: artificial
Usage: development
Order: uninformative
Number of attributes: 5
Prototasks: 
        age
        colour
        income
        sex
        siblings
\end{Session}
Similar results would be obtained if you were in a directory with
dpath \texttt{/demo} and you typed `\texttt{dinfo .}' or
`\texttt{dinfo}'.

If you only wanted to know what prototasks the dataset contained, and
you wanted the output to be machine readable, you could use the
command:
\begin{Session}
unix> dinfo -t -k prototasks /demo
age colour income sex siblings
\end{Session}

\subsubsection*{Files Used}
\begin{ttdescription}
\item[/{\rm\em dataset\/}/Dataset.spec]
    contains specifications describing the dataset with dpath
    \texttt{/{\rm\em dataset}}.
\item[/{\rm\em dataset\/}/Dataset.data]
    contains all the data for the dataset with dpath
    \texttt{/{\rm\em dataset}} in a \delve{} standard format.  The contents of
    the file are not used, but its existence may be checked.
\item[/{\rm\em dataset\/}/{\rm\em prototask\/}/Prototask.spec]
    contains specifications describing the prototask with dpath {\tt
    /{\rm\em dataset\/}/{\rm\em prototask}}.
\item[/{\rm\em dataset\/}/{\rm\em prototask\/}/*.prior]
    files contain prior information to be used when generating tasks for the 
    prototask with dpath \texttt{/{\rm\em dataset\/}/{\rm\em prototask}}.
\end{ttdescription}


%
% The ``dls'' command
%

\synopsis{dls}{List contents of \delve{} data directories}

In the \delve{} environment, if a given dpath refers to a directory,
it could resolve to multiple true directories.  This can be
inconvenient if you want to list the files contained in the dpath.  To
help in this situation, \delve{} supplies the \dls{} utility for
listing all files that reside in directories with a common dpath.

\subsubsection*{Command Summary}
\Usage{dls}{\oo \litr{-l}\oc \AND \oo dpath \alt mpath\oc}

The \dls{} command lists the merged contents of all directories with
the common data path name \textit{dpath}, or if \textit{dpath} refers
to a file, it repeats its name.  If \dls{} is given a method path name
(an mpath) as an argument instead of a data path name, it converts it
to a data path name by removing the method prefix.

The output of the command is sorted alphabetically.  If no path is
given on the command line, it defaults to `\texttt{.}', which must be
a dpath or an mpath.

The \dls{} command recognizes the \texttt{-h} ``help'' option described in
the introduction to this section, as well as:
\begin{ttdescription}
\item[-l] 
    Print a long listing, where files are grouped by the directory
    they are contained in, and the true path name of each directory is
    printed.
\end{ttdescription}

\subsubsection*{Example}

If you wished to list all files in the data directories with dpath
\texttt{/demo/age} you could use \dls{} as follows:
%
\begin{Session}
unix> dls /demo/age
Prototask.data  Prototask.spec  std.prior
\end{Session}
%
Since \dls{} allows you to give either a dpath or a mpath as an
argument, you could obtain the same results using the command
`\texttt{dls /lin-1/demo/age}'.

%
% The ``dmore'' command
%

\synopsis{dmore}{Browse or page through \delve{} data files}

In a manner similar to the \dls{} and \mls{} commands, \delve{}
provides a utility called \dmore{} for viewing text files given their
dpath.  This allows you to look at a file without knowing its true
path.

\subsubsection*{Command Summary}
\Usage{dmore}{dpath \alt mpath~\ldots}

The \dmore{} command displays the contents of text files that reside
in \delve{} data directories.  The \textit{dpath} arguments are the
data path names of the files to be displayed.  If \dmore{} is given
method path names instead of data path names, it converts them to data
path names by removing the method prefix.  Files are displayed on the
terminal, one screenful at a time.

To view the files, \dmore{} passes its output through a pager.  The
default pager is \texttt{more}, but it can be changed by setting the
environment variable \texttt{PAGER} to the name of the command you
wish to use.

The \dmore{} command recognizes only the \texttt{-h} ``help'' option
described in the introduction to this section.

\subsubsection*{Example}

To view the file containing the standard prior information for the
\texttt{demo/age} prototask, you could use the command:
\begin{Session}
unix> dmore /demo/age/std.prior
1 NLMH binary
3 NLMH integer
4 NLMH real
5 NLMH nominal
2 NLMH real
\end{Session}

%
% The ``mgendata'' command
%

\synopsis{mgendata}{Generate task data files}

Once you have created the directory hierarchy that will contain the
data to train and test your method on, you have to populate it with
the actual data.  To do this you use the command \mgendata.

\subsubsection*{Command Summary}
\Usage{mgendata}{\oo \litr{-c} file\oc \AND \oo \litr{-q} \oc \AND \oo path\oc}

The \mgendata{} command generates task data files inside a \delve{}
method directory from dataset or prototask data files inside a
\delve{} data directory.  The \textit{path} argument is the \emph{true}
path of the directory to generate the data in: \emph{not} its mpath
(since the mpath could easily resolve to multiple directories).  The
\textit{path} argument must be a subdirectory of a \delve{} methods directory.

If \textit{path} points to a task directory, only data for that task
will be generated. If it points to a prototask, data for all tasks in
the prototask will be generated.  If it points to a dataset, data for
all tasks in all prototasks will be generated.

In each task directory, four sets of files are generated.  Each set
contains the same number of files as there are training instances in
the task.  Each file in a set has a unique extension `\textit{.n}',
where \textit{n} is the integer index of the training instance which
the file corresponds to.\vspace{-5pt}

\begin{itemize}
\item 
    Each instance in the task has a \textbf{training file} called
    \file{train}{n}.  This file contains cases that are to be used for
    training your learning method.  Each line in the file contains the
    data for one case.  A case contains the encoded representation of
    all attributes to be used for the task (see
    Section~\ref{assess-encodings} for a description of encoding
    schemes) printed to the file such that all values are separated by
    white space.  In each case, input attributes come first, followed
    by target attributes (i.e.~each line contains both input values
    and target values).
\item
    As well as a training file, each instance in the task has a
    \textbf{testing file} called \file{test}{n}.  These files contain
    encoded input attributes for all testing cases (one case per line,
    all values separated by white space).  Testing files do \emph{not}
    contain target values; they contain only input values.
\item
    For each testing file, there is a corresponding \textbf{target
    file}.  This file contains the encoded target attributes for the
    testing cases, one case per line. Target files are called
    \file{targets}{n}.
\item
    Data attributes can be encoded using various forms of
    normalization.  To keep track of the normalization constants, a
    \textbf{normalization file} \file{normalize}{n} is created for
    each instance.  This file contains the mean, variance, median, and
    average absolute deviation from the median for each attribute (one
    attribute per line).
\end{itemize}\vspace{-5pt}

As well as the above sets of files, two single files are generated:
\texttt{Coding-used} which will contain a description of the method
used to encode each of the attributes (in the form described below),
and \texttt{Test-set-stats} which will contain statistics derived from
the testing data.  These files are needed to calculate the losses and
evaluate the method performance after it has been run.

The \mgendata{} command recognizes the \texttt{-h} ``help'' option described in
the introduction to this section, as well as:\vspace{-5pt}
\begin{ttdescription}

\item[-c {\rm\em file}] 
    This option allows you to override the default encoding of
    attributes.  The file should contain one encoding specification
    per line, containing first an identifier of an attribute (either
    number or name) followed by the desired encoding.  If options are
    to be given for the encoding, they should appear on the same line,
    in the form \texttt{option=value}, where \texttt{option} is the
    option's name (for example \texttt{passive}), and \texttt{value}
    is the value it is to be set to.  Multiple option/value pairs may
    appear on the same line, separated by spaces.  No spaces may
    appear between the option's name and the equal sign, or between
    the equal sign and the value.  All attributes which are not
    mentioned in the encoding file retain their default encodings.

    The valid encodings and their options are described in
    section~\ref{assess-encodings}.\vspace{-5pt}

\item[-q]
    Command should run quietly. Normally \mgendata{} prints the names of the
files that it is working on.


\end{ttdescription}

\subsubsection*{Example}

Suppose that you are in a directory whose mpath is \texttt{/lin-1},
and that you have previously run \mgendata.  If you now want to
generate training and testing files for the task of the
\texttt{/demo/age} prototask based on standard prior information and
using 256 cases in each training set, you would use the
command:\vspace{-5pt}
\begin{Session}
unix> mgendata ./demo/age/std.256
./demo/age/std.256
  segmenting cases...
  splitting test inputs and targets...
  encoding instance 0 training data...
  encoding instance 0 test inputs...
  encoding instance 0 test targets...
  encoding instance 1 training data...
  encoding instance 1 test inputs...
  encoding instance 1 test targets...
  encoding instance 2 training data...
  encoding instance 2 test inputs...
  encoding instance 2 test targets...
  encoding instance 3 training data...
  encoding instance 3 test inputs...
  encoding instance 3 test targets...
\end{Session}\vspace{-5pt}

\subsubsection*{Files Used}
\begin{ttdescription}
\item[/{\rm\em dataset\/}/Dataset.spec]
    contains specifications describing the dataset with dpath {\tt
    /{\rm\em dataset}}.
\item[/{\rm\em dataset\/}/Dataset.data]
    contains all the data for the dataset with dpath \texttt{/{\rm\em dataset}}
    in a \delve{} standard format.
\item[/{\rm\em dataset\/}/{\rm\em prototask\/}/Prototask.spec]
    contains specifications describing the prototask with dpath {\tt
    /{\rm\em dataset\/}/{\rm\em prototask}}.
\item[/{\rm\em dataset\/}/{\rm\em prototask\/}/Prototask.data]
    contains all the data for the prototask with dpath {\tt
    /{\rm\em dataset\/}/{\rm\em prototask}} in a \delve{} standard format.  
\item[/{\rm\em dataset\/}/{\rm\em prototask\/}/*.prior]
    files contain prior information to be used when generating tasks for the 
    prototask with dpath \texttt{/{\rm\em dataset\/}/{\rm\em prototask}}.
\item[/{\rm\em dataset\/}/{\rm\em prototask\/}/Random-Order]
    contains the ordering to use when extracting cases from the {\tt
    Dataset} file and generating the \texttt{Prototask} file for the
    prototask with dpath \texttt{/{\rm\em dataset\/}/{\rm\em prototask}}.
\item[/{\rm\em method\/}/{\rm\em dataset\/}/{\rm\em prototask\/}/{\rm\em 
    task\/}/train.\textit{n}]
    created to hold the encoded inputs and targets for training cases.
\item[/{\rm\em method\/}/{\rm\em dataset\/}/{\rm\em prototask\/}/{\rm\em 
    task\/}/test.\textit{n}]
    created to hold the encoded inputs for test cases.
\item[/{\rm\em method\/}/{\rm\em dataset\/}/{\rm\em prototask\/}/{\rm\em 
    task\/}/targets.\textit{n}]
    created to hold the encoded targets for test cases.
\item[/{\rm\em method\/}/{\rm\em dataset\/}/{\rm\em prototask\/}/{\rm\em 
    task\/}/normalize.\textit{n}]
    created to hold the normalization constants used in the encoding.
\item[/{\rm\em method\/}/{\rm\em dataset\/}/{\rm\em prototask\/}/{\rm\em 
    task\/}/Coding-used]
    created to hold coding actually used in creating the data files for \\
    \texttt{/{\rm\em dataset\/}/{\rm\em prototask\/}/{\rm\em task}}.
\item[/{\rm\em method\/}/{\rm\em dataset\/}/{\rm\em prototask\/}/{\rm\em 
    task\/}/Test-set-stats]
    created to hold statistics of the testing data for the task with dpath \\
    \texttt{/{\rm\em dataset\/}/{\rm\em prototask\/}/{\rm\em task}}.
\end{ttdescription}

%
% The ``mgendir'' command
%

\synopsis{mgendir}{Generate task directories}

When you first want to run a new method on a dataset, you must build
the directory tree that will contain all of the training and testing
data.  You could use a normal Unix command such as \texttt{mkdir}, but
that would be quite tedious, or you could use the \delve{} command
\mgendir.

\subsubsection*{Command Summary}
\Usage{mgendir}{\oo \litr{-l}\oc \AND \oo \litr{-q} \oc \AND \oo path\oc}

The \mgendir{} command generates directory trees for \delve{}
datasets, prototasks, or tasks inside a method directory.  The
\textit{path} argument is the \emph{true} path of the root of the tree
to create: \emph{not} its mpath (since the mpath could easily resolve to
multiple directories). The \textit{path} must be a subdirectory of a
\delve{} methods directory.

If path specifies a method, directories for all available datasets are
created in the method directory.  If it specifies a dataset,
directories for all prototasks and tasks of that dataset are
generated. If it specifies a prototask or a task, only directories
associated with them are generated.

\mgendir{} will not complain if parts of the directory tree already exist.

The \mgendir{} command recognizes the \texttt{-h} ``help'' option described in
the introduction to this section, as well as:
\begin{ttdescription}
\item[-l] 
    This causes \mgendir{} to run locally. This means that
    sub-directories are not created.  If you specify a method name, no
    dataset directories are generated.  If you specify a dataset name,
    no prototask directories are generated.  If you specify a
    prototask name, no task directories are generated.
\item[-q]
    Command should run quietly. Normally \mgendir{} prints the names of the
subdirectories as they are created.
\end{ttdescription}

\subsubsection*{Example}

Assuming that you were in a directory with mpath \texttt{lin-1}, and
you wanted to generate the directory tree files for the entire
\texttt{demo} dataset, you could use the command:
\begin{Session}
unix> mgendir ./demo
./demo
./demo/age
./demo/age/std.32
./demo/age/std.64
./demo/age/std.128
./demo/age/std.256
./demo/age/std.512
./demo/colour
./demo/colour/std.32

\ldots

./demo/siblings/std.512
\end{Session}
Similarly, you could generate the directories for just the \texttt{age}
prototask using the command:
\begin{Session}
unix> mgendir ./demo/age
./demo
./demo/age
./demo/age/std.32
./demo/age/std.64
./demo/age/std.128
./demo/age/std.256
./demo/age/std.512
\end{Session}

\subsubsection*{Files Used}
\begin{ttdescription}
\item[/{\rm\em dataset\/}/Dataset.spec]
    contains specifications describing the dataset with dpath {\tt
    /{\rm\em dataset}}.
\item[/{\rm\em dataset\/}/Dataset.data]
    contains all the data for the dataset with dpath \texttt{/{\rm\em dataset}}
    in a \delve{} standard format.  The contents of the file are not
    used by \dinfo\, but its existence is checked.
\item[/{\rm\em dataset\/}/{\rm\em prototask\/}/Prototask.spec]
    contains specifications describing the prototask with dpath {\tt
    /{\rm\em dataset\/}/{\rm\em prototask}}.
\item[/{\rm\em dataset\/}/{\rm\em prototask\/}/*.prior]
    files contain prior information to be used when generating tasks for 
    the prototask with dpath \texttt{/{\rm\em dataset\/}/{\rm\em prototask}}.
\end{ttdescription}

%
% The ``minfo'' commands
%

\synopsis{minfo}{Get information about learning methods}

In a manner similar to \dinfo\, the \minfo{} command can be used to
obtain information about \delve{} methods.

\subsubsection*{Command Summary}
\Usage{minfo}{\oo \litr{-a} \alt \litr{-k} keys\oc \AND
              \oo \litr{-q}\oc \AND
              \oo \litr{-t}\oc \AND
              \oo mpath\oc}

The \minfo{} command prints human readable information summaries about
the \delve{} method, dataset, prototask or task whose \delve{} method
path name is given by the \textit{mpath} argument. If no path is
specified, it defaults to `\texttt{.}', which must be a \delve{}
methods directory. The \minfo{} command returns information about
datasets, prototasks, or tasks \emph{as they were used by the method},
not as they appear in the \texttt{data} directory.  For example, when
the mpath argument specifies a dataset, the list of prototasks
returned by \minfo{} will contain only those the method was run on,
not all of the ones available to be run on.

For different types of paths, \minfo{} returns different types of
information. The \textit{mpath} argument may specify one of the
following:
\begin{itemize}
\item
    Information available for the root data directory \texttt{/}
    includes: the \delvepath{} and a list of the methods that have
    been run on \delve{} datasets.
\item
    For a \texttt{method} directory, the information available
    includes a list of all datasets the method has been run on. An
    example of a method path is `\texttt{/lin-1}'.
\item
    For a \texttt{dataset}, the available information includes all the
    information returned by \dinfo{} for datasets, with the exception
    that the list of prototasks includes only those that the method
    has been run on.  An example of a dataset path is
    `\texttt{/lin-1/demo}'.
\item
    For a \texttt{prototask}, the available information includes all the
    information returned by \dinfo{} for prototasks, with the exception
    that the list of tasks includes only those that the method has
    been run on.  An example of a prototask path is
    `\texttt{/lin-1/demo/age}'.
\item
    For a \texttt{task}, the available information includes all the
    information returned by \dinfo{} for a task, with the exception
    that the actual coding method used for the data is printed,
    \textit{not} the default method.  An example of a task path is
    `\texttt{/lin-1/demo/age/std.128}'.
\end{itemize}

The \minfo{} command recognizes the same options as \dinfo.

\subsubsection*{Example}

An example of a command line that could be used to obtain information
about the \texttt{demo} dataset as it was used by lin-1 would be:
\begin{Session}
unix> minfo /lin-1/demo
Dataset: /demo
Origin: artificial
Usage: development
Order: uninformative
Number of attributes: 5
Prototasks: 
        age
        income
\end{Session}
Similar results would be obtained if your current working directory
had the mpath \mbox{\texttt{/lin-1/demo}}, and you typed
`\texttt{minfo .}' or `\texttt{minfo}'.

If you only wanted to know what \texttt{demo} prototasks the \texttt{lin-1}
method was run on, and you wanted the output to be machine readable,
you could use the command:
\begin{Session}
unix> minfo -t -k prototasks /lin-1/demo
age income
\end{Session}

\subsubsection*{Files Used}
\begin{ttdescription}
\item[/{\rm\em dataset\/}/Dataset.spec]
    contains specifications describing the dataset with dpath
    \texttt{/{\rm\em dataset}}.
\item[/{\rm\em dataset\/}/Dataset.data]
    contains all the data for the dataset with dpath
    \texttt{/{\rm\em dataset}} in a \delve{} standard format.  The contents of
    the file are not used, but its existence may be checked.
\item[/{\rm\em dataset\/}/{\rm\em prototask\/}/Prototask.spec]
    contains specifications describing the prototask with dpath {\tt
    /{\rm\em dataset\/}/{\rm\em prototask}}.
\item[/{\rm\em dataset\/}/{\rm\em prototask\/}/*.prior]
    files contain prior information to be used when generating tasks for the 
    prototask with dpath \texttt{/{\rm\em dataset\/}/{\rm\em prototask}}.
\item[/{\rm\em method\/}/*]
    used to get the list of datasets the method with mpath {\tt
    /{\rm\em method}} has been run on.
\item[/{\rm\em method\/}/{\rm\em dataset\/}/*]
    used to get the list of prototasks from the dataset with dpath
    \texttt{/{\rm\em dataset}} that the method with mpath 
    \texttt{/{\rm\em method}} has been run on.
\item[/{\rm\em method\/}/{\rm\em dataset\/}/{\rm\em prototask\/}/*]
    used to get the list of tasks from the prototask with dpath {\tt
    /{\rm\em dataset\/}/{\rm\em prototask}} that the method with mpath 
    \texttt{/{\rm\em method}} has been run on.
\item[/{\rm\em method\/}/{\rm\em dataset\/}/{\rm\em 
       prototask\/}/{\rm\em task\/}/Coding-used]
    contains the coding scheme used to generate the task data files
    for the task with mpath \texttt{/{\rm\em method\/}/{\rm\em 
    dataset\/}/{\rm\em prototask\/}/{\rm\em task}}.
\end{ttdescription}

%
% The ``mloss'' command
%

\synopsis{mloss}{Generate task loss files}

Once you have run a method on a task and produced predictions, you
will need to calculate the loss from the true targets and your
predictions.  Loss functions are discussed in detail in
Section~\ref{sec-loss}.

\subsubsection*{Command Summary}
\Usage{mloss}{\oo \litr{-i} instances\oc \AND 
              \oo \litr{-l} losses\oc \AND \oo \litr{-q} \oc \AND
              \oo path\oc}

The \mloss{} command decodes prediction files and generates loss
files.  The \textit{path} argument for \mloss{} is the \emph{true}
path of the directory to generate the decoded prediction and loss
files in (\emph{not} its mpath, since the mpath could easily resolve
to multiple directories).  The prediction files used to generate the
losses can reside in any directory with the same mpath.  If not
specified, \textit{path} defaults to the current directory.

If \textit{path} points to a task directory, only loss files for that
task will be generated.  If it points to a prototask, loss files for
all tasks in the prototask will be generated.  If it points to a
dataset, loss files for all tasks in all prototasks will be generated.
Finally, if it points to a method, loss files will be generated for
all tasks that the method has been run on.

The \mloss{} command can generate losses using using any of the five
following measures.  Each measure has a single-character code
associated with it:\vspace{-4pt}
\begin{list}{}{\setlength{\leftmargin}{0.6in}\setlength{\labelsep}{0.2in}}
  \item[{\tt A}] Absolute error loss.
  \item[{\tt S}] Squared error loss.
  \item[{\tt Z}] Zero-one loss.
  \item[{\tt L}] Negative log-probability loss.
  \item[{\tt Q}] Squared-probability loss.
\end{list}\vspace{-4pt}
You must write your predictions to files with special names in order
to get them evaluated with the intended loss measure. Depending on the
type of the prediction, the file may have one of three \emph{root}
names: \texttt{guess} for files that contain guesses for the targets,
\texttt{prob} for files containing class probabilities and \texttt{ptarg}
for files containing the probabilities (or densities) of the true
targets under the method's predictive distribution. In general the
methods do not need to read the \texttt{targets} files, with the
exception of the situations where the method produces a predictive
distribution and the targets are real, integer or angular. In these
cases there seems to be no general convenient way of conveying the
predictive distribution; instead the method must itself evaluate the
probability (or density) of the true target under the predictive
distribution and write this to a prediction file with the
\texttt{ptarg} root name.

A number of prefixes and extensions may be added to these root
names. The instance number is always added as an extension,
e.g.~\texttt{guess.3}. Optionally, the name of a spcific loss function
can also be specified as an extension, e.g.~\texttt{guess.S.3}.  If no
loss function is specified then the predictions can be applied to any
loss function for which that particular root is meaningful (although,
loss specific predictions always take precedence over generic
ones). Some prediction files may have a `\texttt{c}' prefixed to their
name, indicating that the predictions are in the coded domain. This
will normaly be the case for the files your method writes, since it
only sees that training and test files which have been encoded. The
`\texttt{c}' prefix can be applied to files with the \texttt{guess} or
\texttt{ptarg} root names. Files which contain probabilities (or
densities) may have an `\texttt{l}' prefixed their name indicating
that the predictions are made in the (natural) log domain. Some
examples of names of prediction files are \texttt{cguess.A.0},
\texttt{prob.3} and \texttt{clptarg.L.7}. Note, that the prefixes and
extensions must follow the order given in these examples.

The first task \mloss{} performs is decoding the predictions.  It
places the decoded predictions in files with the same names as those
containing the coded ones with the initial `\texttt{c}' removed. For
example the decoded predictions for \texttt{cguess.A.0} would be placed
in \texttt{guess.A.0}.  Similarly, the decoded predictions for
\texttt{clptarg.7} would be placed in \texttt{lptarg.7}.

Once the predictions have been decoded, \mloss{} generates loss files
based on those predictions and the target values.  The losses are
placed in files named \file{loss}{l.n}, where where the `\textit{l}'
and `\textit{n}' characters have the same meanings as above. For loss
files the `.\textit{l}' extension is not optional (as the values in
the file are defined by the loss function).

The prediction files used to generate the losses for a particular
measure are found by first looking for all prediction files specific
to that loss (i.e.~files that have the appropriate `.\textit{l}'
extension). If even one such file exists for a given measure, then
only files with that extension are used to generate the losses.  If no
such files exist for the given loss, \mloss{} looks for prediction
files where the loss was not specified (i.e.~files with the
appropriate root name, but without the `.\textit{l}' extension). It
then uses these to calculate the loss. If none of these files exist,
a warning message is printed, and no loss files for that measure are
generated.

A table of the allowed combinations of target types and loss functions
is given in section \ref{loss-standard}. Whenever predictions are made
in files with the \texttt{prob} root, \mloss{} automatically
normalises the probabilites to sum to unity. However, this is not
possible for predictions with the \texttt{ptarg} root, so users should
be careful to ensure that their method's predictive distribution is
correctly normalised when using these predictions.

The \mloss{} command recognizes the \texttt{-h} ``help'' option
described in the introduction to this section, as well
as:\vspace{-4pt}
\begin{list}{}{\setlength{\leftmargin}{0.6in}\setlength{\labelsep}{0.2in}}
\item[{\tt -i} {\rm\em instances}] 
    This allows you to specify which training instances you want to
    evaluate the loss for.  It should be a list of integer values or
    the string `\texttt{all}'.  It's default value is `\texttt{all}'.
\item[{\tt -l} {\rm\em losses}] 
    This allows you to specify the loss functions \mloss{} attempts to
    evaluate. You can specify any combination of \texttt{A}, \texttt{S},
    \texttt{L}, \texttt{Q}, and \texttt{Z}. By default, \mloss{} attempts to
    evaluate all approriate types.
\item[{\tt -q}]
    Command should run quietly. Normally \mloss{} prints the names of the
files that it is working on.
\end{list}

\emph{Note: The \mloss{} command does not yet support the specialised loss
functions discussed in section \ref{loss-specialized}.}

To generate the loss files, \mloss{} must temporarily decode the
target files.  Because of this, the target files \emph{must} be
present in the \mpath{} of the task.

\subsubsection*{Files Used}
\begin{ttdescription}
\item[/{\rm\em dataset\/}/Dataset.spec]
    contains specifications describing the dataset with dpath {\tt
    /{\rm\em dataset}}.
\item[/{\rm\em dataset\/}/Dataset.data]
    contains all the data for the dataset with dpath \texttt{/{\rm\em dataset}}
    in a \delve{} standard format.  The contents of the file are not
    used by \dinfo\, but its existence is checked.
\item[/{\rm\em dataset\/}/{\rm\em prototask\/}/Prototask.spec]
    contains specifications describing the prototask with dpath {\tt
    /{\rm\em dataset\/}/{\rm\em prototask}}.
\item[/{\rm\em dataset\/}/{\rm\em prototask\/}/*.prior]
    files contain prior information to be used when generating tasks for 
    the prototask with dpath \texttt{/{\rm\em dataset\/}/{\rm\em prototask}}.
\item[/{\rm\em method\/}/{\rm\em dataset\/}/{\rm\em prototask\/}/{\rm\em 
       task\/}/targets.{\sl n}]
    contains the coded targets for the \texttt{n}'th training instance of the 
    task with dpath \texttt{/{\rm\em dataset\/}/{\rm\em prototask\/}/{\rm\em 
    task\/}/}, as made by the method with mpath \texttt{/{\rm\em method}}.
\item[/{\rm\em method\/}/{\rm\em dataset\/}/{\rm\em prototask\/}/{\rm\em 
       task\/}/{[c]}guess{[{\sl .l\/}]}{\sl .n}]
    contains the guesses for the (optianally coded) targets of the
    \texttt{n}'th training instance of the task with dpath
    \texttt{/{\rm\em dataset\/}/{\rm\em prototask\/}/{\rm\em 
    task\/}/}, as made by the method with mpath \texttt{/{\rm\em method}},
    with an optionally specified loss function (\texttt{A}, \texttt{S} or
    \texttt{Z}).
\item[/{\rm\em method\/}/{\rm\em dataset\/}/{\rm\em prototask\/}/{\rm\em 
       task\/}/{[c][l]}ptarg{[.L]}{\sl .n}]
    contains the (optionally log) probabilities (or densities) for the
    (optionally coded) targets of the \texttt{n}'th training instance of
    the task with dpath
    \texttt{/{\rm\em dataset\/}/{\rm\em prototask\/}/{\rm\em task\/}/}, 
    as made by the method with mpath \texttt{/{\rm\em method}}. The
    \texttt{L} loss function may optionally be specified.
\item[/{\rm\em method\/}/{\rm\em dataset\/}/{\rm\em prototask\/}/{\rm\em 
       task\/}/{[l]}prob{[{\sl .l\/}]}{\sl .n}]
    contains the (optionally log) probabilities of the targets for the
    \texttt{n}'th training instance of the task with dpath
    \texttt{/{\rm\em dataset\/}/{\rm\em prototask\/}/{\rm\em task\/}/}, 
    as made by the method with mpath \texttt{/{\rm\em method}}, with
    an optionally specified loss function (\texttt{L} or \texttt{Q}).
\item[/{\rm\em method\/}/{\rm\em dataset\/}/{\rm\em prototask\/}/{\rm\em 
       task\/}/Coding-used]
    contains the coding scheme used to encode the data for the method with
    mpath
    \texttt{/{\rm\em method\/}/{\rm\em dataset\/}/{\rm\em prototask\/}/{\rm\em 
    task}}.
\item[/{\rm\em method\/}/{\rm\em dataset\/}/{\rm\em prototask\/}/{\rm\em 
       task\/}/normalize.{\sl n}]
    contains the normalizing constants used to encode the data for the
    method with mpath 
    \texttt{/{\rm\em method\/}/{\rm\em dataset\/}/{\rm\em prototask\/}/{\rm\em 
    task\/}/}
\end{ttdescription}

%
% The ``mls'' command
%

\synopsis{mls}{List contents of \delve{} method directories}

As with dpaths, if a given mpath refers to a directory, it could
resolve to multiple true directories.  To list all files in
directories with a common mpath, \delve{} supplies the \mls{} utility.

\subsubsection*{Command Summary}
\Usage{mls}{\oo \litr{-l}\oc \AND \oo mpath\oc}

The \mls{} command lists the merged contents of all directories with
the common method path name \texttt{mpath}, or if \texttt{mpath}
refers to a file, it repeats its name.

The output of the command is sorted alphabetically.  If no path is
given on the command line, it defaults to `\texttt{.}', which must be
a \delve{} method directory.

The \mls{} command recognizes the \texttt{-h} ``help'' option described in
the introduction to this section, as well as:
\begin{ttdescription}
\item[-l] 
    Print a long listing, where files are grouped by the directory
    they are contained in, and the true path name of each directory is
    printed.
\end{ttdescription}

\subsubsection*{Example}

If you wished to list all files in the method directories with mpath
\texttt{/lin-1/demo} you could use the \mls{} command as follows:
%
\begin{Session}
unix> mls /lin-1/demo
age income
\end{Session}

%
% The ``mmore'' command
%

\synopsis{mmore}{Browse or page through \delve{} method files}

The command corresponding to \dmore{} for viewing \delve{} method
files is \mmore.

\subsubsection*{Command Summary}
\Usage{mmore}{mpath~\ldots}

The \mmore{} command displays the contents of text files that reside
in \delve{} method directories. The \texttt{mpath} arguments are the
method path names of the files to be displayed.  Files are displayed
on the terminal, one screenful at a time.

To view the files, \mmore{} passes its output through a pager.  The
default pager is \texttt{more}, but it can be changed by setting the
environment variable \texttt{PAGER} to the name of the command you
wish to use.

The \mmore{} command recognizes only the \texttt{-h} ``help'' option
described in the introduction to this section.

\subsubsection*{Example}

To view the source program for the implementation of the \texttt{lin-1}
method, you could use the command:
\begin{Session}
unix> mmore /lin-1/Source/lin-1.c
/* lin-1.c: Robust linear method for regression.  
 *
 * Reads training examples from "train.n", test inputs from "test.n" and
 * targets from "targets.n".  Produces point predictions in "cguess.n" and
 * densities of targets under a predictive distribution in "clptarg.L.n".
 * Here "n" is the instance number, supplied as a command argument. Handles
 * badly conditioned cases where inputs are (close to) linearly dependent.
 *
 * (c) Copyright 1996 by Carl Edward Rasmussen. */

#include <stdio.h>
...
\end{Session}

%
% The ``mstats'' command
%

\synopsis{mstats}{Calculate or compare loss statistics}

Once the loss files for a given method have been generated, you can
see how well or poorly the method performed, either in absolute terms,
or in comparison to another method.  This is done with the \mstats{}
command.

\subsubsection*{Command Summary}
\Usage{mstats}{\oo \litr{-c} methods\oc \AND
               \oo \litr{-i} base\oc \AND
               \oo \litr{-l} losses\oc \AND
               \oo mpath\oc}

The \mstats{} command prints summary statistics about a method's loss
files, or compares the loss files of two methods, and prints summary
statistics about the comparison.  The \textit{mpath} argument is the
\delve{} method path name of the method whose losses are to be
summarized.  If it is omitted, it defaults to `\texttt{.}'  which must
be a \delve{} method directory.

Loss files are generated by \mloss{} and are normally named
\file{loss}{l.n}, where \textit{l} is a single character describing
the loss function used to generate the file, and \textit{n} is an
integer describing the training instance the loss file corresponds to.
See the description of \mloss{} for further details.

Full details of the statistics used to summarize the losses are
described in Section~\ref{sec-analysis}; however a quick summary is
given here.

When summarizing the loss files for a single method, \mstats{}
returns:\vspace{-5pt}
\begin{itemize}
\item the estimated expected loss.
\item the standard error of the estimate.
\item the standard deviation of the losses between training sets.
\item the standard deviation of the losses between testing cases 
	(if applicable).
\item the standard deviation of the residuals.
\end{itemize}\vspace{-5pt}
Both the raw values and a standardized version of these terms are
printed.  See section~\ref{analysis-mstats} for a discussion of the
standardization used.

When comparing the performance of two methods, \mstats{} returns:\vspace{-5pt}
\begin{itemize}
\item the estimated expected loss for both methods.
\item the estimated expected difference in the losses.
\item the standard error of the estimate for the difference.
\item the standard deviation of the losses between training sets.
\item the standard deviation of the residuals.
\end{itemize}\vspace{-5pt}
Both the raw values and a standardized version of these are printed.
The report also includes a probability describing the significance of
the differences of the two loss estimates (calculated using either a
T-test, or F-test as appropriate).

Both reports include a listing of how many training sets and cases,
and test sets and cases were used to calculate the statistics.

The \mstats{} command recognizes the \texttt{-h} ``help'' option described in
the introduction to this section, as well as:\vspace{-4pt}
\begin{list}{}{\setlength{\leftmargin}{0.6in}\setlength{\labelsep}{0.2in}}
\item[\texttt{-c} {\rm\em methods}] 
    This flag causes \mstats{} to compare the current method with the
    selected methods. Summary statistics about the differences of the
    loss files in \texttt{mpath} and those of the other methods are
    returned.  The \texttt{method} arguments may be proper mpaths for
    a method, e.g.~\texttt{/lin-1}, or you may omit the initial slash.
\item[\texttt{-i} {\rm\em base}] 
    This allows you to change the base name of the input loss files.
    These files are generated by \mloss{} and normally have the base
    name \texttt{loss}.
\item[\texttt{-l} {\rm\em losses}] 
    This allows you to change what loss functions \mstats{} attempts
    to summarize. You can specify any combination of \texttt{A},
    \texttt{S}, \texttt{L}, \texttt{Q}, and \texttt{Z}. By default, it
    attempts to summarize all of them, using whatever files exist.
    \texttt{Mstats} will only print warnings if it attempts to
    summarize a loss measure for which there are no loss files.
\end{list}\vspace{-4pt}

\subsubsection*{Example}

Suppose you wished to know how well the \texttt{/lin-1} method did on
the \texttt{/demo/age/std.128} task, using the squared error loss
measure.  You could use the command:\vspace{-5pt}
\begin{Session}
unix> mstats -l S /lin-1/demo/age/std.128
/lin-1/demo/age/std.128
Loss: S (Squared error)
                                                    Raw value   Standardized

                         Estimated expected loss:     400.73      0.819745
                     Standard error for estimate:    28.6111     0.0585277

     SD from training sets & stochastic training:     40.898     0.0836622
SD from test cases & stoch. pred. & interactions:    790.029       1.61611

    Based on 8 disjoint training sets, each containing 128 cases and
             8 disjoint test sets, each containing 128 cases.
\end{Session}\vspace{-5pt}
If you then wanted to compare its performance to the \texttt{/knn-cv-1}
method, you could use:\vspace{-5pt}
\begin{Session}
mstats -c knn-cv-1 -l S /lin-1/demo/age/std.128
/lin-1/demo/age/std.128
Loss: S (Squared error)
                                                    Raw value   Standardized

               Estimated expected loss for lin-1:     400.73      0.819745
           Estimated expected loss for /knn-cv-1:    368.003      0.752798
                   Estimated expected difference:     32.727     0.0669473
          Standard error for difference estimate:     14.075     0.0287922

     SD from training sets & stochastic training:    27.6978     0.0566594
SD from test cases & stoch. pred. & interactions:    323.515      0.661792

    Significance of difference (t-test), p = 0.052988

    Based on 8 disjoint training sets, each containing 128 cases and
             8 disjoint test sets, each containing 128 cases.
\end{Session}\vspace{-5pt}

\subsubsection*{Files Used}
\begin{ttdescription}
\item[/{\rm\em dataset\/}/Dataset.spec]
    contains specifications describing the dataset with dpath
    \texttt{/{\rm\em dataset}}.
\item[/{\rm\em dataset\/}/Dataset.data]
    contains all the data for the dataset with dpath
    \texttt{/{\rm\em dataset}} in a \delve{} standard format.  The contents of
    the file are not used, but its existence is checked.
\item[/{\rm\em dataset\/}/{\rm\em prototask\/}/Prototask.spec]
    contains specifications describing the prototask with dpath {\tt
    /{\rm\em dataset\/}/{\rm\em prototask}}.
\item[/{\rm\em dataset\/}/{\rm\em prototask\/}/*.prior]
    files contain prior information to be used when generating tasks for the 
    prototask with dpath \texttt{/{\rm\em dataset\/}/{\rm\em prototask}}.
\item[/{\rm\em method\/}/{\rm\em dataset\/}/{\rm\em prototask\/}/{\rm\em 
    task\/}/loss.\textit{l}.\textit{n}]
    contains the losses for the \texttt{n}'th training instance of the
    task calculated using the loss function \texttt{\textit{l}} with
    dpath \texttt{/{\rm\em dataset\/}/{\rm\em prototask\/}/{\rm\em task\/}/}, 
    as made by the method with mpath \texttt{/{\rm\em method}}.
\item[/{\rm\em method\/}/{\rm\em dataset\/}/{\rm\em prototask\/}/{\rm\em 
    task\/}/Test-set-stats]
    contains statistics of the testing data for the task with dpath
    \texttt{/{\rm\em dataset\/}/{\rm\em prototask\/}/{\rm\em task}}.
\end{ttdescription}

\newpage

\rhead[]{}%
\lhead[]{}%

%
% $Id: appD.tex,v 1.20 1996/05/08 00:04:59 radford Exp $
%
\newpage

\section{GLOSSARY OF DELVE TERMINOLOGY}\label{app-glossary}
\thispagestyle{plain}
\setcounter{figure}{0}
\chead[\fancyplain{}{\thesection.\ GLOSSARY OF DELVE TERMINOLOGY}]
      {\fancyplain{}{\thesection.\ GLOSSARY OF DELVE TERMINOLOGY}}

\vspace{4pt}

\small
\begin{list}{}{%
\setlength{\itemsep}{0in}%
\setlength{\leftmargin}{2.25in}%
\setlength{\labelsep}{0in}%
\setlength{\labelwidth}{2.25in}}

\item[\bf absolute-error loss \hfill]
A loss function for regression tasks in which the loss is the
absolute value of the difference between the guess and the
target.  When there is more than one target, the absolute loss 
is the sum of such absolute differences for all the targets.

\item[\bf angular attribute \hfill]
An attribute whose value is an angle or some other circular
quantity, such as time-of-day.  By default, such attributes
are encoded as the sine and the cosine of the equivalent angle,
so as to avoid introducing an artificial discontinuity.

\item[\bf artificial dataset/prototask \hfill]
A dataset generated by a program (usually with a random component)
on the basis of some mathematical specification, without any
connection with a real-world problem.  Prototasks based on 
such datasets are also referred to as artificial.

\item[\bf assessment dataset \hfill]
A dataset that is recommended for use in formally assessing
learning methods.

\item[\bf attribute \hfill]
One of the quantities associated with each case in a dataset.
The dataset specification classifies attributes as {\bf controlled} 
or {\bf uncontrolled}, according to how their values were determined.  
The prior information for a task will characterize attributes
as {\bf binary}, {\bf nominal}, {\bf ordinal}, {\bf integer}, {\bf angular},
or {\bf real}.

\item[\bf binary attribute \hfill] 
A categorical attribute that can take on exactly two possible values 
(not counting missing values) --- for example, an attribute with possible 
values of ``male'' and ``female'', or one with values of ``0'' and ``1''.

\item[\bf categorical attribute \hfill]
An attribute that takes on values from some finite set.  The targets for
a classification prototask must be categorical.  The prior information for 
a task further characterizes categorical attributes as {\bf binary},
{\bf nominal}, or {\bf ordinal}, and may designate one of the values
as {\bf passive}.

\item[\bf case \hfill]
A collection of attribute values that all apply to the same thing.  For
example, in a dataset of medical tests on patients, a case might consist
of all the test results for a particular patient.

\item[\bf censored value \hfill]
An indication of the value for an attribute in a case that says
only that the value is known to be greater than or equal to (or less
than or equal to) a specified value.  In DELVE dataset files, a
censored value is recorded as ``{\em number\/}:'' (if the actual value is
greater than or equal to {\em number\/}), or as ``:{\em number\/}'' (if the 
actual value is less than or equal to {\em number\/}).

\item[\bf classification prototask/task \hfill]
A prototask (or task) in which all the target attributes are
{\bf categorical}.

\item[\bf controlled attribute \hfill]
An attribute whose values were fixed by the investigators who gathered
the data.  For example, the amount of fertilizer applied to an agricultural
test plot would likely be a controlled attribute.

\item[\bf common testing scheme \hfill]
An experimental set-up in which a single common test set is used to
assess the performance of a method with all the training sets; distinguished
from a {\bf hierarchical} testing scheme.

\item[\bf commonality index \hfill] 
An integer that may be associated with a case, indicating that the
case has something in common with the other cases with the same
commonality index.  For example, in a dataset where a case records
features of a handwritten digit, all the digits written by one person 
might have the same commonality index.

\item[\bf cultivated dataset/prototask \hfill]
A dataset that comes from a real-world source, but has no real-world
context, having been collected or selected for the purpose of 
creating a DELVE dataset rather than from any genuine interest.  Natural
datasets that have been modified in some way, such as by adding extra
noise, are also in this class.  Prototasks based on cultivated datasets are 
also classified as cultivated, as are prototasks that are based on natural 
datasets but which have little resemblance to the original purpose for 
which the data was gathered.

\item[\bf dataset \hfill] 
A collection of data, consisting of a number of {\bf cases}, each
associated with the values of several {\bf attributes}.  Datasets are
classified as {\bf natural}, {\bf cultivated}, {\bf simulated}, or
{\bf artificial} according to the data's relationship to the real
world.  DELVE also distinguishes among {\bf development datasets},
{\bf assessment datasets}, and {\bf historical datasets}, on the basis
of recommended usage.

\item[\bf default encoding \hfill]
The encoding of an attribute that DELVE will use by default
if a particular learning method does not specify otherwise.
The default encoding is based on the {\bf prior information}
for the task.

\item[\bf dependency (between cases) \hfill]
A situation where knowledge of the values of the targets in one case
would be informative regarding the values of the targets in other
cases with the same {\bf commonality index}, or that are nearby in a
{\bf sequential prototask}.  Here, it is assumed that the inputs in all cases
are already known, and that the true nature of the general relationship 
between inputs and targets is also fully understood --- ie, the dependency 
is between the ``noise'' or ``residuals'' in the related cases (the
part of the variation not explainable by the relationship between inputs
and targets).

\item[\bf development dataset \hfill]
A dataset that is recommended for use in developing learning methods.
To avoid bias, such datasets should not also be used in formal assessments
of performance.

\item[\bf encoding (of an attribute) \hfill]
The way that DELVE represents the value of an attribute (usually as one or
more numbers) when generating data files for task instances.  The 
encoding to use is part of the specification
of a learning method, but DELVE provides a 
{\bf default encoding} that will often be appropriate.

\item[\bf estimated expected loss \hfill]
An estimate for the expected loss of a learning method on some task,
based on the results of a learning experiment.  At present, DELVE's
estimates are simply the average loss over training sets and test cases
tried.  Each estimate has an associated {\bf standard error}, that is
indicative of its likely accuracy.

\item[\bf expected loss \hfill]
The expected performance of a {\bf learning method} on some {\bf task} 
as judged by a specified {\bf loss function}, the expectation being with 
respect to random selection of a training set and a test case.  Put another 
way, the performance the method would achieve on average if it were
applied a great many times to training sets and test cases obtained from
the same source as the actual dataset.  Note that the true expected loss cannot 
be determined exactly, but an {\bf estimated expected loss} can be computed 
from the results of a learning experiment.

\item[\bf guess (for a test case) \hfill]
A prediction for the targets in a test case consisting of a single
value for each target, these values being chosen by the learning 
method with the aim of minimizing the expected {\bf absolute-error},
{\bf squared-error}, or {\bf 0-1 loss}.  If a {\bf no-guess penalty}
has been specified, a learning method also has the option of
making no guess for a particular target in a particular test case.

\item[\bf hierarchical testing scheme \hfill]
An experimental set-up in which separate, non-overlapping test sets are
used to assess the performance of a method as trained on different
training sets; distinguished from a {\bf common} testing scheme.

\item[\bf historical dataset \hfill]
A dataset that is included in the DELVE archive because it has been used
to assess learning methods in the past, but which is not recommended
for future use, except when there is a need to make comparisons with
past results in the literature.

\item[\bf input attribute \hfill]
For a particular {\bf prototask}, an attribute that is available
for use in predicting the values of the {\bf target attributes} in
the same case, but whose values do not themselves need to be predicted.

\item[\bf informative ordering \hfill]
An ordering of cases in a dataset (as originally obtained)
that conveys information that may be significant --- for instance,
an ordering of data on patients by date of admission to hospital.

\item[\bf integer attribute \hfill]
An attribute whose values are integers, and for which the
prior information does not specify an interpretation as
a {\bf categorical attribute}.  Note that the range of
an integer attribute may be restricted (eg, to the positive
integers).

\item[\bf learning experiment \hfill]
An experiment in which the performance of one or more 
{\bf learning methods} on one or more {\bf tasks} is assessed 
by applying the learning methods to several {\bf task instances}.
DELVE defines a standard scheme for conducting such experiments.

\item[\bf learning method \hfill]
A well-defined procedure for discovering relationships among
attributes on the basis of prior information and empirical data, and 
for making predictions for new cases using the relationships 
learned.  Learning can be {\bf supervised} or {\bf unsupervised}.

\item[\bf log-probability loss \hfill]
A loss function used with methods whose predictions are predictive
distributions over target values.  The log-probability loss is minus the 
log (base $e$) of the probability or probability density of the target
values.  This loss function can be used with any task, but for tasks
with real-valued targets (such as regression tasks), the loss must be 
computed by the learning method itself, rather than by DELVE.

\item[\bf loss function \hfill]
A measure of how far off a prediction is, given the actual values of
the targets. The standard loss functions DELVE supports are
{\bf squared-error loss}, {\bf absolute-error loss}, {\bf 0-1 loss}, 
{\bf squared-probability loss}, and {\bf log-probability loss}.  Specialized
loss functions can also be constructed that incorporate a 
{\bf no-guess penalty}, or that are based on a {\bf loss matrix}.

\item[\bf loss matrix \hfill]
For a prototask with one categorical target, a matrix that specifies 
the loss that is suffered for every possible combination of a guessed
value for the target and an actual value for the target.  For each
actual value of the target, the loss suffered when no guess is made
may also be specified.

\item[\bf missing value \hfill]
An indicator that the actual value of an attribute for a particular case
is not known.  In DELVE dataset files, a missing value starts with
a question mark; this may be followed by other characters to distinguish
values that are missing for different reasons.

%\item[\bf mysterious ordering \hfill]
%An ordering of cases in a dataset (as originally obtained) 
%that does not appear to be arbitrary, but whose significance
%(if any) cannot be determined from the available documentation.

\item[\bf natural dataset/prototask \hfill]
A dataset that comes from a real-world source, and for which there
is or was a real interest in learning relationships among the
attributes (for either scientific or engineering purposes).  A
prototask is classified as natural if it is based on a natural
dataset, and involves learning relationships that were of 
interest to the original investigators.

\item[\bf no-guess penalty \hfill]
The loss suffered when a learning method whose predictions take
the form of guesses decides to make no guess for a particular target
in a particular case.

\item[\bf nominal attribute \hfill]
A categorical attribute with at least three possible values (not 
counting missing values) for which the prior information does
not specify any natural ordering of the values.  An example
might be an attribute with values of ``beef'', ``pork'', and ``lamb''.

\item[\bf non-standard task instance \hfill]
A task instance in which the training and test sets are not
selected according the standard DELVE scheme.

\item[\bf noise level (for a target) \hfill]
The proportion of the variation in a target attribute that 
is not explained by the variation in the input attributes, even
given full knowledge of the true relationship between inputs and
targets.

\item[\bf order (of a dataset) \hfill] 
An indicator of whether the order of cases in the dataset (as originally
obtained) is {\bf informative} or {\bf uninformative}.

\item[\bf ordinal attribute \hfill]
A categorical attribute with at least three possible values (not
counting missing values) for which the prior information specifies a
natural ordering of the values.  An example might be an attribute with
values of ``no-education'', ``primary-education'', ``secondary-education'', 
and ``post-secondary-education''.

\item[\bf p-value (for a comparison) \hfill]
When comparing the estimated expected loss of two learning methods
on some task, the probability that a difference in estimated expected 
loss of equal or greater magnitude than that observed might arise by chance 
even if the true expected loss for the two methods is the same.  
A low p-value may give one confidence that the apparently better 
method actually is better.

\item[\bf passive value \hfill]
A value for a categorical attribute that is expected on the basis
of prior information to play a role different from that of the other
value or values of the attribute, with the passive value being
associated with a lack of positive influence.  If a
binary attribute has values of ``hockey-player'' and ``not-a-hockey-player'',
for example, ``not-a-hockey-player'' might be regarded as passive.

\item[\bf performance (of method) \hfill]
In the DELVE context, usually the predictive performance of the
method on some task, formalized in terms of {\bf expected loss}.
One might also be interested in the computational performance of a 
method (its time and memory requirements).

\item[\bf prediction (for a test case) \hfill]
The output of a learning method for a test case, embodying
the method's prediction regarding the likely values of the
targets in this case.  Predictions may be either single-valued 
{\bf guesses} for the target values, or {\bf predictive distributions} 
that say how likely each of the possible target values is.

\item[\bf predictive distribution \hfill]
A probability distribution produced by a learning method as its
prediction for the values of the targets in a test case.
For classification tasks, the predictive distribution consists of
a finite number of probabilities, which may be output in explicit form.
For tasks with real targets, the predictive distribution consists of
a probability density function, which DELVE does not attempt to
represent explicitly; instead, the learning method itself calculates 
the {\bf log-probability loss} based on its internal representation of 
the predictive distribution.

\item[\bf prior information \hfill]
Information regarding the the possible or likely nature of
the relationship being learned that is obtained from the
prior knowledge of the investigator (or a surrogate for the
investigator), rather than from the data itself.  

\item[\bf prototask \hfill]
A supervised learning problem associated with a {\bf dataset},
consisting of a set of {\bf target attributes} that are to be predicted, 
a set of {\bf input attributes} that may be used in making predictions, 
and a pool of {\bf cases} that are seen by the learning method.  A 
prototask can have many associated {\bf tasks}, in
which the available prior information  and the size of the training set are
also specified.  Prototasks are classified as {\bf natural}, {\bf cultivated}, 
{\bf simulated}, or {\bf artificial} according to their relationship to 
the real world.  {\bf Regression}
and {\bf classification} prototasks are distinguished by the nature of
their target attributes.

\item[\bf range (of attribute)\hfill]
The set of {\bf values} that an attribute could conceivably take on, including
the set of {\bf missing values} that are allowed for the attribute.

\item[\bf real attribute \hfill]
An attribute whose values are real numbers, and for which the prior 
information does not specify an interpretation as an {\bf angular},
{\bf integer}, or {\bf categorical attribute}.  Note that the range of 
a real attribute may be restricted (eg, to some interval).

\item[\bf relevance (of an input) \hfill]
The degree to which variation in an input attribute (within its
observed range) affects the values of the target attributes.
Put another way, the degree to which knowledge of the
input attribute's value helps in predicting the values of the
targets, given that the true nature of the relationship between
inputs and targets is known.

\item[\bf regression prototask/task \hfill]
A prototask (or task) in which all the targets attributes are
{\bf real}.

\item[\bf sequential prototask \hfill]
A prototask based on a dataset with an {\bf informative ordering}
in which this ordering has been preserved, and in which there may
therefore be {\bf dependencies} between nearby cases.

\item[\bf simulated dataset/prototask \hfill]
A dataset generated by a program (usually with a random component)
that simulates some actual phenomenon in a realistic fashion.
Prototasks based on such datasets are also referred to as simulated.

\item[\bf squared-error loss \hfill]
A loss function for regression tasks in which the loss is the
square of the difference between the guess and the
target.  When there is more than one target, the squared-error loss 
is the sum of such squared differences for all the targets.

\item[\bf squared-probability loss \hfill]
A loss function for classification tasks, used with methods 
whose predictions are predictive distributions over target values.
The squared-probability loss is the square of one minus the probability 
assigned to the correct target value, plus the sum of the squares of 
the probabilities assigned to all the other target values.  Squared-probability
loss cannot be used when there is more than one target attribute.

\item[\bf standard error (of estimate) \hfill]
The standard deviation of an estimate (eg, of expected loss) that
would be observed if the experiment on which the estimate is based
were to be repeated many times with new data randomly obtained
from the same source as the actual data.  (In practice, the
standard errors quoted are themselves estimates, since the 
true standard deviation usually depends on unknown quantities.)

\item[\bf standard task instance \hfill]
One of the task instances that are used in DELVE's standard scheme for learning 
experiments.

\item[\bf stratified training set \hfill]
A training set for a classification task in which training cases
have been selected in such a way that each of the different possible
target values appears the same number of times.

\item[\bf supervised learning \hfill]
Learning whose goal is to discover the relationship of certain
{\bf target attributes} to other {\bf input attributes}, and
on this basis predict the values of the target attributes for
a new case for which only the input attributes are known.

\item[\bf target attribute \hfill]
For a particular {\bf prototask}, an attribute whose values are to 
be predicted, based on the values of other {\bf input attributes} 
in the same case. 

\item[\bf task \hfill]
A specific learning context for a {\bf prototask}, consisting of the 
{\bf prior information} regarded as being 
available for use in learning, and the size and nature of the {\bf training set}
that will be provided.  A task is sufficiently
well specified that each learning method has a well-defined {\bf expected
loss} for a given task and loss function.  A task may be associated with
many {\bf task instances}, in which particular training sets and 
test cases are specified.

\item[\bf task instance \hfill]
A particular {\bf training set} for a {\bf task}, to which a learning method
can be applied as part of a learning experiment, together with a {\bf test
set} that is used to evaluate the accuracy of the learning method's
predictions. In DELVE's scheme for learning experiments, a set of {\bf standard
task instances} are defined; it is possible to define {\bf non-standard task
instances} as well.

\item[\bf test case \hfill]
A {\bf case} that is used to evaluate the performance of a 
learning method applied to a particular {\bf task instance}.

\item[\bf test set \hfill]
The set of all {\bf test cases} for a particular {\bf task instance}.
Note that although a task instance will normally include many test
cases, the predictions for the targets in each test case are to be made 
without using information from any other test case.

\item[\bf training case \hfill]
A {\bf case} that is part of the {\bf training set} made available
to a learning method.

\item[\bf training set \hfill]
The set of {\bf training cases} that are made available to a 
learning method in a particular {\bf task instance}.

\item[\bf uncontrolled attribute \hfill]
An attribute whose values were not fixed by the investigators who gathered
the data, but by some random process.  For example, the amount of rainfall
on various agricultural test plots would be an uncontrolled attribute 
(even though the investigators influence the amount of rainfall by where
they decide to put the plots).

\item[\bf uninformative ordering \hfill]
An ordering of cases in a dataset (as originally obtained) 
that does not convey any useful information --- for instance, 
a random ordering, or an ordering that is sorted by the value
of one of the attributes.

\item[\bf unsupervised learning \hfill]
Learning whose goal is to discover the relationships amongst
all attributes, without distinguishing some attributes as ``inputs''
and others as ``targets''.  DELVE does not currently
handle methods for unsupervised learning, but may do so in future.

\item[\bf value (of an attribute) \hfill]
The actual numerical or non-numerical quantity taken on by 
an {\bf attribute} in a particular {\bf case}.  Some cases may
have attributes with {\bf missing values}, for which the actual value is
not known, or with {\bf censored values}, for which the actual value is
known only to be beyond some given value.

\item[\bf 0-1 loss \hfill]
A loss function for classification tasks in which the loss is 0
when a guess matches the actual target value and 1 when the guess 
does not match the actual target value.  When there is more than one 
target, the total loss is the number of mis-matches between guesses 
and actual values.

\end{list}


\end{document}
